\documentclass[
	12pt,
	a4paper,
	cleardoubleempty, 
	idxtotoc,
	english,
	openright
	final,
	listof=nochaptergap,
	]{scrbook}
\usepackage{cmap}
\usepackage[T1]{fontenc}
\usepackage[utf8]{inputenc}
\usepackage{graphicx}

\include{preamble}

\begin{document}

\setcounter{secnumdepth}{3}

% Titelblatt
\begin{titlepage}
    \pagestyle{empty}
    
    \begin{center}	
    
    \includegraphics[height=3cm]{content/pictures/hm.png}
    
    \vspace{1cm}

    \docFakultaet\\
    \docStudiengang\\

    \vspace{3cm}
    {\huge \docArtDerArbeit}\\

    \vspace{2cm}
    {\Huge \docTitle}\\
    \vspace{0.5cm}
    %{\selectfont \docUntertitle}

    \vspace{2cm}
    Betreuer: \docErsterReferent\\
    %Co-Supervisor: \docZweiterReferent\\

    \end{center}

    \vspace{2cm}
    \begin{flushright}
    Eingereicht von: \\
    \docVorname~\docNachname, \docMatrikelnummer\\
    \docStrasse,~\docPlz~\docOrt\\
    \docEmail\\


    \vspace{0.5cm}
    Eingereicht am:\\
    \docOrt, den \today
    \end{flushright}

    \end{titlepage}
\cleardoubleemptypage

\frontmatter
\pagenumbering{roman}

% Abstract
\chapter*{Abstract\markboth{Abstract}{}}
\addcontentsline{toc}{chapter}{Abstract}

[Deutsches Abstract (100-120 Worte)]
\cleardoubleemptypage

% Inhaltsverzeichnis 
\tableofcontents
\cleardoubleemptypage

\mainmatter

\chapter{Einleitung}

In einer Zeit, in der datengetriebene Ansätze und automatisierte Modelle immer größere Relevanz erlangen, 
rückt die Notwendigkeit der Erklärbarkeit und Interpretierbarkeit von Modellen in den Vordergrund. 
Eines der vielversprechendsten Konzepte, das sich dieser Herausforderung annimmt, sind die sogenannten Shapley-Werte. 
Diese Masterarbeit erkundet die tiefgreifenden Konzepte der Shapley-Werte, ihre Anwendungen im Kontext von Machine Learning-Modellen, 
insbesondere linearer Modelle und ihre praktische Umsetzung auf reale Datensätze.

Die Arbeit beginnt mit einer umfassenden Einführung in die Shapley-Werte und ihre historischen Wurzeln. 
Dabei wird insbesondere auf die kooperative Spieltheorie als Ursprung dieser Konzepte eingegangen. 
Anhand ausgewählter Literatur werden die theoretischen Grundlagen erörtert.

Im Anschluss daran werden die Shapley-Werte im Kontext des maschinellen Lernens erweitert. 
Es wird beleuchtet, wie die Shapley-Werte adaptiert werden können, um Einblicke in die Gewichtung von Merkmalen in komplexen 
Machine Learning-Modellen zu gewinnen. Dabei wird auf bestehende Methoden und Ansätze Bezug genommen.

Ein zentraler Schwerpunkt der Arbeit liegt auf der praktischen Anwendung der Shapley-Werte. Ein realer Datensatz wird 
vorgestellt und die Methodik wird auf diesen angewendet, um die Wirksamkeit und Aussagekraft der Shapley-Werte in der 
Praxis zu evaluieren.

Abschließend werden die gewonnenen Erkenntnisse zusammengeführt und ein Ausblick auf zukünftige Entwicklungen und 
die Limitierungen der Shapley-Werte aufgezeigt.
\chapter{Hintergrund}


\section{Kooperative Spieltheorie}

Der Ursprung der Shapley Values liegt in der kooperativen Spieltheorie, einem fundamentalen Zweig der Spieltheorie. Dieser Bereich beschäftigt sich mit der Analyse von Situationen, in denen Akteure zusammenarbeiten, um gemeinsame Ziele zu erreichen. Zentrales Anliegen ist dabei die gerechte Verteilung der entstehenden Gewinne unter den Akteuren. Ein Schlüsselkonzept dieser Theorie ist die sogenannte "Charakteristische Funktion", welche die Bewertung der Gewinnverteilung einer Koalition von Akteuren ermöglicht.

Die Shapley Values, entwickelt von Lloyd Shapley in den 1950er Jahren, bieten einen methodischen Ansatz, um den individuellen Beitrag eines jeden Akteurs zur kooperativen Zusammenarbeit gerecht zu bewerten. Dies geschieht durch die Durchschnittsbewertung der Beiträge über sämtliche mögliche Koalitionen hinweg. Diese Methode erweist sich als äußerst nützlich, um eine gerechte und rationale Verteilung von Gewinnen in vielfältigen Szenarien zu ermöglichen, sei es in wirtschaftlichen Verhandlungen oder der Aufteilung von Ressourcen.

Das Verständnis der kooperativen Spieltheorie und ihrer Anwendung in Form der Shapley Values ermöglicht es, dieses theoretische Konzept auf den Bereich des maschinellen Lernens zu übertragen. In dieser Arbeit werden wir den Übergang von abstrakten Spieltheorie-Konzepten zu konkreten Anwendungen in der Welt der datengetriebenen Modelle erforschen.

Zur Erreichung dieses Ziels werden in den kommenden Abschnitten nicht nur die formalen Definitionen und Eigenschaften der Shapley Values erläutert, sondern auch ihre Adaption und Anwendung auf Machine Learning-Modelle in Betracht gezogen. Die Anwendbarkeit wird durch die praktische Anwendung auf einen realen Datensatz verdeutlicht.

\section{Formale Definition}

Sei $\mathcal{N} = \{1, \ldots, n\}$ eine endliche Spielermenge mit $n := |\mathcal{N}|$ Elementen. Sei $v$ die \textbf{Koalitionsfunktion}, die jeder Teilmenge von $\mathcal{N}$ eine reele Zahl zuweist und insbesondere der leeren Koalition den Wert $0$ gibt. 

\[
\begin{array}{rcccl}
  v &:  &\mathcal P(\mathcal{N}) &\longrightarrow &\mathbb{R}\\
  &: &v(\emptyset) &\mapsto &0\\
\end{array}
\]

Eine nicht leere Teilmenge der Spieler $\mathcal{S} \subseteq \mathcal{N}$ heißt Koalition. $\mathcal{N}$ selbst bezichnet die große Koalition. Den Ausdruck $v(\mathcal{S})$ nennt man den Wert der Koalition $\mathcal{S}$.
Der Shapley-Wert ordnet nun jedem Spieler aus $\mathcal{N}$ eine Auszahlung für das Spiel $v$ zu.

Der marginale Beitag eines Spieler $i \in N$, also der Wertbeitrag eines Spielers zu einer Koalition $\mathcal{S} \subseteq \mathcal{N}$, durch seinen Beitritt, ist

\begin{equation*}
v(\mathcal{S} \cup \{i\}) - v(\mathcal{S}).
\end{equation*}

Der Shapley-Wert eines Spielers $i$ errechnet sich als das gewichtete Mittel der marginalen Beiträge zu allen möglichen Koalitionen:

\begin{equation*}
\varphi_i (\mathcal{N}, v) = \sum_{\mathcal{S} \subseteq \mathcal{N} \setminus \{i\}} \underbrace{\frac{|\mathcal{S}|! \cdot (n - 1 - |\mathcal{S}|)!}{n!}}_{\text{Gewicht}} \underbrace{v(\mathcal{S} \cup \{i\}) - v(\mathcal{S})}_{\substack{\text{marginaler Beitrag von} \\ \text{Spieler $i$ zur Koalition $\mathcal{S}$}}}.
\end{equation*}

\section{Eigenschaften}

\paragraph{Pareto-Effizienz}

Der Wert der großen Koalition wird an die Spieler verteilt:

\begin{equation*}
\sum_{i \in \mathcal{N}} \varphi_i (\mathcal{N}, v) = v(\mathcal{N}).
\end{equation*}

\paragraph{Symmetrie}

Zwei Spieler $i$ und $j$, die die gleichen marginalen Beiträgen zu jeder Koalition haben,

\begin{equation*}
v(\mathcal{S} \cup \{i\}) = v(\mathcal{S} \cup \{j\})
\end{equation*}

erhalten das Gleiche:

\begin{equation*}
\varphi_i (\mathcal{N}, v) = \varphi_j (\mathcal{N}, v).
\end{equation*}

\paragraph{Null-Spieler-Eigenschaft}

Ein Spieler der zu jeder Koalition nichts bzw. den Wert seiner Einer-Koalition beiträgt, erhält null bzw. den Wert seiner Einer-Koalition:

\begin{equation*}
\varphi_i (\mathcal{N}, v) = 0,
\end{equation*}

bzw.

\begin{equation*}
\varphi_i (\mathcal{N}, v) = v(\{i\}).
\end{equation*}

\paragraph{Additivität}

Wenn das Spiel in zwei unabhängige Spiele zerlegt werden kann, dann ist die Auszahlung jedes Spielers im zusammengesetzten Spiel die Summe der Auszahlungen in den aufgeteilten Spielen:

\begin{equation*}
\varphi_i (\mathcal{N}, v + w) = \varphi_i (\mathcal{N}, v) + \varphi_i (\mathcal{N}, w).
\end{equation*}


\chapter{Theorie der Shapley-Werte}

In diesem Kapitel werden die Shapley-Werte als Instrument zur gerechten Aufteilung von Gewinnen in 
kooperativen Spielen vorgestellt. Durch die Verwendung eines praktischen Beispiels – der Aufteilung 
eines Preisgeldes aus einem Designwettbewerb unter den Gewinnern – wird zunächst eine intuitive Einführung 
in das Konzept gegeben. Anschließend wird die formale Definition der Shapley-Werte erläutert, um die 
theoretischen Grundlagen für ihre Berechnung und Anwendung zu legen.

\section{Wie lässt sich der Gewinn gerecht aufteilen?}

Angenommen, drei Teilnehmer, Anna, Ben und Carla, haben als Team kooperiert und den ersten Platz bei einem Designwettbewerb belegt\footnote{In Anlehnung an das Beispiel aus Kapitel 4.1 \glqq{}Who's going to pay for that taxi?\grqq{} \cite[S.17-20]{Molnar_2023}.}. 
Dieser Erfolg führt zu einem Gesamtgewinn von 1000 \euro. Das Preisgeld für den zweiten Platz beträgt 750 \euro{} und 500 \euro{} für den dritten Platz.
Die Herausforderung besteht nun darin, den Gewinn auf eine Weise zu verteilen, die den individuellen Beitrag jedes Teilnehmers 
zur Erzielung des ersten Platzes gerecht widerspiegelt.

Die Situation wird komplizierter, wenn man bedenkt, dass jeder Teilnehmer unterschiedlich zu dem Erfolg 
beigetragen hat und ihre individuellen Leistungen auch zu verschiedenen Ausgängen geführt hätten, 
wenn sie alleine oder in anderen Teilkonstellationen angetreten wären.

Um eine faire Aufteilung des Preisgeldes zu erreichen, betrachten wir die hypothetischen Gewinne, 
die Anna, Ben und Carla erzielt hätten, wenn sie in unterschiedlichen Konstellationen am Wettbewerb teilgenommen hätten.
Tabelle \ref{tab:shapley_example} zeigt die gegegbene Gewinnverteilung der verschiedenen Koalitionen. Die Koalition $\emptyset$ entspricht
dabei der leeren Koalition -- der Nichtteilnahme an dem Wettbewerb.

\begin{table}[h]
  \caption{Potenzielle Gewinne für verschiedene Teilnehmerkonstellationen im Designwettbewerb.}
  \footnotesize
  \begin{tabularx}{\textwidth}{Xrr}
  \toprule
  Koalition & Gewinn & Bemerkung \\
  \midrule
  $\emptyset$ & 0 \euro & Keine Teilnahme \\
  $\{$Anna$\}$ & 500 \euro & 3. Platz als Einzelteilnehmerin \\
  $\{$Ben$\}$ & 750 \euro & 2. Platz als Einzelteilnehmer \\
  $\{$Carla$\}$ & 0 \euro & Kein Gewinn als Einzelteilnehmerin \\
  $\{$Anna, Ben$\}$ & 750 \euro & 2. Platz als Team ohne Carla \\
  $\{$Anna, Carla$\}$ & 750 \euro & 2. Platz als Team ohne Ben \\
  $\{$Ben, Carla$\}$ & 500 \euro & 3. Platz als Team ohne Anna \\
  $\{$Anna, Ben, Carla$\}$ & 1000 \euro & 1. Platz als Gesamtteam \\
  \bottomrule
  \end{tabularx}
  \label{tab:shapley_example}
  \normalsize\\
  Quelle: Eigene Darstellung.
\end{table}

Zur Berechnung der Shapley-Werte ist es erforderlich, den marginalen Beitrag jedes Spielers zu erfassen.
Marginalbeiträge in der Spieltheorie, und speziell im Kontext der Shapley-Werte, sind die zusätzlichen Beiträge, 
die ein Spieler (Teilnehmer) zum Gesamtgewinn einer Koalition beiträgt, wenn er dieser beitritt. 
Die Berechnung des marginalen Beitrags eines Teilnehmers erfolgt, indem man den Wert der Koalition ohne diesen Teilnehmer 
vom Wert der Koalition mit dem Teilnehmer subtrahiert \cite[S. 18]{Molnar_2023}.

In diesem Beispiel mit Anna, Ben und Carla, die an einem Designwettbewerb teilnehmen, ist der marginale Beitrag von 
Anna zur Koalition von $\{$Ben$\}$ der zusätzliche Wert, den sie einbringt, wenn sie sich Ben anschließt, 
ausgehend von Bens individuellem Gewinn.

\begin{table}[h]
  \caption{Marginalbeiträge der einzelnen Teilnehmer zu den möglichen Koalitionen.}
  \footnotesize
  \begin{tabularx}{\textwidth}{XXrrr}
  \toprule
  Teilnehmer & Zur Koalition & Gewinn vorher & Gewinn nachher & Marginalbeitrag \\
  \midrule
  Anna & $\emptyset$ & 0 \euro & 500 \euro & 500 \euro \\
  Anna & $\{$Ben$\}$ & 750 \euro & 750 \euro & 0 \euro \\
  Anna & $\{$Carla$\}$ & 0 \euro & 750 \euro & 750 \euro \\
  Anna & $\{$Ben, Carla$\}$ & 500 \euro & 1000 \euro & 500 \euro\\
  Ben & $\emptyset$ & 0 \euro & 750 \euro & 750 \euro \\
  Ben & $\{$Anna$\}$ & 500 \euro & 750 \euro & 250 \euro\\
  Ben & $\{$Carla$\}$ & 0 \euro & 500 \euro & 500 \euro\\
  Ben & $\{$Anna, Carla$\}$ & 750 \euro &  1000 \euro & 250 \euro\\
  Carla & $\emptyset$ & 0 \euro & 0 \euro & 0 \euro \\
  Carla & $\{$Anna$\}$ & 500 \euro & 750 \euro & 250 \euro \\
  Carla & $\{$Ben$\}$ & 750 \euro & 500 \euro & -250 \euro \\
  Carla & $\{$Anna, Ben$\}$ & 750 \euro & 1000 \euro & 250 \euro \\
  \bottomrule
  \end{tabularx}
  \label{tab:shapley_marginal}
  \normalsize\\
  Quelle: Eigene Darstellung.
\end{table}

Die Tabelle \ref{tab:shapley_marginal} illustriert den Gewinn jeder möglichen Koalition ohne den 
betrachteten Spieler und den neuen Gesamtgewinn, sobald dieser Spieler der Koalition beitritt. 
Der marginale Beitrag jedes Spielers wird dann als die Differenz zwischen diesen beiden Werten 
berechnet und gibt Aufschluss über den individuellen Wertbeitrag zum gemeinschaftlichen Erfolg.

Nachdem die marginalen Beiträge jedes Teilnehmers für die verschiedenen Koalitionen festgestellt wurden, 
ist der nächste Schritt, die Shapley-Werte zu bestimmen, welche eine faire Aufteilung des Gesamtgewinns 
erlauben. Hierzu wird jede mögliche Reihenfolge (Permutation) betrachtet, in der die Spieler der 
Koalition beitreten könnten. Jede dieser Permutationen liefert unterschiedliche marginale Beiträge 
für die Spieler, je nach der Reihenfolge ihres Beitritts \cite[S. 307ff]{Shapley+1953+307+318}.

Im Falle dieses Beispiels mit Anna, Ben und Carla bedeutet dies, dass alle möglichen Reihenfolgen 
berücksichtigt werden müssen, in denen sie zum ersten Platz beigetragen haben könnten. 
Die Shapley-Werte werden dann als Durchschnitt der marginalen Beiträge über alle Permutationen berechnet. 
Dies gewährleistet, dass jeder Spieler einen Anteil des Preisgeldes erhält, der seinem durchschnittlichen 
Beitrag zum Erfolg entspricht.

Bei drei Teilnehmern exisitieren $3! = 3 \cdot 2 \cdot 1 = 6$ Permutationen:

\begin{enumerate}[itemsep=0pt, parsep=0pt]
  \item Anna, Ben, Carla
  \item Anna, Carla, Ben
  \item Ben, Anna, Carla
  \item Carla, Anna, Ben
  \item Ben, Carla, Anna
  \item Carla, Ben, Anna
\end{enumerate}

Jede Permutation entspricht einer Koalitionsbildung. Anna wird in zwei Koalitionsbildungen (1. und 2.) einer leeren Koalition hinzugefügt.
In weiteren zwei Koalitionsbildungen (5. und 6.) wird Anna der bestehenden Koalition aus Ben und Carla hinzugefügt. 
In den beiden übrigen Koalitionsbildungen wird Anna einmal der Koaliton bestehend aus Ben (3.) und einmal der Koalition bestehend aus Carla (4.)
hinzugefügt. 

Daraus lässt sich nun der Shapley-Wert mit den gewichteten durchschnittlichen marginalen Beiträge für Anna berechnen:

\begin{equation}
  \frac{1}{6} ( \underbrace{2 \cdot 500 \text{\euro}}_{\text{A $\rightarrow$ $\{\emptyset$\}}} + \underbrace{1 \cdot 0 \text{\euro}}_{\text{A $\rightarrow$ $\{B$\}}} + \underbrace{1 \cdot 750 \text{\euro}}_{\text{A $\rightarrow$ $\{C$\}}} + \underbrace{2 \cdot 500 \text{\euro}}_{\text{A $\rightarrow$ $\{B, C$\}}} ) \approx 458,34 \text{\euro}  
\label{eq:marginal_anna}
\end{equation}

Analog gilt das für Ben:

\begin{equation}
  \frac{1}{6} ( \underbrace{2 \cdot 750 \text{\euro}}_{\text{B $\rightarrow$ $\{\emptyset$\}}} + \underbrace{1 \cdot 250 \text{\euro}}_{\text{B $\rightarrow$ $\{A$\}}} + \underbrace{1 \cdot 500 \text{\euro}}_{\text{B $\rightarrow$ $\{C$\}}} + \underbrace{2 \cdot 250 \text{\euro}}_{\text{B $\rightarrow$ $\{A, C$\}}} ) \approx 458,34 \text{\euro}  
  \label{eq:marginal_ben}
\end{equation}

und Carla:

\begin{equation}
  \frac{1}{6} ( \underbrace{2 \cdot 0 \text{\euro}}_{\text{C $\rightarrow$ $\{\emptyset$\}}} + \underbrace{1 \cdot 250 \text{\euro}}_{\text{C $\rightarrow$ $\{A$\}}} + \underbrace{1 \cdot (-250) \text{\euro}}_{\text{C $\rightarrow$ $\{B$\}}} + \underbrace{2 \cdot 250 \text{\euro}}_{\text{C $\rightarrow$ $\{A, B$\}}} ) \approx 83,34 \text{\euro}  
  \label{eq:marginal_carla}
\end{equation}


Auf Basis der gewichteten durchschnittlichen marginalen Beiträge lässt sich feststellen, 
dass Anna und Ben jeweils einen Shapley-Wert von ungefähr 458,34 \euro{} erhalten, 
während Carla einen Shapley-Wert von etwa 83,34 \euro{} zugewiesen bekommt. 
Diese Werte spiegeln den fairen Anteil jedes Teilnehmers an der Gesamtprämie wider, 
basierend auf ihrem individuellen Beitrag zum Erfolg des Teams. Mit dieser konkreten Anwendung der Shapley-Werte 
auf ein alltagsnahes Beispiel wird nun die zugrunde liegende Theorie und die formale Definition 
der Shapley-Werte, die diese Berechnungen ermöglichen, detaillierter betrachtet.

\section{Formale Definition}

Sei $\mathcal{N} = \{1, \ldots, n\}$ eine endliche Spielermenge mit $n := |\mathcal{N}|$ Elementen. Sei $v$ die Koalitionsfunktion, die jeder Teilmenge von $\mathcal{N}$ eine reele Zahl zuweist und insbesondere der leeren Koalition den Wert $0$ gibt. 

\[
\begin{array}{rcccl}
  v &:  &\mathcal P(\mathcal{N}) &\longrightarrow &\mathbb{R}\\
  &: &v(\emptyset) &\mapsto &0\\
\end{array}
\]

Eine nicht leere Teilmenge der Spieler $\mathcal{S} \subseteq \mathcal{N}$ heißt Koalition. $\mathcal{N}$ selbst bezichnet die große Koalition. Den Ausdruck $v(\mathcal{S})$ nennt man den Wert der Koalition $\mathcal{S}$.
Der Shapley-Wert ordnet nun jedem Spieler aus $\mathcal{N}$ eine Auszahlung für das Spiel $v$ zu.

Der marginale Beitag eines Spieler $i \in \mathcal{N}$, also der Wertbeitrag eines Spielers zu einer Koalition $\mathcal{S} \subseteq \mathcal{N}$, durch seinen Beitritt, ist

\begin{equation}
v(\mathcal{S} \cup \{i\}) - v(\mathcal{S}).
\label{eq:marignal}
\end{equation}


Sei $i = \text{Anna}$ und $\mathcal{S} = \{\text{Ben}\}$, dann ist $v(\mathcal{\{\text{Ben}\}} \cup \{\text{Anna}\}) - v(\mathcal{\{\text{Ben}\}})$ das
zusätzliche Preisgeld, welches gewonnen wird, wenn Anna der Koalition mit Ben beitritt. 


Der Shapley-Wert eines Spielers $i$ errechnet sich als das gewichtete Mittel der marginalen Beiträge zu allen möglichen Koalitionen:

\begin{equation}
\varphi_i (\mathcal{N}, v) = \sum_{\mathcal{S} \subseteq \mathcal{N} \setminus \{i\}} \underbrace{\frac{|\mathcal{S}|! \cdot (n - 1 - |\mathcal{S}|)!}{n!}}_{\text{Gewicht}} \underbrace{v(\mathcal{S} \cup \{i\}) - v(\mathcal{S})}_{\substack{\text{marginaler Beitrag von} \\ \text{Spieler $i$ zur Koalition $\mathcal{S}$}}}.
\end{equation}

Die Summationsnotation \(\sum_{\mathcal{S} \subseteq \mathcal{N} \setminus \{i\}}\) erfasst die marginalen Beiträge, 
die der Spieler \( i \) zu allen Koalitionen leistet, die diesen noch nicht einschließen. Die Verwendung von 
\(\mathcal{N} \setminus \{i\}\) stellt sicher, dass Spieler \( i \) nur für jene Koalitionen berücksichtigt wird, 
zu denen er noch beitragen kann. Im Falle von Anna etwa, beziehen sich die Berechnungen auf die Koalitionen bestehend 
aus der der leeren Koalition \(\emptyset\), aus \(\{\text{Ben}\}\), \(\{\text{Carla}\}\), oder beiden 
zusammen \(\{\text{Ben, Carla}\}\) (vgl. Berechnung \ref{eq:marginal_anna}).

Die Formel \(\frac{|\mathcal{S}|! \cdot (n - 1 - |\mathcal{S}|)!}{n!}\) in der Shapley-Wert-Berechnung 
reflektiert den Gewichtungsfaktor für die marginalen Beiträge eines Spielers. Hierbei gibt \(|\mathcal{S}|!\) die Permutationen 
der Spieler innerhalb der Koalition \(\mathcal{S}\) an, während \((n - 1 - |\mathcal{S}|)!\) die Anordnungen der 
außenstehenden Spieler repräsentiert, nachdem der betrachtete Spieler beigetreten ist. 
Der Bruchteil \(\frac{1}{n!}\) normalisiert diesen Wert über alle möglichen Koalitionszusammensetzungen, 
wodurch die Wahrscheinlichkeit der Bildung einer spezifischen Koalition ausgedrückt wird.

Betrachten wir Anna als den Spieler $i$ und die Koalition \(\mathcal{S} = \{\text{Ben, Carla}\}\). 
Die Formel \(\frac{|\mathcal{S}|! \cdot (n - 1 - |\mathcal{S}|)!}{n!}\) berechnet den Gewichtungsfaktor 
für Annas marginalen Beitrag zur Koalition \(\mathcal{S}\). In diesem Fall ist \(|\mathcal{S}| = 2\) und \(n = 3\). 
Somit ergibt sich \(|\mathcal{S}|! = 2!\) und \(n - 1 - |\mathcal{S}| = 0!\), da nach dem Beitritt von 
Anna keine weiteren Spieler übrig sind. Der Normalisierungsfaktor ist \(n! = 3! = 6\). Daraus folgt:

\begin{equation}
\frac{2! \cdot 0!}{3!} = \frac{2 \cdot 1}{6} = \frac{1}{3}.
\end{equation}

Dies bedeutet, dass unter allen möglichen Permutationen der Spielerreihenfolge, Annas Beitritt zu der Koalition \{Ben, Carla\} 
genau ein Drittel der Zeit am Ende geschieht. Somit wird ihr marginaler Beitrag mit diesem Faktor gewichtet, 
um den Shapley-Wert zu berechnen (vgl. Berechnung \ref{eq:marginal_anna}) \cite[S. 21f]{Molnar_2023}.


\section{Axiome}
\label{sec:axiome-shapley}

Nachdem die Berechnung des Shapley-Werts für das Beispiel konkretisiert wurde, ist es nun von Bedeutung, die zugrundeliegenden Axiome zu betrachten, 
welche die theoretische Rechtfertigung für die Methode liefern. Der Shapley-Wert wird nicht nur durch seine Berechnungsmethode, 
sondern auch durch eine Reihe von Axiomen charakterisiert, die seine Fairness und Kohärenz im Kontext kooperativer Spiele sicherstellen. 
Lloyd Shapley leitete den Shapley-Wert ursprünglich aus diesen Axiomen ab und bewies, 
dass dieser der einzige ist, der den Axiomen genügt\footnote{Eine detaillierte Darstellung dieser Axiome und des Beweises 
ihrer Einzigartigkeit findet sich in Shapleys Originalarbeit, deren umfassende Behandlung jedoch den Rahmen dieser Arbeit überschreiten würde \cite[S. 307-318]{Shapley+1953+307+318}.}. 
Diese Axiome sind wesentliche Bestandteile, die die Einzigartigkeit und die wünschenswerten Eigenschaften des Shapley-Werts als Lösungskonzept definieren \cite[S. 22]{Molnar_2023}. 

\paragraph{Effizienz}

Der Wert der großen Koalition wird an die Spieler verteilt:

\begin{equation}
\sum_{i \in \mathcal{N}} \varphi_i (\mathcal{N}, v) = v(\mathcal{N}).
\end{equation}

Dies bedeutet, dass die Summe der Shapley-Werte aller Spieler dem Gesamtwert entspricht, 
den die Koalition aller Spieler zusammen erreichen kann. Der Gesamtwert, den die große Koalition $\mathcal{N}$, 
bestehend aus Anna, Ben und Carla, generiert, wird komplett unter den Spielern aufgeteilt \cite[S. 22]{Molnar_2023}. 
Unter Vernachlässigung minimaler Rundungsdifferenzen entspricht die Summe der Shapley-Werte, 
berechnet in den Gleichungen \ref{eq:marginal_anna}, \ref{eq:marginal_ben} und \ref{eq:marginal_carla}, 
dem kollektiven Ertrag der großen Koalition:

\begin{equation}
  458,34 \text{\euro} + 458,34 \text{\euro} + 83,32 \text{\euro} \approx 1000 \text{\euro} 
\end{equation}


\paragraph{Symmetrie}

Zwei Spieler $i$ und $j$, die die gleichen marginalen Beiträgen zu jeder Koalition haben erhalten das Gleiche:

\begin{equation}
v(\mathcal{S} \cup \{i\}) = v(\mathcal{S} \cup \{j\}), \; \forall\, \mathcal{S} \subseteq \mathcal{N} \setminus \{i, j\} \Rightarrow \varphi_i (\mathcal{N}, v) = \varphi_j (\mathcal{N}, v).
\end{equation}

Obwohl Anna und Ben den gleichen Shapley-Wert erhalten, ist dies nicht auf das Symmetrieaxiom zurückzuführen, 
da ihre marginalen Beiträge zu den Koalitionen variieren. Zum Beispiel leistet Anna keinen Beitrag zur Koalition, 
wenn Ben bereits Teil davon ist, während Ben einen positiven Beitrag leistet, 
wenn Anna bereits zur Koalition gehört (vgl. Tabelle \ref{tab:shapley_marginal}). Dies zeigt, dass die Gleichheit ihrer Shapley-Werte ein Ergebnis der 
spezifischen Zahlenkonstellation in diesem Szenario ist und nicht aus der symmetrischen Interaktion 
zwischen den beiden Spielern resultiert.

\paragraph{Null-Spieler-Eigenschaft (Dummy-Spieler-Eigenschaft)}

Ein Spieler $i$ der zu jeder Koalition nichts beiträgt erhält den Wert Null:

\begin{equation}
  v(\mathcal{S} \cup \{i\}) =  v(\mathcal{S}), \; \forall\, \mathcal{S} \subseteq \mathcal{N} \setminus \{i\} \Rightarrow \varphi_i (\mathcal{N}, v) = 0.
\end{equation}

Dies stell sicher, dass ein Spieler, der keinen Beitrag leistet, auch nicht belohnt wird. 

\paragraph{Additivität}

Wenn das Spiel in zwei unabhängige Spiele mit Koalitionsfunktionen $v$ und $w$ zerlegt werden kann, dann ist die Auszahlung jedes Spielers im 
zusammengesetzten Spiel die Summe der Auszahlungen in den aufgeteilten Spielen:

\begin{equation}
\varphi_i (\mathcal{N}, v + w) = \varphi_i (\mathcal{N}, v) + \varphi_i (\mathcal{N}, w).
\end{equation}

Wenn Anna, Ben und Carla neben dem ersten Wettbewerb an einem zweiten, unabhängigen Wettbewerb teilnehmen, 
besagt das Additivitätsaxiom, dass die Shapley-Werte jedes Spielers aus beiden Wettbewerben einfach die Summe ihrer individuellen 
Shapley-Werte aus jedem einzelnen Wettbewerb sind. Dies impliziert, dass die faire Aufteilung der Gewinne aus beiden Wettbewerben 
konsistent bleibt, indem die aus dem ersten Wettbewerb abgeleiteten Prinzipien auf den zweiten Wettbewerb übertragen und 
dann addiert werden \cite[S. 5573, S.22f]{ijcai2022p778, Molnar_2022}.
\chapter{Von Shapley-Werten zu SHAP: Brückenschlag zur Modellinterpretation}

Im Rahmen der kooperativen Spieltheorie ermöglichen die Shapley-Werte eine faire Verteilung des kollektiv 
erwirtschafteten Nutzens auf die beteiligten Akteure. Diese Methodik findet eine analoge Anwendung 
in der Welt des maschinellen Lernens, um die Beiträge einzelner Merkmale zur Vorhersageleistung 
eines Modells zu bewerten. Hier wird die Terminologie der Shapley-Werte in den Kontext von Machine Learning 
Modellen übertragen, wobei jedes Merkmal als \glqq{}Spieler\grqq{} betrachtet wird, dessen Beitrag zur 
\glqq{}Auszahlung\grqq{} – der Vorhersage des Modells – evaluiert werden soll. 

\begin{table}[h]
    \caption{Terminologie der originären Shapley-Werte im Kontext des maschinellen Lernens.}
    \footnotesize
    \begin{tabularx}{\textwidth}{XXX}
    \toprule
    Terminologie Konzept & Terminologie Machine Learning & Ausdruck \\
    \midrule
    Spieler & Merkmal Index & $j$ \\
    Anzahl aller Spieler & Anzahl aller Merkmale & $p$ \\
    Große Koalition & Menge aller Merkmale & $\mathcal{N} = \{1, \ldots, p\}$\\
    Koalition & Menge von Merkmalen & $\mathcal{S} \subseteq \mathcal{N}$ \\
    Größe der Koalition & Anzahl der Merkmale in der Koalition $\mathcal{S}$ & $|\mathcal{S}|$\\
    Spieler, die nicht in der Koalition sind & Merkmale, die nicht in der Koalition enthalten sind & $C: C = \mathcal{N} \setminus \mathcal{S}$ \\
    Koalitionsfunktion & Vorhersage für Merkmalswerte in der Koalition $\mathcal{S}$ abzüglich der Vorhersage im Mittel & $v_{f, x^{(i)}}(\mathcal{S}$)\\
    Auszahlung & Vorhersage für eine Beobachtung $x^{(i)}$ abzüglich der Vorhersage im Mittel & $f(x^{(i)}) -  \mathbb{E}(f(X))$\\
    Shapley-Wert & Beitrag des Merkmals $j$ zur Auszahlung des Modells für eine Beobachtung $x^{(i)}$& $\varphi_j^{(i)}(\mathcal{N}, f)$\\
    \bottomrule
    \end{tabularx}
    \label{tab:shapley_terms}
    \normalsize\\
    Quelle: \cite[S. 26]{Molnar_2023}.
\end{table}

Die Koalitionsfunktion $v_{f, x^{(i)}}(\mathcal{S})$ für ein gegebenes Model $f$ und eine Beobachtung $x^{(i)}$ ist definiert als:

\begin{align}
    v_{f, x^{(i)}}(\mathcal{S}) = \int_{\mathbb{R}} f(x^{(i)}_{\mathcal{S}} \cup X_{C}) d\mathbb{P}_{X_{C}} - \mathbb{E}(f(X))
\end{align}

Diese Funktion berechnet den erwarteten Wert der Vorhersage des Modells $f$, wenn nur eine Teilmenge $\mathcal{S}$ der 
Merkmale genutzt wird, um die Vorhersage für die spezifische Beobachtung $x^{(i)} \in \mathbb{R}^{p}$ zu treffen. 
Das Integral $\int_{\mathbb{R}}$ repräsentiert die Berechnung dieses erwarteten Wertes über alle möglichen Werte der Merkmale, 
die nicht in $\mathcal{S}$ enthalten sind ($X_C$), gewichtet durch deren Wahrscheinlichkeitsverteilung $\mathbb{P}_{X_{C}}$. 
Die Differenz zum Erwartungswert der Vorhersagen über alle Merkmale $\mathbb{E}(f(X))$ zeigt, 
wie viel die spezifische Menge an Merkmalen $\mathcal{S}$ zur Vorhersage beiträgt \cite[S. 221, S. 27]{Molnar_2022, Molnar_2023}.

Der marginale Beitrag eines Merkmals $j$ zu einer Koalition $\mathcal{S}$ ist dann:

\begin{align}
    v_{f, x^{(i)}}(\mathcal{S} \cup \{j\}) - v_{f, x^{(i)}}(\mathcal{S}) &= \int_{\mathbb{R}} f(x^{(i)}_{\mathcal{S} \cup \{j\}} \cup X_{C \setminus \{j\}}) d\mathbb{P}_{X_{C \setminus \{j\}}} - \mathbb{E}(f(X)) \\ \notag
    &\quad - \left( \int_{\mathbb{R}} f(x^{(i)}_{\mathcal{S}} \cup X_{C}) d\mathbb{P}_{X_{C}} - \mathbb{E}(f(X)) \right) \\ \notag
    &= \int_{\mathbb{R}} f(x^{(i)}_{\mathcal{S} \cup \{j\}} \cup X_{C \setminus \{j\}}) d\mathbb{P}_{X_{C \setminus \{j\}}} \\ \notag
    &\quad - \int_{\mathbb{R}} f(x^{(i)}_{\mathcal{S}} \cup X_{C}) d\mathbb{P}_{X_{C}}
\end{align}

Diese Gleichung beschreibt, wie sich der erwartete Wert der Vorhersage ändert, wenn das Merkmal $j$ zu der Menge der Merkmale $\mathcal{S}$ hinzugefügt wird \cite[S. 29]{Molnar_2023}.

Der Beitrag $\varphi_j^{(i)}(\mathcal{N}, f)$ eines Merkmals $j$ für eine Beobachtung $x^{(i)} \in \mathbb{R}^{p}$ für die Vorhersage $f(x^{(i)})$ ist gegeben als:

\begin{align}
    \label{eq:shap-eq}
    \varphi^{(i)}_{j} (\mathcal{N}, f) &= \sum_{\mathcal{S} \subseteq \mathcal{N} \setminus \{j\}} \frac{|\mathcal{S}|! \cdot (p - 1 - |\mathcal{S}|)!}{p!} \\ \notag
    &\quad \cdot \left( \int_{\mathbb{R}} f(x^{(i)}_{\mathcal{S} \cup \{j\}} \cup X_{C \setminus \{j\}}) d\mathbb{P}_{X_{C \setminus \{j\}}} -
    \int_{\mathbb{R}} f(x^{(i)}_{\mathcal{S}} \cup X_{C}) d\mathbb{P}_{X_{C}} \right) 
\end{align}

Diese Formel ist die zentrale Berechnung der SHAP-Werte im maschinellen Lernen. 
Sie summiert den gewichteten, marginalen Beitrag des Merkmals $j$ über alle möglichen Kombinationen der anderen Merkmale. 
Die Gewichtung berücksichtigt die Anzahl der Merkmale in der Koalition $\mathcal{S}$ und die Anzahl der verbleibenden Merkmale, 
die noch hinzugefügt werden können. Dies ergibt den durchschnittlichen Beitrag des Merkmals $j$ zur Vorhersage für die 
Beobachtung $x^{(i)}$ \cite[S. 29, 30]{Molnar_2023}.

Die Integration in der SHAP-Formel ist ein zentraler Schritt, um den erwarteten Beitrag jedes Merkmals unter 
Berücksichtigung der gesamten Verteilung der Daten zu ermitteln. 
In diesem Ansatz werden die Merkmale als Zufallsvariablen behandelt, und die Integration erfolgt über 
die Wahrscheinlichkeitsverteilungen dieser Zufallsvariablen. Durch das Berechnen der erwarteten Vorhersagewerte mit 
und ohne des jeweiligen Merkmals, unter Einbeziehung der Verteilung aller anderen Merkmale, 
ermöglicht SHAP eine präzise und umfassende Einschätzung des Einflusses jedes einzelnen Merkmals. 
Dieser Prozess der Marginalisierung, bei dem man über die Wahrscheinlichkeitsverteilungen der Merkmale integriert, 
erlaubt es, den Beitrag eines jeden Merkmals zu isolieren und unabhängig von der spezifischen Zusammensetzung 
der anderen Merkmale zu bewerten. Dies führt zu einer fairen und ganzheitlichen Bewertung der Beiträge aller Merkmale 
zur Vorhersage des Modells \cite[S. 28]{Molnar_2023}.

\section{Berechnung der SHAP-Werte unter Berücksichtigung der zugrundeliegenden Verteilung}
\label{sec:example}

Ein einfaches Beispiel soll helfen, die Anwendung von SHAP-Werten im 
Kontext des maschinellen Lernens zu illustrieren\footnote{In Anlehnung an das Beispiel aus Kapitel 8.5.1 \glqq{}General Idea\grqq{} \cite[S.215f]{Molnar_2022}.}. Betrachtet wird ein fiktiver 
Immobilien-Datensatz mit drei Merkmalen: Größe des Hauses in Quadratmetern ($x_1$), Anzahl der Zimmer ($x_2$) 
und Entfernung zum Stadtzentrum in Kilometern ($x_3$). Es gibt zwei Beobachtungen in diesem Datensatz:

\begin{table}[h]
    \caption{Merkmale von Beobachtungen in einem Immobilien-Datensatz.}
    \footnotesize
    \begin{tabularx}{\textwidth}{Xrrr}
    \toprule
     & $x_1$: Größe (in $m^2$) &  $x_2$: Anzahl Zimmer &  $x_3$: Entfernung zum Zentrum (in km) \\
    \midrule
    $x^{(1)}$ & 100 & 3 & 5 \\
    $x^{(2)}$ & 150 & 4 & 10 \\
    \bottomrule
    \end{tabularx}
    \label{tab:example}
    \normalsize\\
    Quelle: Eigene Darstellung.
\end{table}

Angenommen das Modell $f(x^{(i)})$ prognostiziert den Preis eines Hauses in Euro als eine lineare Kombination der Merkmale:

\begin{equation}
    f(x^{(i)}) = 5x_1^{(i)} + 20x_2^{(i)} - 2x_3^{(i)}.
\end{equation}

Die Vorhersagen für die beiden Beobachtungen lauten dann:

\begin{align}
    \label{eq:fx1}
    f(x^{(1)}) &= 5x_1^{(1)} + 20x_2^{(1)} - 2x_3^{(1)} \\ \notag
        &= 5 \cdot 100 + 20 \cdot 3 - 2 \cdot 5 \\ \notag
        &= 550\,\text{\euro} 
\end{align}

und 

\begin{align}
    f(x^{(2)}) &= 5x_1^{(2)} + 20x_2^{(2)} - 2x_3^{(2)} \\ \notag
        &= 5 \cdot 150 + 20 \cdot 4 - 2 \cdot 10 \\ \notag
        &= 810\,\text{\euro}. 
\end{align}

Die erwartete Auszahlung des Modells $\mathbb{E}(f(X))$ wird berechnet als:
\begin{align}
    \label{eq:efx}
    \mathbb{E}(f(X)) &= 5 \cdot \mathbb{E}(X_1) + 20 \cdot \mathbb{E}(X_2) - 2 \cdot \mathbb{E}(X_3) \\ \notag
                     &= 5 \cdot 125 + 20 \cdot 3,5 - 2 \cdot 7,5 \\ \notag
                     &= 680 \,\text{\euro},
\end{align}

mit 

\begin{align}
    \label{eq:e}
    \mathbb{E}(X_j) = \frac{1}{n} \sum_{i=1}^{n} x_j^{(i)}.
\end{align}     

Sei $\mathcal{N} = \{1, 2, 3\}$ die Menge aller Merkmale und die Beobachtung $x^{(1)} = [100, 3, 5]$. 
Der SHAP-Wert für jedes Merkmal $j \in \mathcal{N}$ wird unter Berücksichtigung der Verteilung der 
Daten und der Formel \ref{eq:shap-eq} berechnet:

\begin{align}
    \label{eq:shap-formular}
    \varphi^{(1)}_{j} (\mathcal{N}, f) &= \sum_{\mathcal{S} \subseteq \mathcal{N} \setminus \{j\}} \frac{|\mathcal{S}|! \cdot (p - 1 - |\mathcal{S}|)!}{p!} \\ \notag
    &\quad \cdot \left( \int_{\mathbb{R}} f(x^{(1)}_{\mathcal{S} \cup \{j\}} \cup X_{C \setminus \{j\}}) d\mathbb{P}_{X_{C \setminus \{j\}}} -
    \int_{\mathbb{R}} f(x^{(1)}_{\mathcal{S}} \cup X_{C}) d\mathbb{P}_{X_{C}} \right) 
\end{align}

wobei $p = |\mathcal{N}| = 3$ die Anzahl der Merkmale ist und $X_C$ die Menge der Merkmale 
außerhalb der Koalition $\mathcal{S}$ repräsentiert. Die Integrale repräsentieren die erwartete 
Vorhersage des Modells über die Verteilung der nicht in der Koalition enthaltenen Merkmale.

In linearen Modellen, unter der Prämisse, dass die Merkmale unabhängig voneinander und gleichverteilt sind, 
ist es möglich, die Berechnung der SHAP-Werte zu vereinfachen. Anstelle der komplexen Integration 
über die Verteilungen aller Merkmale, kann der Fokus auf die Unterschiede in den Modellvorhersagen gelegt werden, 
die sich aus dem Hinzufügen oder Entfernen einzelner Merkmale ergeben. 
Hierbei wird anstelle der spezifischen Werte der nicht in der betrachteten Koalition enthaltenen Merkmale 
Erwartungswerte herangezogen. Diese Vereinfachung ermöglicht es, den Einfluss jedes Merkmals auf 
die Modellvorhersage auf eine direktere und rechnerisch weniger aufwendige Weise zu erfassen.
Diese Vereinfachung ist für lineare Modelle angemessen, da die Auswirkungen jedes Merkmals 
auf die Vorhersage des Modells additiv und unabhängig sind. Bei komplexeren, 
nichtlinearen Modellen ist eine detailliertere Berechnung erforderlich, 
die oft auf numerischen Methoden oder Annäherungen basiert, mehr dazu in Kapitel \ref{sec:estimators}.

Der Beitrag durch das Hinzufügen des Merkmals $x_1$ zur bestehenden Koalition $\mathcal{S} = \{x_2\}$ wird
nach Formel \ref{eq:shap-formular} berechnet als:

\begin{align}
    \varphi^{(1)}_{1}(\{x_2\}, f) &= \frac{1! \cdot (3 - 1 - 1)!}{3!} \\ \notag
        &\quad \cdot \int f(x_1, x_2 , X_3) d\mathbb{P}(X_3) - \int f(X_1, x_2, X_3) d\mathbb{P}(X_1, X_3)
\end{align}

Da \( X_1 \) und \( X_3 \) unabhängig und gleichmäßig verteilt sind, können \( X_2 \) und \( X_3 \) durch ihre Erwartungswerte (Gleichung \ref{eq:e}) ersetzt werden:

\begin{align}
    \varphi^{(1)}_{1}(\{x_2\}, f) &= \frac{1}{6} \Big(f(x_1, x_2, \mathbb{E}(X_3)) - f(\mathbb{E}(X_1), x_2, \mathbb{E}(X_3))\Big) \\ \notag
        &= \frac{1}{6} \Big( f(100, 3, 7.5) - f(125, 3, 7.5) \Big) \\ \notag
        &= \frac{1}{6} \Big( (5 \cdot 100 + 20 \cdot 3 - 2 \cdot 7,5) - (5 \cdot 125 + 20 \cdot 3 - 2 \cdot 7,5) \Big) \\ \notag
        &= \frac{1}{6} (545 - 670) \\ \notag
        &= \frac{1}{6} (-125)
\end{align}
 
Die in Tabelle \ref{tab:shapley_marginal_features} dargestellten Kombinationen illustrieren die marginalen Beiträge und SHAP-Werte 
für jedes Merkmal in jeder möglichen Koalition von Merkmalen, bezogen auf die Beobachtung $x^{(1)}$. 

\begin{table}[h]
    \caption{Marginalbeiträge der einzelnen Merkmale zu den möglichen Koalitionen für die Beobachtung $x^{(1)}$.}
    \footnotesize
    \begin{tabularx}{\textwidth}{XXrrrrr}
    \toprule
    $x_{j}$ & $\mathcal{S}$ & $v_{f, x^{(1)}}(\mathcal{S})$ & $v_{f, x^{(1)}}(\mathcal{S} \cup \{j\})$ & $v_{f, x^{(1)}}(\mathcal{S} \cup \{j\}) - v_{f, x^{(1)}}(\mathcal{S})$ & Gewicht & $\varphi_{j}^{(1)}(\mathcal{S}, f)$\\
    \midrule
    $x_1$ & $\emptyset$ & 680 & 555 & -125 & $\frac{1}{3}$ & -41,67 \\
    $x_1$ & $\{x_2\}$ & 670 & 545 & -125 & $\frac{1}{6}$ & -20.83 \\
    $x_1$ & $\{x_3\}$ & 685 & 560 & -125 & $\frac{1}{6}$ & -20.83 \\
    $x_1$ & $\{x_2, x_3\}$ & 675 & 550 & -125 & $\frac{1}{3}$ & -41,67 \\
    $x_2$ & $\emptyset$ & 680 & 670 & -10 & $\frac{1}{3}$ & -3,33 \\
    $x_2$ & $\{x_1\}$ & 555 & 545 & -10 & $\frac{1}{6}$ & -1,67 \\
    $x_2$ & $\{x_3\}$ & 685 & 675 & -10 & $\frac{1}{6}$ & -1,67 \\
    $x_2$ & $\{x_1, x_3\}$ & 560 & 550 & -10 & $\frac{1}{3}$ & -3,33 \\
    $x_3$ & $\emptyset$ & 680 & 685 & 5 & $\frac{1}{3}$ & 1,67 \\
    $x_3$ & $\{x_1\}$ & 555 & 560 & 5 & $\frac{1}{6}$ & 0,83 \\
    $x_3$ & $\{x_2\}$ & 670 & 675 & 5 & $\frac{1}{6}$ & 0,83 \\
    $x_3$ & $\{x_1, x_2\}$ & 545 & 550 & 5 & $\frac{1}{3}$ & 1,67 \\
    \bottomrule
    \end{tabularx}
    \label{tab:shapley_marginal_features_x1}
    \normalsize\\
    Quelle: Eigene Darstellung.
\end{table}

Diese Analyse ist ebenso auf die Beobachtung $x^{(2)}$ anwendbar und erfordert eine analoge Vorgehensweise:

\begin{table}[h]
    \caption{Marginalbeiträge der einzelnen Merkmale zu den möglichen Koalitionen für die Beobachtung $x^{(2)}$.}
    \footnotesize
    \begin{tabularx}{\textwidth}{XXrrrrr}
    \toprule
    $x_{j}$ & $\mathcal{S}$ & $v_{f, x^{(2)}}(\mathcal{S})$ & $v_{f, x^{(2)}}(\mathcal{S} \cup \{j\})$ & $v_{f, x^{(2)}}(\mathcal{S} \cup \{j\}) - v_{f, x^{(2)}}(\mathcal{S})$ & Gewicht & $\varphi_{j}^{(2)}(\mathcal{S}, f)$\\
    \midrule
    $x_1$ & $\emptyset$ & 680 & 805 & 125 & $\frac{1}{3}$ & 41,67 \\
    $x_1$ & $\{x_2\}$ & 690 & 815 & 125 & $\frac{1}{6}$ & 20.83 \\
    $x_1$ & $\{x_3\}$ & 675 & 800 & 125 & $\frac{1}{6}$ & 20.83 \\
    $x_1$ & $\{x_2, x_3\}$ & 685 & 810 & 125 & $\frac{1}{3}$ & 41,67 \\
    $x_2$ & $\emptyset$ & 680 & 690 & 10 & $\frac{1}{3}$ & 3,33 \\
    $x_2$ & $\{x_1\}$ & 805 & 815 & 10 & $\frac{1}{6}$ & 1,67 \\
    $x_2$ & $\{x_3\}$ & 675 & 685 & 10 & $\frac{1}{6}$ & 1,67 \\
    $x_2$ & $\{x_1, x_3\}$ & 800 & 810 & 10 & $\frac{1}{3}$ & 3,33 \\
    $x_3$ & $\emptyset$ & 680 & 675 & -5 & $\frac{1}{3}$ & -1,67 \\
    $x_3$ & $\{x_1\}$ & 805 & 800 & -5 & $\frac{1}{6}$ & -0,83 \\
    $x_3$ & $\{x_2\}$ & 690 & 685 & -5 & $\frac{1}{6}$ & -0,83 \\
    $x_3$ & $\{x_1, x_2\}$ & 815 & 810 & -5 & $\frac{1}{3}$ & -1,67 \\
    \bottomrule
    \end{tabularx}
    \label{tab:shapley_marginal_features_x2}
    \normalsize\\
    Quelle: Eigene Darstellung.
\end{table}

Die in den Tabellen \ref{tab:shapley_marginal_features_x1} und \ref{tab:shapley_marginal_features_x2} 
dargestellten SHAP-Werte für die Beobachtungen \( x^{(1)} \) und \( x^{(2)} \) zeigen eine interessante Symmetrie. 
Für jedes Merkmal entspricht der SHAP-Wert für \( x^{(2)} \) genau dem negativen Wert für \( x^{(1)} \). 
Die beobachtete Inversion der SHAP-Werte zwischen den Beobachtungen \( x^{(1)} \) und \( x^{(2)} \) 
lässt sich durch die zentrale Rolle des Erwartungswerts des Modells erklären. 
Das Modell prognostiziert im Mittel einen Immobilienpreis von 680 \euro. In einem Szenario, in dem nur zwei 
Beobachtungen vorhanden sind, spiegeln die Beobachtungen \( x^{(1)} \) und \( x^{(2)} \) entgegengesetzte 
Abweichungen vom Erwartungswert wider. Die SHAP-Werte, die den Beitrag jedes Features zur Abweichung 
der Vorhersage vom Mittelwert messen, zeigen daher für \( x^{(1)} \) und \( x^{(2)} \) genau entgegengesetzte Werte. 
Dieses Phänomen resultiert aus der Tatsache, dass die Merkmalsbeiträge in Bezug auf den Erwartungswert 
berechnet werden und unsere beispielhaften Beobachtungen symmetrisch um diesen Mittelwert verteilt sind. 
Folglich heben sich ihre Beiträge im Kontext unseres linearen Modells exakt auf, was die Konsistenz und 
Zuverlässigkeit der SHAP-Wert-Berechnung unterstreicht.


\begin{figure}[H]
    \caption{Beitrag der Merkmale $x_{j \in \{1, 2, 3\}}$ zur Modellvorhersage $f(x^{(1)})$.}
    \includegraphics[width=1\textwidth]{../scripts/images/model-output-x1.png}
    Quelle: Eigene Darstellung, \ref{charts}.
    \label{pic:model-fx1}
\end{figure}

Das Modell $f(x^{(i)})$
prognostiziert im Mittel einen Immobilienpreis von 680 \euro{}. Im Vergleich zur
Verteilung des jeweiligen Merkmals, reduziert die Größe der 
Wohnung ($x_1^{(1)}$) und die Anzahl der Zimmer ($x_2^{(1)}$) die Prognose 
des Preises für die Immobilie $x^{(1)}$ um insgesamt 135 \euro{}, während die Entfernung zum Stadtzentrum
($x_3^{(1)}$) den Preis der Wohnung um 5 \euro{} erhöht, wie in Abbildung 
\ref{pic:model-fx1} veranschaulicht. Abbildung \ref{pic:model-fx2} visualisiert die SHAP-Werte für die Immobilie $x^{(2)}$:

\begin{figure}[H]
    \caption{Beitrag der Merkmale $x_{j \in \{1, 2, 3\}}$ zur Modellvorhersage $f(x^{(2)})$.}
    \includegraphics[width=1\textwidth]{../scripts/images/model-output-x2.png}
    Quelle: Eigene Darstellung, \ref{charts}.
    \label{pic:model-fx2}
\end{figure}


\section{Axiome}

Die in Tabellen \ref{tab:shapley_marginal_features_x1} und \ref{tab:shapley_marginal_features_x2} präsentierten Ergebnisse bieten eine Grundlage, 
um die Konformität der SHAP-Werte mit den etablierten Axiomen der Shapley-Werte, wie sie im 
Kapitel \ref{sec:axiome-shapley} diskutiert wurden, zu beurteilen. Die Axiome der SHAP-Werte stellen 
eine adaptierte und kontextualisierte Anwendung dieser Prinzipien auf die Interpretation von 
Modellvorhersagen dar \cite{NIPS2017_8a20a862}.


\paragraph{Effizienz}

Das Effizienzaxiom besagt, dass die Summe der SHAP-Werte aller Features für eine gegebene Beobachtung $x^{(i)}$ 
gleich der Differenz zwischen der Modellvorhersage für diese Beobachtung $f(x^{(i)})$ 
und der durchschnittlichen Modellvorhersage $\mathbb{E}(f(X))$ sein muss:

\begin{align}
    \sum_{j=1}^{p}\varphi_j^{(i)}(\mathcal{N}, f) = f(x^{(i)}) - \mathbb{E}(f(X)),
\end{align}

\cite[S. 221]{Molnar_2022}. Für die Beobachtung $x^{(1)}$ aus Kapitel \ref{sec:example} und den Berechnungen 
für $f(x^{(1)})$ (Gleichung \ref{eq:fx1}), sowie $\mathbb{E}(f(X))$ (Gleichung \ref{eq:efx}) ergibt sich:

\begin{align}
    \sum_{j=1}^{3}\varphi_j^{(1)}(\mathcal{N}, f) &=  -130 \\ \notag
    f(x^{(1)}) - \mathbb{E}(f(X)) &= 550 - 680 = -130,   
\end{align}

womit das Effizienzaxiom erfüllt ist. Die Differenz der Vohersage 
einer konkreten Beobachtung zur durchschnittlichen 
Modellvorhersage wird auf alle Merkmale verteilt.

\paragraph{Symmetrie}

Das Symmetrieaxiom fordert, dass zwei Merkmale $i$ und $j$, die 
in jeder Koalition denselben Beitrag leisten, auch denselben SHAP-Wert 
erhalten müssen. In dem hier betrachteten Fall der Immobilienpreisprognose 
würde dies bedeuten, dass wenn zwei Merkmale, beispielsweise die Größe einer 
Wohnung und die Anzahl der Zimmer, immer den gleichen Einfluss auf den Preis hätten, 
unabhängig von der Kombination anderer Merkmale, ihre SHAP-Werte identisch 
sein müssen:

\begin{equation}
    v(\mathcal{S} \cup \{i\}) = v(\mathcal{S} \cup \{j\}), \; \forall\, \mathcal{S} \subseteq \mathcal{N} \setminus \{i, j\} \Rightarrow \varphi_i (\mathcal{N}, v) = \varphi_j (\mathcal{N}, v),
\end{equation}

\cite[S. 221]{Molnar_2022}. Dies wird durch die Tabellen \ref{tab:shapley_marginal_features_x1} und \ref{tab:shapley_marginal_features_x2} 
nicht illustriert, da jedes Merkmal einen unterschiedlichen Beitrag liefert, 
was die Anwendung dieses Axioms in diesem speziellen Fall ausschließt.

\paragraph{Null-Spieler-Eigenschaft (Dummy-Spieler-Eigenschaft)}

Ein Merkmal $i$, das keinen Einfluss auf die Modellvorhersage hat, erhält gemäß der 
Null-Spieler-Eigenschaft einen SHAP-Wert von Null. 
Im Kontext des Beispiels würde ein Merkmal, das keine Veränderung in der Vorhersage 
bewirkt, unabhängig von den anderen Merkmalen, einen SHAP-Wert von Null erhalten:

\begin{equation}
    v(\mathcal{S} \cup \{i\}) =  v(\mathcal{S}), \; \forall\, \mathcal{S} \subseteq \mathcal{N} \setminus \{i\} \Rightarrow \varphi_i (\mathcal{N}, v) = 0,
\end{equation}

\cite[S. 222]{Molnar_2022}. In der fiktiven Datenlage der Tabellen \ref{tab:shapley_marginal_features_x1} und \ref{tab:shapley_marginal_features_x2} hat jedes Merkmal 
einen gewissen Einfluss, sodass die Null-Spieler-Eigenschaft hier nicht beobachtet werden kann.

\paragraph{Additivität}

Das Additivitätsaxiom ist ein zentrales Konzept, das die Konsistenz von Shapley-Werten 
über die Zusammensetzung von Spielen hinweg beschreibt. Es garantiert, 
dass für zwei separate Spiele oder Modelle \(v\) und \(w\), die Summe der SHAP-Werte eines 
Merkmals über beide Spiele seinem SHAP-Wert im kombinierten Spiel entspricht:

\begin{equation}
    \varphi_i(\mathcal{N}, v + w) = \varphi_i (\mathcal{N}, v) + \varphi_i (\mathcal{N}, w).
\end{equation}

Im Rahmen von Machine Learning-Modellen, insbesondere bei Ensemble Modellen wie dem Random Forest, 
ist die Additivität besonders relevant. Ein Random Forest-Modell ist im Wesentlichen eine Sammlung 
von Entscheidungsbäumen, die jeweils als unabhängige Modelle angesehen werden können. 
Diese Bäume arbeiten zusammen, um eine gemeinsame Vorhersage zu treffen, indem sie ihre 
individuellen Vorhersagen durchschnittlich zusammenführen. Gemäß dem Additivitätsaxiom können die SHAP-Werte, 
die aus den einzelnen Entscheidungsbäumen berechnet werden, als additive Beiträge betrachten, 
die zusammen den Gesamteinfluss eines Features auf die Vorhersage des gesamten Random Forest-Modells ausmachen \cite[S. 32]{Molnar_2023}.


\chapter{Schätzung von SHAP-Werten}
\label{sec:estimators}

Die Berechnung von SHAP-Werten stellt in der Praxis eine erhebliche Herausforderung dar, 
insbesondere bei komplexen Modellen Random Forests oder tiefen neuronalen Netzwerken. 
Während das Beispiel mit Immobilienpreisen noch eine direkte Ermittlung aller möglichen Kombinationen 
von Merkmalen und deren Beiträgen ermöglicht, ist dies bei umfangreichen Datensätzen und Modellen 
mit einer hohen Anzahl von Merkmalen kaum praktikabel. Die rechnerische Komplexität steigt exponentiell 
mit der Anzahl der Merkmale, wodurch eine vollständige Berechnung aller möglichen Merkmalskombinationen schnell unhandlich wird.

In dem Immobilienbeispiel ist es noch möglich, die SHAP-Werte direkt aus den Modellgleichungen abzuleiten, 
da die Beziehungen zwischen den Merkmalen und dem Zielwert klar definiert und leicht zugänglich sind. 
Dies spiegelt sich in der integralen Form der SHAP-Wert-Formel wider, die eine Kenntnis der Verteilung der 
Eingabedaten voraussetzt. Um das Integral zu berechnen, ist es notwendig, die Wahrscheinlichkeitsverteilung 
über die Merkmale zu kennen, was in realen Szenarien oft nicht der trivial ist.

In diesem Kapitel konfrontieren wir uns mit der Fragestellung, wie SHAP-Werte in praktischen, 
weniger transparenten Situationen geschätzt werden können. Wir diskutieren Ansätze, 
die eine Näherung der wahren SHAP-Werte erlauben, ohne auf die vollständige Berechnung 
aller Feature-Kombinationen zurückgreifen zu müssen. Besondere Aufmerksamkeit widmen wir dem 
linearen Estimator, der auf der Annahme beruht, dass das Modellverhalten in der Nähe der untersuchten Beobachtung 
durch ein lineares Modell angenähert werden kann. Diese Annäherung ermöglicht es, die SHAP-Werte 
effizient zu schätzen, ohne dabei die rechnerische Last einer vollständigen kombinatorischen Analyse 
tragen zu müssen. Die daraus resultierenden Schätzungen bieten einen praktikablen Kompromiss zwischen Genauigkeit 
und Berechnungsaufwand und erlauben es, die Beiträge der einzelnen Features zur Modellvorhersage in einer für 
den Anwender interpretierbaren Weise zu quantifizieren.

\chapter{Praktische Anwendung von SHAP auf lineare Modelle}

In diesem Kapitel wird der Einsatz des SHAP-Frameworks zur Interpretation linearer Modelle im 
Kontext des maschinellen Lernens untersucht. Lineare Modelle, gekennzeichnet durch ihre Transparenz 
und einfache Struktur, bilden oft die Basis für das Verständnis komplexerer Algorithmen. 
Dennoch bleibt die Herausforderung bestehen, die Beiträge individueller Merkmale zur Modellvorhersage zu 
quantifizieren und zu interpretieren.

Die Anwendung von SHAP-Werten ermöglicht es, diesen Herausforderungen zu begegnen und Einblicke in 
die Modellvorhersagen zu gewähren, die über traditionelle Methoden hinausgehen. 
Dieses Kapitel führt in die Grundlagen des \textsf{shap}-Pakets ein, demonstriert dessen Anwendung auf einen 
spezifischen Datensatz und diskutiert die Berechnung sowie Interpretation der resultierenden SHAP-Werte.

\section{Lineare Modelle als analytische Grundlage}

In linearen Regressionsmodellen wird die Zielgröße als eine gewichtete Kombination der Eingangsmerkmale bestimmt. 
Die einfache lineare Struktur dieser Modelle erleichtert das Verständnis der Beziehungen zwischen den Eingangsdaten 
und den Vorhersagen. 

Lineare Modelle sind ein grundlegendes Werkzeug in der statistischen Modellierung und dienen dazu, das Verhältnis zwischen 
einer abhängigen Variablen, die üblicherweise mit $y^{(i)}$ bezeichnet wird, 
und einem oder mehreren Prädiktoren, den unabhängigen Variablen $x_i$, zu erfassen. 
Diese Beziehungen werden mittels linearer Gleichungen dargestellt, die für jede 
einzelne Beobachtung $i$ im Datensatz folgendermaßen formuliert werden können:

\begin{equation}
    y^{(i)} = \beta_0 + \sum_{j=1}^{p} \beta_j x^{(i)}_j + \epsilon^{(i)},
\label{eq:reg-model}
\end{equation}

wobei das Ergebnis, das von einem linearen Modell für eine gegebene Beobachtung vorhergesagt wird, sich als Summe der mit 
Gewichten $\beta_j$ versehenen Merkmale $p$ ergibt.

Hierbei stellt $y^{(i)}$ den beobachteten Wert der abhängigen Variablen für die Beobachtungseinheit
$i$ dar. Der Term $\beta_0$ ist der Achsenabschnitt oder y-Achsenabschnitt des Modells, 
welcher den erwarteten Wert von $y$ darstellt, wenn alle unabhängigen Variablen $x$ null sind. 
Die Summe $\sum_{j=1}^{p} \beta_j x^{(i)}_j$ berechnet sich aus den Produkten der Koeffizienten 
$\beta_j$ und den Werten der unabhängigen Variablen $x^{(i)}_j$ für jede Beobachtungseinheit $i$ 
und jeden Prädiktor $j$, wobei die Koeffizienten $\beta_j$ den geschätzten Einfluss der 
entsprechenden unabhängigen Variablen auf die abhängige Variable beschreiben.

Der Fehlerterm $\epsilon^{(i)}$ steht für die Residuen, also die Differenzen zwischen den beobachteten 
und durch das Modell geschätzten Werten von $y^{(i)}$. Es wird angenommen, dass diese Fehler normalverteilt sind, 
was bedeutet, dass Abweichungen in beiden Richtungen um den Mittelwert (hier Null) 
mit abnehmender Wahrscheinlichkeit für größere Fehler auftreten \cite[S. 37]{Molnar_2022}.

In einem linearen Modell stellt der Achsenabschnitt die Basislinie dar, an der die Auswirkungen aller 
anderen Merkmale gemessen werden. Dieser Wert gibt an, was das Modell für die Zielvariable vorhersagen 
würde, wenn alle anderen Merkmale nicht vorhanden wären – der Ausgangspunkt der Vorhersage 
für einen Datensatz, in dem alle anderen Variablen auf null gesetzt sind. 
Es ist wichtig zu erwähnen, dass der Achsenabschnitt für sich genommen nicht immer eine praktische 
Bedeutung hat, da es selten vorkommt, dass alle Variablen tatsächlich den Wert null annehmen. 
Die wahre Aussagekraft des Achsenabschnitts tritt zutage, wenn die Daten so standardisiert wurden, 
dass ihre Mittelwerte bei null und die Standardabweichung bei eins liegen. Unter diesen Umständen repräsentiert der Achsenabschnitt 
die erwartete Zielvariable für einen hypothetischen Fall, in dem alle Merkmale ihren Durchschnittswert 
aufweisen.

Bei der Betrachtung einzelner Merkmale innerhalb des Modells sagt das Gewicht $\beta_j$ eines Merkmals, 
um wie viel sich die Zielvariable $y^{(i)}$ ändert, wenn das Merkmal $x^{(i)}_j$ um eine Einheit erhöht wird – und zwar unter 
der Annahme, dass alle anderen Merkmale unverändert bleiben. 
Dies ermöglicht es, den isolierten Effekt eines jeden Merkmals auf die Vorhersage zu verstehen \cite[S. 39]{Molnar_2022}.

Die optimalen Gewichte, oder Koeffizienten, eines linearen Regressionsmodells werden üblicherweise durch ein Verfahren bestimmt, 
das als Methode der kleinsten Quadrate (engl. \textit{Ordinary Least Squares}, OLS) bekannt ist. 
Diese Methode sucht die Koeffizienten \( \beta_0, \ldots, \beta_p \), welche die Summe der quadrierten 
Differenzen zwischen den beobachteten Werten der Zielvariablen \( y^{(i)} \) und den von dem Modell 
vorhergesagten Werten minimieren:

\begin{equation}
    \hat{\beta} = \arg \underset{\beta_0, \ldots, \beta_p}{\min} \ \sum_{i=1}^{n} \left( y^{(i)} - \left( \beta_0 + \sum_{j=1}^{p} \beta_j x_j^{(i)}\right)\right)^2.
\end{equation}

Das Ergebnis der Minimierung, \( \hat{\beta} \) stellt den Vektor der geschätzten Koeffizienten dar \cite[S. 37]{Molnar_2022}. 
In der vorliegenden Arbeit wird das Python-Paket \textsf{scikit-learn}\footnote{\url{https://scikit-learn.org}} verwendet, um die lineare Regression durchzuführen und die Koeffizienten 
\( \hat{\beta} \) zu bestimmen. 


\section{Einführung in das \textsf{shap} Python-Paket}
\label{sec:shap-package}

Das Python-Paket \textsf{shap}\footnote{\url{https://shap.readthedocs.io}} ist eine Open-Source-Bibliothek, die es Nutzern ermöglicht, 
die Auswirkungen von Merkmalen auf Vorhersagen von maschinellen Lernmodellen zu interpretieren und zu visualisieren. 
Entwickelt wurde die Bibliothek ursprünglich von Scott Lundberg und weiteren Mitwirkenden im Rahmen der Forschungsarbeit 
an der University of Washington \cite{NIPS2017_8a20a862}. Das Paket basiert auf dem Konzept der Shapley-Werte aus der kooperativen Spieltheorie 
und überträgt diese auf den Kontext des maschinellen Lernens, um als Tool für die Interpretierbarkeit und Erklärbarkeit 
von Modellvorhersagen zu dienen.

\urldef{\ploturl}\url{https://shap.readthedocs.io/en/latest/api.html#plots}

Die Kernfunktion des \textsf{shap}-Pakets ist die Berechnung von SHAP-Werten, welche die Auswirkung der 
Einzelmerkmale auf die Modellvorhersage quantifizieren. Jeder SHAP-Wert ist ein Maß dafür, wie viel jedes Merkmal 
zur Vorhersage beigetragen hat, im Vergleich zu einer durchschnittlichen Vorhersage über den gesamten Datensatz. 
Diese Werte sind besonders wertvoll, weil sie ein Maß für die Bedeutung jedes Merkmals liefern, 
das sowohl lokal (für einzelne Vorhersagen) als auch global (über das gesamte Modell) interpretiert werden kann.

Mit \textsf{shap} können Benutzer die Vorhersagen einer Vielzahl von Modellen interpretieren, 
von linearen Modellen bis hin zu komplexen Konstrukten wie tiefe neuronale Netzwerke. 
Die Bibliothek bietet eine vielseitige Auswahl an Visualisierungsoptionen, darunter Beeswarm-Plots, Dependence-Plots und 
Bar-Plots, die es ermöglichen, die SHAP-Werte intuitiv zu verstehen. Eine Übersicht aller Visualisierungsoptionen ist in der Dokumentation 
des Pakets zu finden\footnote{\ploturl}.
Diese Visualisierungen erleichtern es, Muster und Beiträge einzelner Merkmale zu erkennen, 
was nicht nur wertvolle Einblicke in die Leistung des Modells bietet, sondern auch zu faireren und transparenteren 
Modellentscheidungen führen kann. 

\urldef{\shapurl}\url{https://shap.readthedocs.io/en/latest/api.html#explainers}
\urldef{\exacturl}\url{https://github.com/shap/shap/blob/master/shap/explainers/_exact.py}

Im Kapitel \ref{subsec:linear-shap-estimator} wurde bereits der LinearExplainer aus dem \textsf{shap}-Paket 
vorgestellt, ein Beispiel für die verschiedenen Estimators, die das Paket in Form von Explainern bereitstellt. 
Das Paket bietet eine Vielzahl von Explainern, die auf unterschiedliche Modelltypen zugeschnitten sind. 
Einer der bemerkenswerten Aspekte von \textsf{shap} ist der auto Modus des Estimators, 
der automatisch den am besten geeigneten Explainer für das gegebene Modell auswählt. 
Diese Funktion ist besonders nützlich, da sie die Komplexität der Auswahl des richtigen Explainers 
reduziert und den Anwendungsprozess vereinfacht. Speziell für Modelle mit ca. 15 Merkmalen\footnote{\exacturl} wählt der 
auto Modus den exakten Explainer, der präzise SHAP-Werte auf Grundlage aller Daten und Koalitionen berechnet, was für Modelle mit einer geringeren Anzahl 
von Merkmalen effizient und praktikabel ist \cite[S. 40f]{Molnar_2023}. Eine Übersicht der zur Verfügung stehenden \textsf{shap}-Explainern ist in der
Dokumentation des Pakets zu finden\footnote{\shapurl}. 


\section{Einführung in den Datensatz}

In der vorliegenden Arbeit wird der SGEMM GPU Kernel Performance Datensatz verwendet, welcher die Laufzeitmessung eines Matrix-Matrix-Produkts 
\(A*B = C\) dokumentiert, wobei alle Matrizen die Größe 2048 x 2048 haben. 
Der Datensatz wurde erstellt, um verschiedene Konfigurationen eines parameterisierbaren SGEMM 
(Single-precision General Matrix Multiply) GPU-Kernels zu testen. Mit insgesamt 261400 möglichen Parameterkombinationen, 
von denen jede viermal ausgeführt wurde, bietet der Datensatz eine umfassende Grundlage zur Untersuchung der Laufzeitperformance.
Die Daten wurden von Enrique G. Paredes und Rafael Ballester-Ripoll vom Visualization and MultiMedia Lab der Universität 
Zürich gesammelt \cite{ballesterripoll2017sobol, Nugteren_2015}. 

Der Datensatz umfasst folgende Variablen \cite{misc_sgemm_gpu_kernel_performance_440}:

\begin{itemize}
    \item \textbf{MWG, NWG} (Matrix Widths for Global tile size): Dimensionen der 2D-Aufteilung einer Matrix auf Workgroup-Ebene. Mögliche Werte sind \{16, 32, 64, 128\}, was die Anzahl der Datenpunkte in jeder Dimension angibt.
    \item \textbf{KWG} (Kernel Width for Global tile size): Innere Dimension der 2D-Aufteilung auf Workgroup-Ebene. Mögliche Werte sind \{16, 32\}, ebenfalls als Anzahl der Datenpunkte.
    \item \textbf{MDIMC, NDIMC} (Local Workgroup Size): Dimensionen für die lokale Workgroup-Größe. Werte in \{8, 16, 32\}, repräsentieren die Anzahl der Threads pro Dimension innerhalb einer Workgroup.
    \item \textbf{MDIMA, NDIMB} (Local Memory Shape): Dimensionen, die die Form des lokalen Speichers bestimmen. Werte sind \{8, 16, 32\} und beziehen sich auf die Größe des Speicherblocks pro Dimension.
    \item \textbf{KWI} (Kernel Workgroup unrolling): Ein Faktor für das Kernel Loop Unrolling. Mögliche Werte sind \{2, 8\}, die den Grad der Schleifenvereinfachung angeben.
    \item \textbf{VWM, VWN} (Vector Widths for loading/storing): Vektorbreiten für das Laden und Speichern von Matrizen. Werte in \{1, 2, 4, 8\} repräsentieren die Anzahl der Datenpunkte, die simultan pro Operation gehandhabt werden.
    \item \textbf{STRM, STRN} (Enable stride): Binäre Einstellungen, die festlegen, ob ein Stride für den Speicherzugriff innerhalb eines Threads aktiviert ist. Werte \{0, 1\} repräsentieren ausgeschaltet bzw. eingeschaltet.
    \item \textbf{SA, SB} (Manual Caching): Binäre Variablen zur Aktivierung der manuellen Zwischenspeicherung von 2D Workgroup-Kacheln. Werte \{0, 1\} repräsentieren ausgeschaltet bzw. eingeschaltet.
    \item \textbf{Run1, Run2, Run3, Run4} (Performance Times): Vier unabhängige Laufzeiten (in Millisekunden) für jede Parameterkombination. Sie zeigen die Performanz des GPU-Kernels unter verschiedenen Einstellungen und Bedingungen. Die Zeiten variieren zwischen 13.25 ms und 3397.08 ms.
\end{itemize}

\section{Explorative Datenanalyse \& Datenaufbereitung}

Der Datensatz wurde in Python mithilfe der Bibliothek \textsf{pandas} als \textsf{Dataframe} eingelesen.
Die Zielvariable Runtime für die Regression wurde als Durchschnitt der vier unabhängigen Laufzeiten (Run1, Run2, Run3 und Run4) berechnet, 
um eine repräsentative Maßzahl für die Gesamtperformance zu erhalten.

Der vollständige Quellcode für das Einlesen der Daten sowie alle weiteren Analyseschritte ist 
im Anhang \ref{linreg} dieser Arbeit zu finden.

Tabelle \ref{tab:df-head} zeigt die ersten fünf Beobachtungen des Datensatzes:

\begin{table}[!h]
    \caption{Auszug aus dem SGEMM GPU Kernel Performance Datensatz}
    \footnotesize
    \begin{tabularx}{\textwidth}{Xrrrrrrrrrrrrrrr}
    \toprule
    Index & \rotatebox{90}{MWG} & \rotatebox{90}{NWG} & \rotatebox{90}{KWG} & \rotatebox{90}{MDIMC} & \rotatebox{90}{NDIMC} & \rotatebox{90}{MDIMA} & \rotatebox{90}{NDIMB} & \rotatebox{90}{KWI} & \rotatebox{90}{VWM} & \rotatebox{90}{VWN} & \rotatebox{90}{STRM} & \rotatebox{90}{STRN} & \rotatebox{90}{SA} & \rotatebox{90}{SB} & \rotatebox{90}{$\varnothing$ Runtime (ms)} \\
    \midrule
    0 & 16 & 16 & 16 & 8 & 8 & 8 & 8 & 2 & 1 & 1 & 0 & 0 & 0 & 0 & 116.3700 \\
    1 & 16 & 16 & 16 & 8 & 8 & 8 & 8 & 2 & 1 & 1 & 0 & 0 & 0 & 1 & 78.7050 \\
    2 & 16 & 16 & 16 & 8 & 8 & 8 & 8 & 2 & 1 & 1 & 0 & 0 & 1 & 0 & 80.5650 \\
    3 & 16 & 16 & 16 & 8 & 8 & 8 & 8 & 2 & 1 & 1 & 0 & 0 & 1 & 1 & 86.6375 \\
    4 & 16 & 16 & 16 & 8 & 8 & 8 & 8 & 2 & 1 & 1 & 0 & 1 & 0 & 0 & 118.6625 \\
    \bottomrule
    \end{tabularx}
    \label{tab:df-head}
    \normalsize\\
    Quelle: Eigene Darstellung auf Basis der Datengrundlage \cite{misc_sgemm_gpu_kernel_performance_440}.
\end{table}

In Tabelle \ref{tab:statistics} sind deskriptive Statistiken der verschiedenen Variablen des Datensatzes dargestellt. 

\begin{table}[!h]
    \caption{Auszug aus dem SGEMM GPU Kernel Performance Datensatz}
    \footnotesize
    \begin{tabularx}{\textwidth}{Xrrrrrrr}
    \toprule
    Variable & mean & std & min & 25\% & 50\% & 75\% & max \\
    \midrule
    MWG & 80.415364 & 42.469220 & 16.0000 & 32.0000 & 64.00 & 128.0000 & 128.0000 \\
    NWG & 80.415364 & 42.469220 & 16.0000 & 32.0000 & 64.00 & 128.0000 & 128.0000 \\
    KWG & 25.513113 & 7.855619 & 16.0000 & 16.0000 & 32.00 & 32.0000 & 32.0000 \\
    MDIMC & 13.935894 & 7.873662 & 8.0000 & 8.0000 & 8.00 & 16.0000 & 32.0000 \\
    NDIMC & 13.935894 & 7.873662 & 8.0000 & 8.0000 & 8.00 & 16.0000 & 32.0000 \\
    MDIMA & 17.371126 & 9.389418 & 8.0000 & 8.0000 & 16.00 & 32.0000 & 32.0000 \\
    NDIMB & 17.371126 & 9.389418 & 8.0000 & 8.0000 & 16.00 & 32.0000 & 32.0000 \\
    KWI & 5.000000 & 3.000006 & 2.0000 & 2.0000 & 5.00 & 8.0000 & 8.0000 \\
    VWM & 2.448609 & 1.953759 & 1.0000 & 1.0000 & 2.00 & 4.0000 & 8.0000 \\
    VWN & 2.448609 & 1.953759 & 1.0000 & 1.0000 & 2.00 & 4.0000 & 8.0000 \\
    STRM & 0.500000 & 0.500001 & 0.0000 & 0.0000 & 0.50 & 1.0000 & 1.0000 \\
    STRN & 0.500000 & 0.500001 & 0.0000 & 0.0000 & 0.50 & 1.0000 & 1.0000 \\
    SA & 0.500000 & 0.500001 & 0.0000 & 0.0000 & 0.50 & 1.0000 & 1.0000 \\
    SB & 0.500000 & 0.500001 & 0.0000 & 0.0000 & 0.50 & 1.0000 & 1.0000 \\
    Runtime & 217.571953 & 368.750161 & 13.3175 & 40.6675 & 69.79 & 228.3875 & 3341.5075 \\
    \bottomrule
    \end{tabularx}
    \label{tab:statistics}
    \normalsize
    \\ Quelle: Eigene Darstellung, \ref{linreg}.
\end{table}

Die Zielvariable Runtime weist eine erhebliche Variabilität und Spannweite auf. 
Um eine homogenere Verteilung für die Regression zu erreichen und den Einfluss 
von Ausreißern zu verringern, wurde sie einer logarithmischen Transformation unterzogen, 
wie auch von den Datensatzautoren empfohlen \cite{misc_sgemm_gpu_kernel_performance_440}. 
Grafik \ref{pic:box} illustriert die Verteilungen der Runtime mittels Boxplots und Histogrammen, 
vor und nach der log-Transformation, und verdeutlicht den Effekt der Normalisierung:

\begin{figure}[!h]
    \caption{Verteilungen der Runtime vor und nach log-Transformation}
    \includegraphics[width=1\textwidth]{../scripts/images/combined_runtime_plots.png}
    Quelle: Eigene Darstellung auf Basis der Datengrundlage \cite{misc_sgemm_gpu_kernel_performance_440}, \ref{linreg}.
    \label{pic:box}
\end{figure}

Die Korrelationsmatrix, die in Abbildung \ref{pic:corr} dargestellt ist, 
umfasst die Zielvariable Runtime, wodurch direkte Einblicke in die Beziehungen 
zwischen den Merkmalen und der Zielvariablen möglich sind. Die Koeffizienten variieren 
von -0.25 bis 0.46, was darauf hindeutet, dass einige Variablen eine moderate 
Korrelation mit der Runtime aufweisen. Positive Werte, wie der höchste Koeffizient 
von 0.46, implizieren, dass eine Erhöhung der entsprechenden Merkmalsausprägungen tendenziell 
mit längeren Ausführungszeiten verbunden ist. Negative Werte, wie der niedrigste Koeffizient 
von -0.25, deuten hingegen auf eine umgekehrte Beziehung hin, 
bei der höhere Merkmalsausprägungen mit kürzeren Laufzeiten korrelieren. 
Diese Korrelationen liefern wichtige Informationen für die Modellierung, da sie aufzeigen, 
welche Parameter möglicherweise einen stärkeren Einfluss auf die Laufzeit haben. 

\begin{figure}[!h]
    \caption{Korrelationsmatrix der Merkmale im Datensatz.}
    \includegraphics[width=1\textwidth]{../scripts/images/corr_gpu.png}
    Quelle: Eigene Darstellung, \ref{linreg}.
    \label{pic:corr}
\end{figure}

\section{Modellierung der linearen Regression}

Um die Beziehung zwischen den unabhängigen Variablen und der Zielvariablen 
Runtime zu untersuchen, wurde ein lineares Regressionsmodell aufgestellt. 
Zur Bewertung der Vorhersageleistung des Modells und zur Vermeidung von Overfitting wurde der 
Datensatz in zwei Teile aufgeteilt: 80\% der Daten dienten als Trainingsset zur 
Anpassung des Modells, während die restlichen 20\% als Testset verwendet wurden, 
um die Modellleistung anhand neuer, unbekannter Daten zu evaluieren. 
Diese Aufteilung erfolgte zufällig, aber reproduzierbar, durch Festlegen eines Seed-Werts 
für den Zufallszahlengenerator, der eine konsistente Teilung des Datensatzes ermöglicht.

Das Trainingsset wurde dazu verwendet, die Koeffizienten der linearen Regression zu schätzen, 
die den Einfluss jeder unabhängigen Variablen auf die Zielvariable quantifizieren. 
Anschließend wurde das Modell mit dem Testset geprüft, um seine Vorhersagegenauigkeit zu bewerten. 
Die Leistung des Modells wurde anhand von Metriken wie dem mittleren quadratischen Fehler (Mean Squared Error, MSE) gemessen, 
die ein Maß für die Abweichung der Modellvorhersagen von den tatsächlichen Werten darstellen.

Codeauschnitt \ref{code:model} und \ref{code:model-train} zeigen das Trainieren und Testen der zugrundeliegenden Daten 
eines linearen Regressionsmodells:

\lstinputlisting[language=Python,label=code:model, firstline=34, lastline=52, 
    caption={Initialisierung eines linearen Regressionsmodells, \ref{linreg}.}, captionpos=top]{../scripts/linreg.py}


\lstinputlisting[language=Python,label=code:model-train, firstline=183, lastline=190, 
    caption={Training und Testen eines linearen Regressionsmodells, \ref{linreg}.}, captionpos=top]{../scripts/linreg.py}


\section{Berechnung von SHAP-Werten}

Um SHAP-Werte zu berechnen, wird zunächst ein SHAP-Explainer-Objekt erstellt. In diesem Fall wird der Explainer 
von SHAP mit dem trainierten linearen Regressionsmodell und dem Trainingsdatensatz initialisiert. 
Anschließend werden die SHAP-Werte für die Testdaten berechnet, um die Beiträge der einzelnen Merkmale 
zu analysieren. Der Typ des Explainers wird durch die Art des übergebenen Modells bestimmt. 
Da in diesem Beispiel ein lineares Modell verwendet wird, wird automatisch ein geeigneter Explainer 
für lineare Modelle ausgewählt.

Das folgende Codeausschnitt \ref{code:shap} zeigt die Initialisierung des SHAP-Explainers und die Berechnung der SHAP-Werte:

\lstinputlisting[language=Python,label=code:shap, firstline=191, lastline=193, 
    caption={Berechnung von SHAP-Werten für das lineare Regressionsmodell, \ref{linreg}.}, captionpos=top]{../scripts/linreg.py}

Das Explainer-Objekt enthält neben den SHAP-Werten (.values), 
die die Einflüsse der einzelnen Merkmale der Testmenge auf die Modellvorhersage quantifizieren, 
auch die Basiswerte (.base\_values), die die durchschnittliche Vorhersage des Modells darstellen, 
und die ursprünglichen Merkmalsausprägungen (.data), die für die Berechnung dieser Werte verwendet wurden \cite[S. 51]{Molnar_2023}.

Dies bildet die Grundlage für den nächsten entscheidenden Schritt: 
die Visualisierung und tiefere Analyse dieser Werte. Die SHAP-Bibliothek bietet eine 
Reihe von leistungsstarken Visualisierungswerkzeugen, die es ermöglichen, die Auswirkungen 
der einzelnen Merkmale auf die Modellvorhersagen intuitiv und verständlich darzustellen. 

Im folgenden Kapitel \ref{chapter:results} werden diese Visualisierungen im Detail vorgestellt. 
Anhand von Beeswarm-Plots, Dependence-Plots und Bar-Plots werden die Ergebnisse der 
SHAP-Analyse dargestellt, die ein umfassendes Bild der Einflüsse und Wichtigkeiten der 
verschiedenen Merkmale im Kontext des linearen Regressionsmodells bieten.

Die Grafiken wurden mithilfe der \textsf{shap}-Bibliothek wie folgt erzeugt:

\lstinputlisting[language=Python,label=code:img, firstline=92, lastline=113, 
    caption={Erzeugen der SHAP Plots, \ref{linreg}.}, captionpos=top]{../scripts/linreg.py}


\chapter{Ergebnisse}
\label{chapter:results}

In diesem Kapitel werden die Resultate der angewandten linearen Regressionsanalyse 
zur Vorhersage der Druckfestigkeit von Beton dargestellt. 
Die Analyse berücksichtigt sowohl die geschätzten Koeffizienten des linearen Modells 
als auch verschiedene Evaluierungsmetriken wie den mittleren absoluten Fehler (MAE), 
den mittleren quadratischen Fehler (MSE) sowie die Bestimmtheitsmaße ($R^2$) 
für Trainings- und Testdaten. Diese Metriken liefern Aufschluss über die Güte des Modells 
und die Präzision der Vorhersagen.

\section{Lineares Regressionmodell}

Die Koeffizienten des Modells, die in 
Tabelle \ref{tab:model-coefficients} aufgeführt sind, zeigen, wie stark sich eine 
Einheitänderung jedes unabhängigen Merkmals auf die Runtime auswirkt. 
Positive Koeffizienten deuten auf eine Erhöhung der Runtime bei Zunahme der Variablen hin, 
während negative Koeffizienten eine Verringerung anzeigen. Der Intercept-Wert repräsentiert 
die geschätzte Runtime, wenn alle unabhängigen Variablen den Wert Null annehmen. Daraus ergbit sich
zusammen mit Gleichung \ref{eq:reg-model} die Regressionsgerade für das Modell.

\begin{table}[!h]
    \caption{Koeffizienten des linearen Regressionsmodells}
    \begin{tabularx}{\textwidth}{Xr}
    \toprule
    Merkmal ($\beta_j$) & Koeffizient \\
    \midrule
    Intercept ($\beta_0$) & 4.36596 \\
    cement & 0.75091 \\
    blast & 0.06610 \\
    ash & 0.02683 \\
    water & -0.92315 \\
    superplasticizer & 0.06410 \\
    coarse & 0.08554 \\
    fine &  -0.31901 \\
    age & 0.29090 \\
    \bottomrule
    \end{tabularx}
    \label{tab:model-coefficients}
    \\ Quelle: Eigene Darstellung, \ref{linreg}.
\end{table}

Die Modellmetriken, dargestellt in Tabelle \ref{tab:model-metrics}, 
geben Auskunft über die Vorhersagegenauigkeit und die Anpassungsgüte des Modells. 
Der MAE und RMSE liefern dabei Informationen über die durchschnittliche Größe 
der Fehler in den Vorhersagen, und die R²-Werte zeigen, wie gut das Modell die Varianz 
der Zielvariable erklärt.

\begin{table}[!h]
    \caption{Modellmetriken des linearen Regressionsmodells}
    \begin{tabularx}{\textwidth}{Xr}
    \toprule
    Metrik & Wert \\
    \midrule
    Mean Absolute Error (MAE) & 0.19 \\
    Mean Squared Error (MSE) & 0.06 \\
    Root Mean Squared Error (RMSE) & 0.24 \\
    Training Score (R²) & 0.7852 \\
    Test Score (R²) & 0.8155 \\
    \bottomrule
    \end{tabularx}
    \label{tab:model-metrics}
    \\ Quelle: Eigene Darstellung, \ref{linreg}.
\end{table}

Darüber hinaus wurden die tatsächlichen gegen die vorhergesagten Werte der Runtime
in Abbildung \ref{pic:residuals} visualisiert, die eine allgemeine Einschätzung der 
Modellgenauigkeit ermöglicht. Ein weiterer wichtiger Aspekt sind die Residuen des Modells. 
Die Residuen, also die Differenzen zwischen den tatsächlichen und vorhergesagten Werten, 
sollten idealerweise zufällig um Null verteilt sein und keine Muster aufweisen, 
die auf eine Verletzung der Modellannahmen hindeuten könnten.

\begin{figure}[!h]
    \caption{Residuenanalyse: Beziehung zwischen Vorhersagen und Abweichungen.}
    \includegraphics[width=1\textwidth]{../scripts/images/residuals.png}
    Quelle: Eigene Darstellung, \ref{linreg}.
    \label{pic:residuals}
\end{figure}

\section{Interpretation}

Die Interpretation der Ergebnisse der Modellanalyse bietet wertvolle Einsichten in die Daten 
und das Verhalten des linearen Regressionsmodells. Durch die Verwendung von SHAP-Werten wird 
es möglich, die Beiträge der einzelnen Merkmale zur Vorhersageleistung des Modells nicht nur 
auf globaler Ebene, sondern auch auf lokaler, individueller Ebene zu verstehen. 
Diese tiefgehende Analyse ermöglicht es, das Modell auf seine Fairness, 
Genauigkeit und Transparenz zu überprüfen.

\subsection{Bisherige Möglichkeiten der Feature Importance Interpreation}

TODO 

Überleitung warum SHAP jetzt so dolle ist.


\subsection{Lokale Interpretation}

Die lokale Interpretation konzentriert sich auf das Verständnis der Vorhersagen 
für eine einzelne Beobachtung aus dem Datensatz. Hierzu wird der SHAP Waterfall Plot eingesetzt, 
der eine visuelle Darstellung des Beitrags eines jeden Merkmals zu einer spezifischen Vorhersage liefert.

\begin{figure}[!h]
    \caption{SHAP Waterfall Plot}
    \includegraphics[width=1\textwidth]{../scripts/images/shap_waterfall_plot.png}
    Quelle: Eigene Darstellung, \ref{linreg}.
    \label{pic:shap_waterfall}
\end{figure}


In Abbildung \ref{pic:shap_waterfall} ist ein Waterfall Plot der ersten Beobachtung $x^{(0)}$ dargestellt, 
der die Zerlegung einer einzelnen Modellvorhersage zeigt. Der Plot beginnt mit dem Basiswert $\mathbb{E}[f(X)] = 4.45$, 
der durchschnittlichen Vorhersage des Modells. 

Von diesem Wert ausgehend, illustrieren die Balken, wie jede Merkmalausprägung – 
angezeigt durch die grauen Zahlen entlang der y-Achse – die Vorhersage $f(x_{j}^{(0)})$ beeinflusst. 
So steigert beispielsweise MWG mit einem Wert von $128$ die Vorhersage deutlich um $+0.72$, 
wohingegen SA mit einem Wert von $1$ die Vorhersage um $-0.09$ verringert.

Rote Balken repräsentieren Merkmale, die die Vorhersage erhöhen, während blaue Balken solche 
darstellen, die sie senken. Die Größe jedes Balkens zeigt das Ausmaß des jeweiligen Beitrags, 
und die abschließende Vorhersage $f(x) = 5.584$ wird am Ende der Kette dieser Effekte erreicht. 

Kleine positive und negative Beiträge von Merkmalen wie KWG ($32$), STRM ($0$), SB ($0$) und VWM ($4$) 
zeigen, wie feingranulare Anpassungen der Merkmalsausprägungen die Vorhersage leicht erhöhen oder senken können.

Für die Beobachtung $x^{(0)}$ führt die kumulative Abweichung der Merkmal-Effekte 
vom Basiswert $\mathbb{E}[f(X)] = 4.45$ zu einem tatsächlichen Modelloutput von $f(x) = 5.584$, 
was eine Differenz von $+1.134$ zwischen der durchschnittlichen Vorhersage 
und der spezifischen Vorhersage für diese Beobachtung offenlegt. 
Diese Differenz entspricht der Summe aller SHAP-Werte für diese konkrete Beobachtung \cite[S. 52f]{Molnar_2023}.

Da die Zielgröße einer logarithmischen Transformation unterzogen wurde, muss diese für die Interpreation wieder rückgängig gemacht werden. 
Dies bedeutet, dass der tatsächliche erwartete Wert der Laufzeit der Exponentialfunktion des prognostizierten Wertes entspricht, also $e^{4.45} \approx 85.63$ ms. 
Dieser Rücktransformationsprozess ist notwendig, um die Modellprognosen in der ursprünglichen Skala der Zielvariablen zu interpretieren.
Dies gilt darüberhinaus sowhl für die einzelnen SHAP-Werte, als auch für die konkrete Vorhersage $f(x) = 5.584$. 
Die prognostizierte Laufzeit für die Beobachtung $x^{(0)}$ beträgt folglich $e^{5.584} \approx 266.13$ ms.

Dies ermöglicht eine detaillierte Analyse, wie das Modell zu einer bestimmten Vorhersage kommt, 
und hilft dabei, die Beiträge und Interaktionen zwischen verschiedenen Merkmalen zu verstehen.

Die lokale Interpretation mittels SHAP-Werten ermöglicht zwar eine präzise Erklärung 
der Modellvorhersagen für individuelle Beobachtungen, jedoch stellt sich bei einer 
solchen Betrachtung das Problem der fehlenden Generalisierbarkeit. 
Lokale Analysen können dazu führen, dass spezifische Merkmal-Kontributionen überinterpretiert werden, 
ohne die übergeordneten Muster und Einflüsse zu berücksichtigen, 
die das Modellverhalten im gesamten Datensatz charakterisieren. 
Eine globale Interpretation ist daher erforderlich, um die Konsistenz und Zuverlässigkeit 
des Modells über verschiedene Beobachtungen hinweg zu erfassen. 

\subsection{Globale Interpretation}

Der folgende Abschnitt widmet sich dieser globalen Sichtweise und untersucht, 
wie die Merkmal-Beiträge sich im Kontext des gesamten Datensatzes darstellen lassen.

\begin{figure}[!h]
    \caption{SHAP Beeswarm Plot}
    \includegraphics[width=1\textwidth]{../scripts/images/shap_beeswarm_plot.png}
    Quelle: Eigene Darstellung, \ref{linreg}.
    \label{pic:shap_beeswarm}
\end{figure}

Der SHAP Beeswarm Plot in Abbildung \ref{pic:shap_beeswarm} bietet eine globale 
Sicht auf die Modellvorhersagen, indem er die Verteilung der SHAP-Werte für jedes Merkmals 
über alle Beobachtungen hinweg darstellt. Jeder Punkt repräsentiert eine Beobachtung aus dem Datensatz.
Die Farbe der Punkte zeigt die Merkmalsausprägungen an: hohe Werte in Rot und niedrige Werte in Blau. 
Die Position auf der x-Achse gibt den Einfluss des Merkmals auf die Modellvorhersage an. 
Positive SHAP-Werte (rechts von der Nulllinie) zeigen eine Erhöhung der Vorhersage an, 
während negative Werte (links von der Nulllinie) eine Verringerung bedeuten. 

Das Merkmal MWG mit den spezifischen Ausprägungen 16, 32, 64 und 128 zeigt eine Variabilität 
in seinem Einfluss auf die Modellvorhersage. Höhere Werte von MWG, insbesondere 128, 
sind mit einer Zunahme der Vorhersage (positive SHAP-Werte) assoziiert, 
was durch die rechtsseitigen Punkte in der Grafik dargestellt wird. 
iedrigere Werte wie 16 führen hingegen zu einer geringeren Vorhersage, 
erkennbar an den linksseitigen Punkten. Diese Streuung der Punkte zeigt, 
dass die Auswirkung von MWG auf die Vorhersage stark von seiner quantitativen Ausprägung abhängt.

Diese Darstellung ermöglicht es, die Merkmale zu identifizieren, 
die den größten Einfluss auf das Modell haben und wie dieser Einfluss über 
unterschiedliche Beobachtungen variiert.

\begin{figure}[!h]
    \caption{SHAP Bar Plot}
    \includegraphics[width=1\textwidth]{../scripts/images/shap_bar_plot.png}
    Quelle: Eigene Darstellung, \ref{linreg}.
    \label{pic:shap_bar}
\end{figure}

Der SHAP Bar Plot in Abbildung \ref{pic:shap_bar} illustriert die durchschnittliche 
Auswirkung jedes Merkmals auf das Modell, gemessen an der absoluten Größe der SHAP-Werte 
über alle Beobachtungen hinweg. Die Balken zeigen die durchschnittlichen Beiträge der 
Merkmale zur Vorhersage: Je länger der Balken, desto größer ist der Einfluss des jeweiligen Merkmale. 
Hier ist das Merkmal MWG mit dem höchsten durchschnittlichen SHAP-Wert (+0.51) das einflussreichste 
Merkmal, was auf eine starke positive Beziehung zur Zielvariablen hinweist. Die weiteren Merkmale 
folgen in absteigender Reihenfolge ihrer Bedeutung, wobei auch die Summe der Beiträge der 
fünf weiteren Merkmale am unteren Rand der Grafik dargestellt wird.

\include{content/chapters/8_prospect} 
\include{content/chapters/9_conclusion}

% Schalgwortverzeichnis (Index)
%\printindex

% Literaturverzeichnis
\nocite{*}
\bibliographystyle{plain}
\cleardoublepage
\phantomsection
\addcontentsline{toc}{chapter}{Literaturverzeichnis}
\bibliography{bibtex.bib}
\cleardoubleemptypage

% Abbildungsverzeichnis einbinden und ins Inhaltsverzeichnis
% WORKAROUND: tocloft und KOMA funktionieren zusammen nicht
% korrekt\phantomsection
\phantomsection 
\addcontentsline{toc}{chapter}{\listfigurename} 
\listoffigures
\cleardoubleemptypage

% Tabellenverzeichnis einbinden und ins Inhaltsverzeichnis
% WORKAROUND: tocloft und KOMA funktionieren zusammen nicht
% korrekt\phantomsection
\phantomsection
\addcontentsline{toc}{chapter}{\listtablename}
\listoftables
\cleardoubleemptypage

% Quellcodeverzeichnis einbinden und ins Inhaltsverzeichnis
\phantomsection
\addcontentsline{toc}{chapter}{Quellcodeverzeichnis}

%Define listing
\makeatletter
\begingroup\let\newcounter\@gobble\let\setcounter\@gobbletwo
  \globaldefs\@ne \let\c@loldepth\@ne
  \newlistof{listings}{lol}{\lstlistlistingname}
\endgroup
\let\l@lstlisting\l@listings
\makeatother
\setlength{\cftlistingsindent}{0em}
\renewcommand{\cftlistingsafterpnum}{\vskip0pt} %Spacing between entries
\renewcommand*{\cftlistingspresnum}{\lstlistingname~}
\settowidth{\cftlistingsnumwidth}{\cftlistingspresnum}
\renewcommand{\lstlistlistingname}{Quellcodeverzeichnis}
% Tabellenverzeichnis anpassen
\renewcommand{\lstlistingname}{Codeauschnitt}
\renewcommand{\cftlistingsaftersnum}{:}
% Breite des Nummerierungsbereiches [Codeauschnitt 1:]
\newlength{\codeLength}
\settowidth{\codeLength}{\bfseries\lstlistingname\cftlistingsaftersnum}
\addtolength{\codeLength}{5mm}
\setlength{\cftlistingsnumwidth}{\codeLength}
\lstlistoflistings
\cleardoubleemptypage


% Eidesstattliche Erklärung
\chapter*{Eidesstattliche Erklärung\markboth{Eidesstattliche Erklärung}{}}
% Eintrag in das Inhaltsverzeichnis 
\addcontentsline{toc}{chapter}{Eidesstattliche Erklärung}

Ich versichere an Eides statt, dass ich die vorstehende Arbeit selbständig 
und ohne fremde Hilfe angefertigt und mich anderer als der im beigefügten 
Verzeichnis angegebenen Hilfsmittel nicht bedient habe. Alle Stellen, die wörtlich oder sinngemäß aus Veröffentlichungen 
übernommen wurden, sind als solche kenntlich gemacht. Alle Internetquellen sind der Arbeit beigefügt. 

Des Weiteren versichere ich, dass ich die Arbeit vorher nicht in 
einem anderen Prüfungsverfahren eingereicht habe und dass die eingereichte 
schriftliche Fassung der auf dem elektronischen Speichermedium entspricht.


\vspace*{1.5cm} \par
\docOrt, den \docAbgabedatum
\vspace{1cm}


\noindent\rule[0cm]{5cm}{0.5pt}\\
\textsc{\docVorname~\docNachname}


%Zurücksetzen \chaptermark
\let\chaptermark\oldchaptermark

% Hier können Anhaenge angefuegt werden
\begin{appendices}
\chapter{Quellcode}
Quellcode
\newpage
%\appendixsection{Monatsbericht Februar}{content/attachments/report.pdf}
%\appendixsection{Monatsbericht Januar}{content/attachments/report.pdf}
\end{appendices}
\end{document} 
