

\nomenclature{$\mathcal{N}$}{Große Koalition (Koalition aller Spieler)}
\nomenclature{$\mathcal{S}$}{Koalition, mit $\mathcal{S} \subseteq \mathcal{N}$}
\nomenclature{$\vert\mathcal{S}\vert$}{Größe der Koalition $\mathcal{S}$}
\nomenclature{$1, \ldots, \vert\mathcal{N}\vert$}{Mögliche Spieler einer Koalition}
\nomenclature{$\mathcal{P}$}{Potenzmenge}
\nomenclature{$v$}{Koalitionsfunktion}
\nomenclature{$v(\mathcal{S})$}{Wert der Koalition $\mathcal{S}$}
\nomenclature{$\emptyset$}{Leere Menge (leere Koalition)}
\nomenclature{$\varphi_i (\mathcal{N}, v)$}{Der Shapley-Wert eines Spielers $i$}

\nomenclature{\( y \)}{Zielgröße, abhängige Variable}
\nomenclature{\( x_{ij} \)}{Wert der unabhängigen Variablen \( j \) für die Beobachtung \( i \)}
\nomenclature{\( \beta_j \)}{Koeffizient oder Gewichtung des Prädiktors \( j \) im linearen Modell}
\nomenclature{\( \epsilon_i \)}{Residuum für die Beobachtung \( i \)}
\nomenclature{\( \beta_0 \)}{Achsenabschnitt im linearen Model (Intercept)}
\nomenclature{\( p \)}{Anzahl der Prädiktoren im Modell}
\nomenclature{\( i \)}{Index für die Beobachtungseinheiten}
\nomenclature{\( j \)}{Index für die Prädiktoren}

\setlength{\nomlabelwidth}{2.5cm}
\renewcommand{\nomname}{Symbolverzeichnis}
\printnomenclature