\chapter*{Abstract\markboth{Abstract}{}}
\thispagestyle{empty}

Diese Masterarbeit erforscht die Interpretation linearer Modelle unter Verwendung von SHAP, 
einem Ansatz, der auf den Shapley-Werten aus der kooperativen Spieltheorie basiert. 
Die Arbeit beginnt mit einer historischen Kontextualisierung der Shapley-Werte, gefolgt von einer 
formalen Ableitung und einer Untersuchung ihrer axiomatischen Prinzipien. Es wird erläutert, wie SHAP 
aus den Shapley-Werten entwickelt wurde, um den Beitrag einzelner Merkmale zur Modellprognose hervorzuheben. 
Durch die Anwendung auf einen realen Datensatz mittels des Python-Pakets \textsf{shap} wird die praktische 
Umsetzbarkeit von SHAP demonstriert. Abschließend erfolgt eine kritische Diskussion über die Grenzen von SHAP 
und ein Ausblick auf seine potenzielle Rolle für transparente und nachvollziehbare Entscheidungsfindungen in der Datenwissenschaft.