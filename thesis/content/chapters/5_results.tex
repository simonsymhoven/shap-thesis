\chapter{Ergebnisse}
\label{chapter:results}

In diesem Kapitel werden die Resultate der angewandten linearen Regressionsanalyse 
zur Vorhersage der Druckfestigkeit von Beton dargestellt. 
Die Analyse berücksichtigt sowohl die geschätzten Koeffizienten des linearen Modells 
als auch verschiedene Evaluierungsmetriken wie den mittleren absoluten Fehler (engl. \textit{Mean Absolut Error}, MAE), 
den mittleren quadratischen Fehler (engl. \textit{Mean Squared Error}, MSE), sowie den  sowie die Bestimmtheitsmaße ($R^2$) 
für Trainings- und Testdaten. Diese Metriken liefern Aufschluss über die Güte des Modells 
und die Präzision der Vorhersagen. Zunächst werden die Koeffizienten 
des linearen Modells interpretiert. Anschließend wird gezeigt, 
wie SHAP-Werte eine tiefere und detailliertere Analyse ermöglichen.

\section{Lineares Regressionsmodell}

Die Koeffizienten eines linearen Modells stellen die Änderung der abhängigen Variable dar, 
in diesem Fall die Druckfestigkeit von Beton. Diese Änderung erfolgt für eine Einheitsänderung der 
unabhängigen Variablen, unter der Annahme, dass alle anderen Variablen konstant bleiben.

Positive Koeffizienten deuten auf eine Erhöhung der Beton-Druckfestigkeit 
bei Zunahme der Variablen hin, während negative Koeffizienten eine Verringerung anzeigen. 
Der Intercept-Wert repräsentiert die geschätzte Beton-Druckfestigkeit, 
wenn alle unabhängigen Variablen den Wert null annehmen. 

Daraus ergibt sich zusammen mit Gleichung \ref{eq:reg-model} die Regressionsgerade für das Modell.

\begin{table}[!h]
    \caption{Koeffizienten des linearen Regressionsmodells.}
    \begin{tabularx}{\textwidth}{Xr}
    \toprule
    Merkmal ($\beta_j$) & Koeffizient \\
    \midrule
    Intercept ($\beta_0$) & 4.36596 \\
    cement & 0.75091 \\
    blast & 0.06610 \\
    ash & 0.02683 \\
    water & -0.92315 \\
    superplasticizer & 0.06410 \\
    coarse & 0.08554 \\
    fine &  -0.31901 \\
    age & 0.29090 \\
    \bottomrule
    \end{tabularx}
    \label{tab:model-coefficients}
    \\ Quelle: Eigene Darstellung, \ref{linreg}.
\end{table}

Die Modellmetriken, dargestellt in Tabelle \ref{tab:model-metrics}, 
geben Auskunft über die Vorhersagegenauigkeit und die Anpassungsgüte des Modells. 

Der mittlere absolute Fehler (MAE) von $0.19$ zeigt an, dass die Vorhersagen des Modells im Durchschnitt um $0.19$ 
Einheiten vom tatsächlichen Wert abweichen. In Relation zu dem betrachteten Wertebereich (vgl. Abbildung \ref{pic:residuals}) der Zielvariablen stellt dies einen kleinen Prozentsatz 
der möglichen maximalen Abweichung dar, was auf eine zufriedenstellende Vorhersagegenauigkeit des Modells hindeutet.

Der mittlere quadratische Fehler (MSE) von $0.06$ misst die durchschnittliche quadratische Abweichung der Vorhersagen 
vom tatsächlichen Wert. Ein niedriger MSE-Wert wie $0.06$ deutet auf eine geringe Fehlervarianz und somit auf eine 
hohe Konsistenz der Modellvorhersagen hin.

Die Wurzel des mittleren quadratischen Fehlers (engl. \textit{Root Mean Squared Error}, RMSE) von $0.24$, bietet eine noch präzisere Darstellung 
der durchschnittlichen Fehlergröße. Da größere Fehler stärker gewichte werden, deutet ein Wert von $0.24$ darauf hin, 
dass das Modell tendenziell genaue Vorhersagen liefert, wobei größere Abweichungen seltener auftreten.

Die R²-Werte für Training und Test von $0.7852$ bzw. $0.8155$ sind ein Maß für die Güte der Modellanpassung. 
Ein R²-Wert nahe 1 deutet auf eine hohe Erklärungskraft des Modells hin. In diesem Fall erklären die R²-Werte von $0.7852$ und $0.8155$, 
dass das Modell einen signifikanten Anteil der Varianz in den Daten erfasst, 
was auf eine effektive Modellierung der Zusammenhänge in den Daten hindeutet.

\begin{table}[!h]
    \caption{Modellmetriken des linearen Regressionsmodells.}
    \begin{tabularx}{\textwidth}{Xr}
    \toprule
    Metrik & Wert \\
    \midrule
    Mean Absolute Error (MAE) & 0.19 \\
    Mean Squared Error (MSE) & 0.06 \\
    Root Mean Squared Error (RMSE) & 0.24 \\
    Training Score (R²) & 0.7852 \\
    Test Score (R²) & 0.8155 \\
    \bottomrule
    \end{tabularx}
    \label{tab:model-metrics}
    \\ Quelle: Eigene Darstellung, \ref{linreg}.
\end{table}

Darüber hinaus wurden die tatsächlichen gegen die vorhergesagten Werte
in Abbildung \ref{pic:residuals} visualisiert, welche eine allgemeine Einschätzung der 
Modellgenauigkeit ermöglichen. Ein weiterer Aspekt sind die Residuen des Modells. 
Die Residuen, also die Differenzen zwischen den tatsächlichen und vorhergesagten Werten, 
sollten idealerweise zufällig um null verteilt sein und keine Muster aufweisen, 
die auf eine Verletzung der Modellannahmen hindeuten könnten. Die Residuen aus Abbildung \ref{pic:residuals} 
erfüllen diese wünschenswerten Eigenschaften.

\begin{figure}[!h]
    \caption{Residuenanalyse: Beziehung zwischen Vorhersagen und Abweichungen.}
    \includegraphics[width=1\textwidth]{../scripts/images/residuals.png}
    Quelle: Eigene Darstellung, \ref{linreg}.
    \label{pic:residuals}
\end{figure}


\section{Interpretation der Koeffizienten \& Permutation der Merkmalrelevanz}

Die Koeffizienten beschreiben eine bedingte Assoziation. Das bedeutet, sie quantifizieren die 
Variation der Druckfestigkeit, wenn eine bestimmte unabhängige Variable 
verändert wird, während alle anderen unabhängigen Variablen konstant gehalten werden.

Die Koeffizienten sollten somit nicht als marginale Beiträge betrachtet werden. 
Das bedeutet, sie beschreiben nicht die Beziehung zwischen den Variablen unabhängig von anderen Einflussfaktoren. 
Stattdessen zeigen sie, wie sich die Druckfestigkeit ändert, wenn eine bestimmte unabhängige Variable variiert wird, 
während alle anderen konstant gehalten werden.

Abbildung \ref{pic:coef} zeigt die Koeffizienten des Regressionsmodells. Die Stärke des Einflusses einer 
unabhängigen Variable auf die abhängige Variable hängt von der Größe der Merkmalsausprägung ab. 
Ob ein bestimmtes Merkmal einen großen oder kleinen Einfluss auf die abhängige Variable hat, hängt
von den spezifischen Werten der Merkmale und der Streuung der Merkmalsausprägungen ab. 

\begin{figure}[!h]
    \caption{Koeffizienten des linearen Regressionsmodells.}
    \includegraphics[width=1\textwidth]{../scripts/images/coef.png}
    Quelle: Eigene Darstellung, \ref{linreg}.
    \label{pic:coef}
\end{figure}

Die logarthimische Transformation der Merkmale hat dieses Problem bereits 
teilweise gelöst, indem sie die Skalierung der Daten angepasst hat und alle Merkmale auf eine einheitliche Skala transformiert hat.
Daraus lassen sich erste Schlüsse der Merkmalsrelevanz ableiten. Im Vergleich zu allen anderen 
Koeffizienten tragen die Merkmale Flugasche, Superplastifikator, Hochofenschlacke und grober Zuschlag nur wenig zur abhängigen
Variable bei. Eine Erhöhung von Zement führt zu einer Steigerung der Druckfestigkeit. Im Gegensatz dazu führt eine Erhöhung des Wasseranteils 
zu einer Verringerung der Druckfestigkeit, immer unter der Vorraussetzung, das alle
anderen Merkmale konstant bleiben.

Abbildung \ref{pic:effect} zeigt einen Effektplot für die erste Beobachtung im Testdatensatz. Dieser 
ermöglicht es, den individuellen Effekt eines Merkmals auf die Vorhersage in Relation zur gesamten 
Verteilung aller Beobachtungen zu veranschaulichen. Die graphische Darstellung zeigt die Auswirkungen 
der Merkmale auf die Vorhersage für diese spezielle Beobachtung. 
Jedes Merkmal wird durch ein rotes Kreuz markiert. Dieses Kreuz stellt den Effekt des Merkmals auf die Vorhersage dar, 
welcher sich aus dem Produkt des Koeffizienten und der Merkmalsausprägung ergibt.
Die horizontale Position des Kreuzes zeigt den Wert des Effekts, während die vertikale Position das entsprechende Merkmal repräsentiert.

\begin{figure}[!h]
    \caption{Effektplot für Beoabchtung $x^{(0)}$ der Testmenge.}
    \includegraphics[width=1\textwidth]{../scripts/images/feature_effects_boxplot.png}
    Quelle: Eigene Darstellung, \ref{linreg}.
    \label{pic:effect}
\end{figure}

Der Effektplot ist ein nützliches Werkzeug, um zu verstehen, wie sich Änderungen in den Merkmalswerten auf die Vorhersage 
auswirken und wie dieser individuelle Effekt im Vergleich zur gesamten Datenverteilung steht. Für ein vollständiges Verständnis 
der Bedeutung jedes Merkmals im gesamten Datensatz, insbesondere im Kontext der Vorhersage der Druckfestigkeit von Beton, 
erweist sich jedoch eine globale Betrachtung des Einflusses der Merkmale als unerlässlich.

Ein bewährtes Verfahren zur Bestimmung der 
Merkmalsrelevanz ist die sogenannte Permutation der Merkmalrelevanz (engl. \textit{Permutation Feature Importance}, PFI). 
Diese Methode misst die Veränderung im Vorhersagefehler des Modells nach der 
zufälligen Vertauschung der Werte eines bestimmten Merkmals. 
Ein Merkmal gilt als wichtig, wenn die Vertauschung seiner Werte den Modellfehler erhöht, 
da das Modell stark auf dieses Merkmal für seine Vorhersagen angewiesen ist. 
Umgekehrt wird ein Merkmal als unwichtig betrachtet, wenn die Vertauschung seiner Werte 
den Modellfehler unverändert lässt, da das Modell das Merkmal für seine Vorhersagen ignoriert \cite[S. 157]{Molnar_2022}. 

Abbildung \ref{pic:permutation} zeigt die Merkmalsrelevanz auf der Test- und Trainingsmenge 
anhand der Veränderung des mittleren quadratischen Fehlers des Modells.

\begin{figure}[!h]
    \caption{Permutation der Merkmalrelevanz.}
    \includegraphics[width=1\textwidth]{../scripts/images/permutation_importance.png}
    Quelle: Eigene Darstellung, \ref{linreg}.
    \label{pic:permutation}
\end{figure}

Es ist festzustellen, dass die Merkmale grober Zuschlag, feiner Zuschlag und Flugasche keine erhebliche Veränderung im mittleren quadratischen
Fehler des Modells hervorrufen und diese somit im Vergleich zu den anderen Merkmalen als eher irrelevant angesehen werden können.

\section{Interpretation mit SHAP}

SHAP-Werte berücksichtigen die Wechselwirkungen zwischen den Merkmalen und 
quantifizieren den Beitrag jedes Merkmals zur Abweichung der Prognose im Mittel. 
Dies ermöglicht eine präzisere Interpretation, da die SHAP-Werte den Einfluss 
eines Merkmals auf die Vorhersage unter Berücksichtigung aller anderen Merkmale anzeigen. 
Durch die Verwendung von SHAP-Werten wird 
es möglich, die Beiträge der einzelnen Merkmale zur Vorhersageleistung des Modells nicht nur 
auf globaler Ebene, sondern auch auf lokaler, individueller Ebene zu verstehen. 

\subsection{Lokale Interpretation}

Die lokale Interpretation konzentriert sich auf das Verständnis der Vorhersagen 
für eine einzelne Beobachtung aus dem Datensatz.

\begin{figure}[!h]
    \caption{SHAP Partial Dependence Plot für Beobachtung $x^{(0)}$ der Testmenge.}
    \includegraphics[width=1\textwidth]{../scripts/images/shap_dependence_plot.png}
    Quelle: Eigene Darstellung, \ref{linreg}.
    \label{pic:shap_dependence}
\end{figure}

Im Partial Dependence Plot der Abbildung \ref{pic:shap_dependence} wird die Beziehung zwischen 
dem Merkmal Zement und der Zielvariable für die spezifische Beobachtung $x^{(0)}$ aus der Testmenge visualisiert. 
Das Histogramm zeigt die Verteilung der konkreten Merkmalsausprägungen über die Gesamtheit der Daten.
Die blaue Linie im Diagramm repräsentiert die modellierte Abhängigkeit der Vorhersage vom Merkmal Zement, 
unter Konstanthaltung aller anderen Merkmale. Der schwarze Punkt markiert die tatsächliche Ausprägung des 
Merkmals Zement ($5.035$) für die betrachtete Beobachtung und die korrespondierende Vorhersage 
des linearen Regressionsmodells. Die rote Linie illustriert die marginale Abweichung der Vorhersage 
von der durchschnittlichen Modellprognose $\mathbb{E}[f(X)] = 3.457$, ausgehend vom 
beobachteten Wert von Zement nach Gleichung \ref{eq:shap-single}:

\begin{align}
    \label{eq:shap0}
    \varphi_{\text{cement}}^{(0)}(f, x) &= \beta_{\text{cement}} (x_{\text{cement}}^{(0)} - \mathbb{E}[X_{\text{cement}}]) \\ \notag
                        &= 0.75091 \cdot (5.035 - 5.568) \\ \notag
                        &\approx -0.40
\end{align}

Codeausschnitt \ref{code:shap-idx} zeigt die manuelle Berechnung der für $\varphi_{\text{cement}}^{(0)}(f, x)$, basierend
auf den zur Verfügung stehenden Daten für das Merkmal Zement\footnote{Die Anwendung des SHAP-Explainers, 
insbesondere im Kontext eines Train-Test-Splits, stößt aufgrund der aktuellen Implementierung im SHAP-Python-Paket 
auf Herausforderungen. Die SHAP-Werte werden auf Basis des Datensatzes $X$ berechnet, 
was bei der Verwendung von Teilmengen wie $X_{\text{test}}$ zu geringfügigen Abweichungen führen kann, wie Berechnung \ref{eq:shap0} zeigt.
Dieses Verhalten ist insbesondere in Diskussionen auf GitHub zu erkennen, wie beispielsweise in dem Issue 
\url{https://github.com/shap/shap/issues/3456} diskutiert. Eine Erörterung dieses Problems in Bezug auf die interne Logik und die Verwendung des Pakets
findet sich in einem von mir verfassten Stack Overflow Beitrag, verfügbar unter 
\url{https://stackoverflow.com/questions/77820555/shap-partial-dependence-plot-misalignment-with-train-test-split-in-linear-regres}.}
.

\lstinputlisting[language=Python,label=code:shap-idx, linerange={248-265,330-331},
    caption={Berechnen der SHAP-Werte für ein Merkmal und eine Beobachtung, \ref{linreg}.}, captionpos=top]{../scripts/linreg.py}

Die Anordnung des schwarzen Punktes entlang der funktionalen Beziehung gibt den spezifischen Wert 
von Zement an und reflektiert, wie dieser Wert in den Kontext des gesamten Wertebereichs dieses Merkmals 
eingeordnet wird. Die vertikale Distanz zwischen der durchschnittlichen Vorhersage (dargestellt durch die horizontale 
gestrichelte Linie) und dem Punkt auf der funktionalen Abhängigkeit (blaue Linie) zeigt den Einfluss des Merkmals 
Zement auf die individuelle Vorhersage im Vergleich zum Modellmittelwert. Dieser Magnitude der roten Linie verkörpert den SHAP-Wert,
den marginalen Beitrag des Merkmals Zement zur Prognoseabweichung für die ausgewählte Beobachtung.

Während der Partial Dependence Plot einen wertvollen Einblick in die Modellabhängigkeit von 
einzelnen Merkmalen bietet, ist für ein umfassendes Verständnis der Modellvorhersage eine ganzheitliche 
Betrachtung aller Merkmale notwendig. Der SHAP Waterfall Plot adressiert diese Notwendigkeit, 
indem er eine kumulative Darstellung aller marginalen Beiträge liefert. Jedes Merkmal wird in 
Form einer Sequenz von Beiträgen visualisiert, beginnend mit dem Basiswert der Vorhersage, welcher 
durch die Addition oder Subtraktion der individuellen Merkmalsbeiträge schrittweise zur finalen Vorhersage 
modifiziert wird.

\begin{figure}[!h]
    \caption{SHAP Waterfall Plot für Beobachtung $x^{(0)}$ der Testmenge.}
    \includegraphics[width=1\textwidth]{../scripts/images/shap_waterfall_plot.png}
    Quelle: Eigene Darstellung, \ref{linreg}.
    \label{pic:shap_waterfall}
\end{figure}

In Abbildung \ref{pic:shap_waterfall} ist ein Waterfall Plot der ersten Beobachtung $x^{(0)}$ dargestellt, 
der die Zerlegung einer einzelnen Modellvorhersage zeigt. Der Plot beginnt mit dem Basiswert $\mathbb{E}[f(X)] = 3.457$, 
der durchschnittlichen Vorhersage des Modells. 

Von diesem Wert ausgehend, illustrieren die Balken, wie jede Merkmalausprägung – 
angezeigt durch die grauen Zahlen entlang der y-Achse – die Vorhersage $f(x_{j}^{(0)})$ beeinflusst. 
So steigert beispielsweise die Hochofenschlacke mit einem Wert von $4.982$ die Vorhersage um $+0.18$, 
wohingegen Zement mit einem Wert von $5.035$ die Vorhersage um $-0.39$ verringert.

Rote Balken repräsentieren Merkmale, die die Vorhersage erhöhen, während blaue Balken solche 
darstellen, die sie senken. Die Größe jedes Balkens zeigt das Ausmaß des jeweiligen Beitrags, 
und die abschließende Vorhersage $f(x) = 3.257$ wird am Ende der Kette dieser Effekte erreicht. 

Kleine positive und negative Beiträge der Merkmale wie feiner Zuschlag ($6.713$), grober Zuschlag ($6.908$) und dem Anteil von Wasser ($5.188$) 
in der Zusammensetzung des Betons zeigen, wie feingranulare Anpassungen der Merkmalsausprägungen die Vorhersage nicht verändern, 
leicht erhöhen oder senken können.

Für die Beobachtung $x^{(0)}$ führt die kumulative Abweichung der Merkmal-Effekte 
vom Basiswert $\mathbb{E}[f(X)] = 3.457$ zu einem tatsächlichen Modelloutput von $f(x) = 3.257$, 
was eine Differenz von $-0.2$ zwischen der durchschnittlichen Vorhersage 
und der spezifischen Vorhersage für diese Beobachtung offenlegt. 
Diese Differenz entspricht der Summe aller SHAP-Werte für diese konkrete Beobachtung \cite[S. 52f]{Molnar_2023}.

Da die Zielgröße einer logarithmischen Transformation unterzogen wurde, muss diese für die Interpreation wieder rückgängig gemacht werden. 
Dies bedeutet, dass der tatsächliche erwartete Wert der Druckfestigkeit der Exponentialfunktion des prognostizierten Wertes entspricht, also $e^{3.457} \approx 34.71$ MPa. 
Dieser Rücktransformationsprozess ist notwendig, um die Modellprognosen in der ursprünglichen Skala der Zielvariablen zu interpretieren.
Dies gilt darüberhinaus sowohl für die einzelnen SHAP-Werte, als auch für die konkrete Vorhersage $f(x) = 3.257$. 
Die prognostizierte Durckfestigkeit für die Beobachtung $x^{(0)}$ beträgt folglich $e^{3.257} \approx 25.97$ MPa.

Dies ermöglicht eine detaillierte Analyse, wie das Modell zu einer bestimmten Vorhersage kommt, 
und hilft dabei, die Beiträge und Interaktionen zwischen verschiedenen Merkmalen zu verstehen.

Die lokale Interpretation mittels SHAP-Werten ermöglicht zwar eine präzise Erklärung 
der Modellvorhersagen für individuelle Beobachtungen, jedoch stellt sich bei einer 
solchen Betrachtung das Problem der fehlenden Generalisierbarkeit. 
Lokale Analysen können dazu führen, dass spezifische Merkmal-Kontributionen überinterpretiert werden, 
ohne die übergeordneten Muster und Einflüsse zu berücksichtigen, 
die das Modellverhalten im gesamten Datensatz charakterisieren. 
Eine globale Interpretation ist daher erforderlich, um die Konsistenz und Zuverlässigkeit 
des Modells über verschiedene Beobachtungen hinweg zu erfassen. 

\subsection{Globale Interpretation}

Der SHAP Beeswarm Plot in Abbildung \ref{pic:shap_beeswarm} bietet eine globale 
Sicht auf die Modellvorhersagen, indem er die Verteilung der SHAP-Werte für jedes Merkmals 
über alle Beobachtungen hinweg darstellt. Jeder Punkt repräsentiert eine Beobachtung aus dem Datensatz.
Die Farbe der Punkte zeigt die Merkmalsausprägungen an: hohe Werte in Rot und niedrige Werte in Blau. 
Die Position auf der x-Achse gibt den Einfluss des Merkmals auf die Modellvorhersage an. 
Positive SHAP-Werte (rechts von der Nulllinie) zeigen eine Erhöhung der Vorhersage an, 
während negative Werte (links von der Nulllinie) eine Verringerung bedeuten. 

\begin{figure}[!h]
    \caption{SHAP Beeswarm Plot.}
    \includegraphics[width=1\textwidth]{../scripts/images/shap_beeswarm_plot.png}
    Quelle: Eigene Darstellung, \ref{linreg}.
    \label{pic:shap_beeswarm}
\end{figure}

Das Merkmal Alter zeigt eine hohe Variabilität in seinem Einfluss auf die Modellvorhersage. 
Höhere Werte von Zement sind mit einer Zunahme der Vorhersage (positive SHAP-Werte) assoziiert, 
was durch die rechtsseitigen Punkte in der Grafik dargestellt wird. 
Niedrigere Werte führen hingegen zu einer geringeren Vorhersage. Diese Streuung der Punkte zeigt, 
dass die Auswirkung von Zement auf die Vorhersage stark von seiner quantitativen Ausprägung abhängt, 
wie Abbildung \ref{pic:shap_scatter} verdeutlicht. 

\begin{figure}[!h]
    \caption{SHAP Scatter Plot.}
    \includegraphics[width=1\textwidth]{../scripts/images/shap_scatter_plot.png}
    Quelle: Eigene Darstellung, \ref{linreg}.
    \label{pic:shap_scatter}
\end{figure}

Der Scatter Plot zeigt die SHAP-Werte für Zement bezüglich ihrer quantitivaten Ausprägung über alle 
Daten. Steigende Merkmalsausprägungen von Zement führen zu größeren SHAP-Werten und somit zu einer Erhöhung der 
Modellprognose, was die ersten Analysen der Korrelationsmatrix aus Abbildung \ref{pic:corr} bestätigen. Der Einfluss des Wasseranteils
ist invers. Eine Erhöhung des Wasseranteils resultiert in geringeren SHAP-Werten und reduziert somit die 
Druckfestigkeit. 

Diese Darstellungen ermöglichen es, die Merkmale zu identifizieren, 
die den größten Einfluss auf das Modell haben und wie dieser Einfluss über 
unterschiedliche Beobachtungen variiert.

\begin{figure}[!h]
    \caption{SHAP Bar Plot.}
    \includegraphics[width=1\textwidth]{../scripts/images/shap_bar_plot.png}
    Quelle: Eigene Darstellung, \ref{linreg}.
    \label{pic:shap_bar}
\end{figure}

Der SHAP Bar Plot in Abbildung \ref{pic:shap_bar} illustriert die durchschnittliche 
Auswirkung jedes Merkmals auf das Modell, gemessen an der absoluten Größe der SHAP-Werte 
über alle Beobachtungen hinweg. Die Balken zeigen die durchschnittlichen Beiträge der 
Merkmale zur Vorhersage: Je länger der Balken, desto größer ist der Einfluss des jeweiligen Merkmals. 
Hier ist das Merkmal Alter mit dem höchsten durchschnittlichen SHAP-Wert ($+0.27$) das einflussreichste 
Merkmal, was auf eine starke positive Beziehung zur Zielvariablen hinweist. Die weiteren Merkmale 
folgen in absteigender Reihenfolge ihrer Bedeutung.

Es fällt auf, dass eine Übereinstimmung in der Reihenfolge der Merkmalsrelevanz zwischen 
der Permutation der Merkmalrelevanz aus Abbildung \ref{pic:permutation} und den SHAP-Werten aus Abbildung \ref{pic:shap_bar}
besteht, obwohl die zugrundeliegenden Interpretationen dieser beiden Methoden deutlich verschieden sind. 

Die Permutation der Merkmalrelevanz konzentriert sich darauf, die Veränderungen im Vorhersagefehler des Modells, 
speziell den mittleren quadratischen Fehler, zu messen, die eintreten, wenn die Werte eines Merkmals zufällig vertauscht werden \cite[S. 157]{Molnar_2022}. 
Diese Methode gibt Aufschluss darüber, wie stark das Modell auf das jeweilige Merkmal für seine Vorhersagegenauigkeit angewiesen ist. 
Ein wesentlicher Aspekt dieser Methode ist, dass sie nicht direkt die Wechselwirkungen zwischen den Merkmalen berücksichtigt. 
Sie zeigt vielmehr, wie wichtig ein Merkmal isoliert für die Gesamtleistung des Modells ist.

Im Gegensatz dazu bieten die SHAP-Werte einen tieferen Einblick in die Beiträge jedes Merkmals zur Vorhersageleistung des Modells. 
Sie quantifizieren den Einfluss eines Merkmals auf die Abweichung der Prognose vom Basiswert, also der durchschnittlichen Vorhersage des Modells. 
Hierbei wird nicht nur die individuelle Wichtigkeit jedes Merkmals hervorgehoben, sondern auch deren Wechselwirkungen mit anderen 
Merkmalen berücksichtigt. Diese Methodik ermöglicht es, sowohl eine globale als auch eine lokale Perspektive auf die Modellvorhersagen zu werfen. 
Während die globale Interpretation durchschnittliche Auswirkungen aller Merkmale aufzeigt, erlaubt die lokale Sichtweise, die Vorhersagen 
für einzelne Beobachtungen präzise zu erklären.

Diese unterschiedlichen Herangehensweisen und Interpretationen der Merkmalsrelevanz, einerseits durch die 
Permutation der Merkmalrelevanz und andererseits durch die SHAP-Werte, verdeutlichen die Komplexität und die Tiefe der Analyse, 
die für ein umfassendes Verständnis von Vorhersagemodellen erforderlich ist. Beide Methoden ergänzen sich gegenseitig und tragen dazu bei, 
ein vollständigeres Bild der Dynamiken innerhalb des Modells zu zeichnen.

SHAP-Werte bieten für die Prognose der Druckfestigkeit von Beton einen signifikanten Vorteil, 
da sie eine gerechte Verteilung des Vorhersagebeitrags über alle Merkmale gewährleisten. 
Diese gerechte Verteilung ist eine der Kernstärken der 
SHAP-Werte, die sie von anderen Methoden abhebt.

Die Druckfestigkeit von Beton ist das Ergebnis einer komplexen Interaktion verschiedener 
Materialkomponenten und Eigenschaften wie Zementgehalt, Wasser-Zement-Verhältnis, 
Zuschlagstoffe und Alter des Betons. Jede dieser Komponenten trägt unterschiedlich 
zur Endfestigkeit bei. Ihre Effekte können nicht isoliert betrachtet werden, 
da sie aufeinander abgestimmt werden müssen [S. 2]\cite{Nandhini2021}.

Andere Methoden, wie beispielsweise die Betrachtung von Regressionskoeffizienten, 
geben zwar Aufschluss über die Richtung und Stärke des Zusammenhangs zwischen Merkmalen 
und der Zielvariable, vernachlässigen jedoch die Interaktionseffekte zwischen den Merkmalen.

In der Praxis bedeutet dies für die Prognose der Druckfestigkeit von Beton, 
dass SHAP-Werte eine präzise Zuordnung der Einflussstärke jedes Bestandteils 
und Verarbeitungsmerkmals erlauben. Da die Festigkeit von Beton von einer Vielzahl von 
Faktoren abhängt, ermöglicht die granulare Aufschlüsselung der SHAP-Werte eine 
detailreiche Einsicht, welche Komponenten optimiert werden sollten, um die gewünschten 
Eigenschaften des Betons zu erreichen.

Im Kontext von industriellen Anwendungen und Forschung, wo Entscheidungen auf Grundlage 
der Modellvorhersagen getroffen werden, gewährleisten SHAP-Werte somit eine transparente 
und objektive Grundlage. Dies ist nicht nur für die Entwicklung von Betonmischungen 
von Bedeutung, sondern auch für die Einhaltung von Bauvorschriften und die Gewährleistung 
der Sicherheit. Durch die Verwendung von SHAP-Werten kann die Forschung im Bereich 
der Materialwissenschaften fundierter und zielgerichteter gestaltet werden, was zu einer 
effizienteren und effektiveren Materialentwicklung führt.