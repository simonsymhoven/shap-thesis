\chapter{Einleitung}

In einer Ära, in der datengetriebene Entscheidungsfindung und automatisierte Modelle zunehmend an Bedeutung gewinnen, 
wird die Fähigkeit zur Erklärung und Interpretation dieser Modelle immer wichtiger. Vor diesem Hintergrund gewinnen 
Shapley-Werte, ein Konzept aus der kooperativen Spieltheorie, zunehmend an Bedeutung in der Welt des maschinellen Lernens. 
In dieser Masterarbeit wird eine tiefgehende Untersuchung der Shapley-Werte vorgenommen, wobei ihr Potenzial für die 
Interpretation von Machine Learning-Modellen, insbesondere linearen Modellen, und ihre Anwendbarkeit auf reale Datensätze 
erforscht wird.

Die Arbeit beginnt mit einer ausführlichen Einführung in die Shapley-Werte, indem ihre historischen Ursprünge 
und ihre Verbindung zur kooperativen Spieltheorie beleuchtet werden. Durch die Analyse relevanter Literatur 
werden die theoretischen Grundlagen dieser Werte ergründet und ihre Bedeutung für die Interpretation von Modellen hervorgehoben.

Anschließend wird der Fokus auf die Anpassung und Erweiterung der Shapley-Werte für maschinelles Lernen gelegt. 
Hierbei wird untersucht, wie Shapley-Werte modifiziert werden können, um tiefere Einsichten in die Gewichtung 
und Bedeutung einzelner Merkmale innerhalb komplexer Machine Learning-Modelle zu liefern. Dabei wird ein Vergleich 
mit bestehenden Methoden, wie der Permutation Feature Importance, gezogen, um die Einzigartigkeit und den Mehrwert der 
Shapley-Werte hervorzuheben.

Ein wesentlicher Teil der Arbeit widmet sich der praktischen Anwendung der Shapley-Werte. Durch die Analyse 
eines realen Datensatzes wird die Methodik in der Praxis angewendet. Dies dient nicht nur dazu, die Wirksamkeit 
der Shapley-Werte zu demonstrieren, sondern auch, um ihre Aussagekraft und Anwendungsgrenzen in realen Szenarien 
zu evaluieren. Die Fallstudie zur Vorhersage der Druckfestigkeit von Beton bietet dabei ein konkretes Beispiel, 
anhand dessen die Nuancen und die Tiefe der durch SHAP ermöglichten Analysen veranschaulicht werden.

Abschließend werden die gewonnenen Erkenntnisse zusammengefasst und reflektiert. Es wird speziell die Rolle und das 
Potenzial der Shapley-Werte in diesem Kontext beleuchtet. Dabei werden auch die Grenzen und Herausforderungen, 
die mit der Anwendung von Shapley-Werten verbunden sind, kritisch diskutiert. Diese umfassende Betrachtung soll 
nicht nur die Bedeutung der Shapley-Werte für die aktuelle Forschung und Praxis hervorheben, sondern auch den Weg 
für weitere Untersuchungen in diesem dynamischen und zunehmend wichtigen Forschungsfeld ebnen.