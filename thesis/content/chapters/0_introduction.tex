\chapter{Einleitung}

In einer Zeit, in der datengetriebene Ansätze und automatisierte Modelle immer größere Relevanz erlangen, rückt die Notwendigkeit der Erklärbarkeit und Interpretierbarkeit von Modellen in den Vordergrund. Eines der vielversprechendsten Konzepte, das sich dieser Herausforderung annimmt, sind die sogenannten Shapley Values. Diese Masterarbeit erkundet die tiefgreifenden Konzepte der Shapley Values, ihre Anwendungen im Kontext von Machine Learning-Modellen und ihre praktische Umsetzung auf reale Datensätze.

Die Arbeit beginnt mit einer umfassenden Einführung in die Shapley Values und ihre historischen Wurzeln. Dabei wird insbesondere auf die kooperative Spieltheorie als Ursprung dieser Konzepte eingegangen. Anhand ausgewählter Literatur werden die theoretischen Grundlagen erörtert und der Forschungsstand auf diesem Gebiet aufgezeigt.

Im Anschluss daran wird die Brücke zur aktuellen Landschaft des maschinellen Lernens geschlagen. Es wird beleuchtet, wie die Shapley Values adaptiert werden können, um Einblicke in die Gewichtung von Merkmalen in komplexen Machine Learning-Modellen zu gewinnen. Dabei wird auf bestehende Methoden und Ansätze Bezug genommen und diskutiert, wie diese auf verschiedene Modelle angewendet werden können.

Ein zentraler Schwerpunkt der Arbeit liegt auf der praktischen Anwendung der Shapley Values. Ein realer Datensatz wird vorgestellt und die Methodik wird auf diesen angewendet, um die Wirksamkeit und Aussagekraft der Shapley Values in der Praxis zu evaluieren. Dies ermöglicht eine kritische Reflexion über die Stärken und Limitationen dieses Ansatzes im Kontext der Datenerklärung.

Abschließend werden die gewonnenen Erkenntnisse zusammengeführt und ein Ausblick auf zukünftige Entwicklungen und Forschungsrichtungen gegeben. Die Arbeit trägt somit dazu bei, das Verständnis für die Shapley Values als Instrument der Erklärbarkeit in komplexen Modellen zu vertiefen und ihre praktische Anwendbarkeit zu beleuchten.