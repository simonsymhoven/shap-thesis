\chapter{Einleitung}

In einer Zeit, in der datengetriebene Ansätze und automatisierte Modelle immer größere Relevanz erlangen, 
rückt die Notwendigkeit der Erklärbarkeit und Interpretierbarkeit von Modellen in den Vordergrund. 
Eines der vielversprechendsten Konzepte, das sich dieser Herausforderung annimmt, sind die sogenannten Shapley-Werte. 
Diese Masterarbeit erkundet die tiefgreifenden Konzepte der Shapley-Werte, ihre Anwendungen im Kontext von Machine Learning-Modellen, 
insbesondere linearer Modelle und ihre praktische Umsetzung auf reale Datensätze.

Die Arbeit beginnt mit einer umfassenden Einführung in die Shapley-Werte und ihre historischen Wurzeln. 
Dabei wird insbesondere auf die kooperative Spieltheorie als Ursprung dieser Konzepte eingegangen. 
Anhand ausgewählter Literatur werden die theoretischen Grundlagen erörtert.

Im Anschluss daran werden die Shapley-Werte im Kontext des maschinellen Lernens erweitert. 
Es wird beleuchtet, wie die Shapley-Werte adaptiert werden können, um Einblicke in die Gewichtung von Merkmalen in komplexen 
Machine Learning-Modellen zu gewinnen. Dabei wird auf bestehende Methoden und Ansätze Bezug genommen.

Ein zentraler Schwerpunkt der Arbeit liegt auf der praktischen Anwendung der Shapley-Werte. Ein realer Datensatz wird 
vorgestellt und die Methodik wird auf diesen angewendet, um die Wirksamkeit und Aussagekraft der Shapley-Werte in der 
Praxis zu evaluieren.

Abschließend werden die gewonnenen Erkenntnisse zusammengeführt und ein Ausblick auf zukünftige Entwicklungen und 
die Limitierungen der Shapley-Werte aufgezeigt.