\chapter{Von Shapley-Werten zu SHAP: Brückenschlag zur Modellinterpretation}

Im Rahmen der kooperativen Spieltheorie ermöglichen die Shapley-Werte eine faire Verteilung des kollektiv 
erwirtschafteten Nutzens auf die beteiligten Akteure. Diese Methodik findet eine analoge Anwendung 
in der Welt des maschinellen Lernens, um die Beiträge einzelner Merkmale zur Vorhersageleistung 
eines Modells zu bewerten. Hier wird die Terminologie der Shapley-Werte in den Kontext von Machine Learning 
Modellen übertragen, wobei jedes Merkmal als \glqq{}Spieler\grqq{} betrachtet wird, dessen Beitrag zur 
\glqq{}Auszahlung\grqq{} – der Vorhersage des Modells – evaluiert werden soll. 

\begin{table}[H]
    \footnotesize
    \begin{tabularx}{\textwidth}{XXX}
    \toprule
    Terminologie Konzept & Terminologie Machine Learning & Ausdruck \\
    \midrule
    Spieler & Merkmal Index & $j$ \\
    Anzahl aller Spieler & Anzahl aller Merkmale & $p$ \\
    Große Koalition & Menge aller Merkmale & $\mathcal{N} = \{1, \ldots, p\}$\\
    Koalition & Menge von Merkmalen & $\mathcal{S} \subseteq \mathcal{N}$ \\
    Größe der Koalition & Anzahl der Merkmale in der Koalition $\mathcal{S}$ & $|\mathcal{S}|$\\
    Spieler, die nicht in der Koalition sind & Merkmale, die nicht in der Koalition enthalten sind & $C: C = \mathcal{N} \setminus \mathcal{S}$ \\
    Koalitionsfunktion & Vorhersage für Merkmalswerte in der Koalition $\mathcal{S}$ abzüglich der Vorhersage im Mittel & $v_{f, x^{(i)}}(\mathcal{S}$)\\
    Auszahlung & Vorhersage für eine Beobachtung $x^{(i)}$ abzüglich der Vorhersage im Mittel & $f(x^{(i)}) -  \mathbb{E}(f(X))$\\
    Shapley-Wert & Beitrag des Merkmals $j$ zur Auszahlung des Modells für eine Beobachtung $x^{(i)}$& $\varphi_j^{(i)}(\mathcal{N}, f)$\\
    \bottomrule
    \end{tabularx}
    \caption{Terminologie der originären Shapley-Werte im Kontext des maschinellen Lernens \cite[S. 26]{Molnar_2023}.}
    \label{tab:shapley_terms}
\end{table}

Die Koalitionsfunktion $v_{f, x^{(i)}}(\mathcal{S})$ für ein gegebenes Model $f$ und eine Beobachtung $x^{(i)}$ ist definiert als:

\begin{align}
    v_{f, x^{(i)}}(\mathcal{S}) = \int_{\mathbb{R}} f(x^{(i)}_{\mathcal{S}} \cup X_{C}) d\mathbb{P}_{X_{C}} - \mathbb{E}(f(X))
\end{align}

Diese Funktion berechnet den erwarteten Wert der Vorhersage des Modells $f$, wenn nur eine Teilmenge $\mathcal{S}$ der 
Merkmale genutzt wird, um die Vorhersage für die spezifische Beobachtung $x^{(i)} \in \mathbb{R}^{p}$ zu treffen. 
Das Integral $\int_{\mathbb{R}}$ repräsentiert die Berechnung dieses erwarteten Wertes über alle möglichen Werte der Merkmale, 
die nicht in $\mathcal{S}$ enthalten sind ($X_C$), gewichtet durch deren Wahrscheinlichkeitsverteilung $\mathbb{P}_{X_{C}}$. 
Die Differenz zum Erwartungswert der Vorhersagen über alle Merkmale $\mathbb{E}(f(X))$ zeigt, 
wie viel die spezifische Menge an Merkmalen $\mathcal{S}$ zur Vorhersage beiträgt \cite[S. 221, S. 27]{Molnar_2022, Molnar_2023}.

Der marginale Beitrag eines Merkmals $j$ zu einer Koalition $\mathcal{S}$ ist dann:

\begin{align}
    v_{f, x^{(i)}}(\mathcal{S} \cup \{j\}) - v_{f, x^{(i)}}(\mathcal{S}) &= \int_{\mathbb{R}} f(x^{(i)}_{\mathcal{S} \cup \{j\}} \cup X_{C \setminus \{j\}}) d\mathbb{P}_{X_{C \setminus \{j\}}} - \mathbb{E}(f(X)) \\ \notag
    &\quad - \left( \int_{\mathbb{R}} f(x^{(i)}_{\mathcal{S}} \cup X_{C}) d\mathbb{P}_{X_{C}} - \mathbb{E}(f(X)) \right) \\ \notag
    &= \int_{\mathbb{R}} f(x^{(i)}_{\mathcal{S} \cup \{j\}} \cup X_{C \setminus \{j\}}) d\mathbb{P}_{X_{C \setminus \{j\}}} \\ \notag
    &\quad - \int_{\mathbb{R}} f(x^{(i)}_{\mathcal{S}} \cup X_{C}) d\mathbb{P}_{X_{C}}
\end{align}

Diese Gleichung beschreibt, wie sich der erwartete Wert der Vorhersage ändert, wenn das Merkmal $j$ zu der Menge der Merkmale $\mathcal{S}$ hinzugefügt wird \cite[S. 29]{Molnar_2023}.

Der Beitrag $\varphi_j^{(i)}(\mathcal{N}, f)$ eines Merkmals $j$ für eine Beobachtung $x^{(i)} \in \mathbb{R}^{p}$ für die Vorhersage $f(x^{(i)})$ ist gegeben als:

\begin{align}
    \label{eq:shap-eq}
    \varphi^{(i)}_{j} (\mathcal{N}, f) &= \sum_{\mathcal{S} \subseteq \mathcal{N} \setminus \{j\}} \frac{|\mathcal{S}|! \cdot (p - 1 - |\mathcal{S}|)!}{p!} \\ \notag
    &\quad \cdot \left( \int_{\mathbb{R}} f(x^{(i)}_{\mathcal{S} \cup \{j\}} \cup X_{C \setminus \{j\}}) d\mathbb{P}_{X_{C \setminus \{j\}}} -
    \int_{\mathbb{R}} f(x^{(i)}_{\mathcal{S}} \cup X_{C}) d\mathbb{P}_{X_{C}} \right) 
\end{align}

Diese Formel ist die zentrale Berechnung der SHAP-Werte im maschinellen Lernen. 
Sie summiert den gewichteten, marginalen Beitrag des Merkmals $j$ über alle möglichen Kombinationen der anderen Merkmale. 
Die Gewichtung berücksichtigt die Anzahl der Merkmale in der Koalition $\mathcal{S}$ und die Anzahl der verbleibenden Merkmale, 
die noch hinzugefügt werden könnten. Dies ergibt den durchschnittlichen Beitrag des Merkmals $j$ zur Vorhersage für die 
Beobachtung $x^{(i)}$ \cite[S. 29, 30]{Molnar_2023}.

Die Integration in der SHAP-Formel ist ein zentraler Schritt, um den erwarteten Beitrag jedes Merkmals unter 
Berücksichtigung der gesamten Verteilung der Daten zu ermitteln. 
In diesem Ansatz werden die Merkmale als Zufallsvariablen behandelt, und die Integration erfolgt über 
die Wahrscheinlichkeitsverteilungen dieser Zufallsvariablen. Durch das Berechnen der erwarteten Vorhersagewerte mit 
und ohne des jeweiligen Merkmals, unter Einbeziehung der Verteilung aller anderen Merkmale, 
ermöglicht SHAP eine präzise und umfassende Einschätzung des Einflusses jedes einzelnen Merkmals. 
Dieser Prozess der Marginalisierung, bei dem man über die Wahrscheinlichkeitsverteilungen der Merkmale integriert, 
erlaubt es, den Beitrag eines jeden Merkmals zu isolieren und unabhängig von der spezifischen Zusammensetzung 
der anderen Merkmale zu bewerten. Dies führt zu einer fairen und ganzheitlichen Bewertung der Beiträge aller Merkmale 
zur Vorhersage des Modells \cite[S. 28]{Molnar_2023}.

\section{Berechnung der SHAP-Werte unter Berücksichtigung der zugrundeliegenden Verteilung}
\label{sec:example}

Ein einfaches Beispiel soll helfen, die Anwendung von SHAP-Werten im 
Kontext maschinellen Lernens zu illustrieren\footnote{In Anlehnung an das Beispiel aus Kapitel 8.5.1 \glqq{}General Idea\grqq{} \cite[S.215f]{Molnar_2022}.}. Betrachtet wird ein fiktiver 
Immobilien-Datensatz mit drei Merkmalen: Größe des Hauses in Quadratmetern ($x_1$), Anzahl der Zimmer ($x_2$) 
und Entfernung zum Stadtzentrum in Kilometern ($x_3$). Es gibt zwei Beobachtungen in diesem Datensatz:

\begin{table}[H]
    \footnotesize
    \begin{tabularx}{\textwidth}{Xrrr}
    \toprule
     & $x_1$: Größe (in $m^2$) &  $x_2$: Anzahl Zimmer &  $x_3$: Entfernung zum Zentrum (in km) \\
    \midrule
    $x^{(1)}$ & 100 & 3 & 5 \\
    $x^{(2)}$ & 150 & 4 & 10 \\
    \bottomrule
    \end{tabularx}
    \caption{Merkmale von Beobachtungen in einem Immobilien-Datensatz.}
    \label{tab:example}
\end{table}

Angenommen das Modell $f(x^{(i)})$ prognostiziert den Preis eines Hauses in Euro als eine lineare Kombination der Merkmale:

\begin{equation}
    f(x^{(i)}) = 5x_1^{(i)} + 20x_2^{(i)} - 2x_3^{(i)}.
\end{equation}

Die Vorhersagen für die beiden Beobachtungen lauten dann:

\begin{align}
    \label{eq:fx1}
    f(x^{(1)}) &= 5x_1^{(1)} + 20x_2^{(1)} - 2x_3^{(1)} \\ \notag
        &= 5 \cdot 100 + 20 \cdot 3 - 2 \cdot 5 \\ \notag
        &= 550\,\text{\euro} 
\end{align}

und 

\begin{align}
    f(x^{(2)}) &= 5x_1^{(2)} + 20x_2^{(2)} - 2x_3^{(2)} \\ \notag
        &= 5 \cdot 150 + 20 \cdot 4 - 2 \cdot 10 \\ \notag
        &= 810\,\text{\euro}. 
\end{align}

Die erwartete Auszahlung des Modells $\mathbb{E}(f(X))$ wird berechnet als:
\begin{align}
    \label{eq:efx}
    \mathbb{E}(f(X)) &= 5 \cdot \mathbb{E}(X_1) + 20 \cdot \mathbb{E}(X_2) - 2 \cdot \mathbb{E}(X_3) \\ \notag
                     &= 5 \cdot 125 + 20 \cdot 3,5 - 2 \cdot 7,5 \\ \notag
                     &= 680 \,\text{\euro}
\end{align}

Sei $\mathcal{N} = \{1, 2, 3\}$ die Menge aller Merkmale und die Beobachtung $x^{(1)} = [100, 3, 5]$. 
Der SHAP-Wert für jedes Merkmal $j \in \mathcal{N}$ wird unter Berücksichtigung der Verteilung der 
Daten und der Formel \ref{eq:shap-eq} berechnet:

\begin{align}
    \label{eq:shap-formular}
    \varphi^{(1)}_{j} (\mathcal{N}, f) &= \sum_{\mathcal{S} \subseteq \mathcal{N} \setminus \{j\}} \frac{|\mathcal{S}|! \cdot (p - 1 - |\mathcal{S}|)!}{p!} \\ \notag
    &\quad \cdot \left( \int_{\mathbb{R}} f(x^{(1)}_{\mathcal{S} \cup \{j\}} \cup X_{C \setminus \{j\}}) d\mathbb{P}_{X_{C \setminus \{j\}}} -
    \int_{\mathbb{R}} f(x^{(1)}_{\mathcal{S}} \cup X_{C}) d\mathbb{P}_{X_{C}} \right) 
\end{align}

wobei $p = |\mathcal{N}| = 3$ die Anzahl der Merkmale ist und $X_C$ die Menge der Merkmale 
außerhalb der Koalition $\mathcal{S}$ repräsentiert. Die Integrale repräsentieren die erwartete 
Vorhersage des Modells über die Verteilung der nicht in der Koalition enthaltenen Merkmale.

In linearen Modellen, unter der Prämisse, dass die Merkmale unabhängig voneinander und gleichverteilt sind, 
ist es möglich, die Berechnung der SHAP-Werte zu vereinfachen. Anstelle der komplexen Integration 
über die Verteilungen aller Merkmale, kann der Fokus auf die Unterschiede in den Modellvorhersagen gelegt werden, 
die sich aus dem Hinzufügen oder Entfernen einzelner Merkmale ergeben. 
Hierbei wird anstelle der spezifischen Werte der nicht in der betrachteten Koalition enthaltenen Merkmale 
Erwartungswerte herangezogen. Diese Vereinfachung ermöglicht es, den Einfluss jedes Merkmals auf 
die Modellvorhersage auf eine direktere und rechnerisch weniger aufwendige Weise zu erfassen.
Diese Vereinfachung ist für lineare Modelle angemessen, da die Auswirkungen jedes Merkmals 
auf die Vorhersage des Modells additiv und unabhängig sind. Bei komplexeren, 
nichtlinearen Modellen ist eine detailliertere Berechnung erforderlich, 
die oft auf numerischen Methoden oder Annäherungen basiert, mehr dazu in Kapitel \ref{sec:estimators}.

Der Beitrag durch das Hinzufügen des Merkmals $x_1$ zur bestehenden Koalition $\mathcal{S} = \{x_2\}$ wird
nach Formel \ref{eq:shap-formular} berechnet als:

\begin{align}
    \varphi^{(1)}_{1}(\{x_2\}, f) &= \frac{1! \cdot (3 - 1 - 1)!}{3!} \\ \notag
        &\quad \cdot \int f(x_1, x_2 , X_3) d\mathbb{P}(X_3) - \int f(X_1, x_2, X_3) d\mathbb{P}(X_1, X_3)
\end{align}

Da \( X_1 \) und \( X_3 \) unabhängig und gleichmäßig verteilt sind, ersetzen wir \( X_2 \) und \( X_3 \) durch ihre Erwartungswerte:

\begin{align}
    \varphi^{(1)}_{1}(\{x_2\}, f) &= \frac{1}{6} \Big(f(x_1, x_2, \mathbb{E}(X_3)) - f(\mathbb{E}(X_1), x_2, \mathbb{E}(X_3))\Big) \\ \notag
        &= \frac{1}{6} \Big( f(100, 3, 7.5) - f(125, 3, 7.5) \Big) \\ \notag
        &= \frac{1}{6} \Big( (5 \cdot 100 + 20 \cdot 3 - 2 \cdot 7,5) - (5 \cdot 125 + 20 \cdot 3 - 2 \cdot 7,5) \Big) \\ \notag
        &= \frac{1}{6} (545 - 670) \\ \notag
        &= \frac{1}{6} (-125)
\end{align}
 
Die in Tabelle \ref{tab:shapley_marginal_features} dargestellten Kombinationen illustrieren die marginalen Beiträge und SHAP-Werte 
für jedes Merkmal in jeder möglichen Koalition von Merkmalen, bezogen auf die Beobachtung $x^{(1)}$. 
Diese Analyse ist ebenso auf die Beobachtung $x^{(2)}$ anwendbar und erfordert eine analoge Vorgehensweise.

\begin{table}[H]
    \footnotesize
    \begin{tabularx}{\textwidth}{XXrrrrr}
    \toprule
    $x_{j}$ & $\mathcal{S}$ & $v_{f, x^{(i)}}(\mathcal{S})$ & $v_{f, x^{(i)}}(\mathcal{S} \cup \{j\})$ & $v_{f, x^{(i)}}(\mathcal{S} \cup \{j\}) - v_{f, x^{(i)}}(\mathcal{S})$ & Gewicht & $\varphi_{j}^{(1)}(\mathcal{S}, f)$\\
    \midrule
    $x_1$ & $\emptyset$ & 680 & 555 & -125 & $\frac{1}{3}$ & -41,67 \\
    $x_1$ & $\{x_2\}$ & 670 & 545 & -125 & $\frac{1}{6}$ & -20.83 \\
    $x_1$ & $\{x_3\}$ & 685 & 560 & -125 & $\frac{1}{6}$ & -20.83 \\
    $x_1$ & $\{x_2, x_3\}$ & 675 & 550 & -125 & $\frac{1}{3}$ & -41,67 \\
    $x_2$ & $\emptyset$ & 680 & 670 & -10 & $\frac{1}{3}$ & -3,33 \\
    $x_2$ & $\{x_1\}$ & 555 & 545 & -10 & $\frac{1}{6}$ & -1,67 \\
    $x_2$ & $\{x_3\}$ & 685 & 675 & -10 & $\frac{1}{6}$ & -1,67 \\
    $x_2$ & $\{x_1, x_3\}$ & 560 & 550 & -10 & $\frac{1}{3}$ & -3,33 \\
    $x_3$ & $\emptyset$ & 680 & 685 & 5 & $\frac{1}{3}$ & 1,67 \\
    $x_3$ & $\{x_1\}$ & 555 & 560 & 5 & $\frac{1}{6}$ & 0,83 \\
    $x_3$ & $\{x_2\}$ & 670 & 675 & 5 & $\frac{1}{6}$ & 0,83 \\
    $x_3$ & $\{x_1, x_2\}$ & 545 & 550 & 5 & $\frac{1}{3}$ & 1,67 \\
    \bottomrule
    \end{tabularx}
    \caption{Marginalbeiträge der einzelnen Merkmale zu den möglichen Koalitionen für die Beobachtung $x^{(1)}$.}
    \label{tab:shapley_marginal_features}
\end{table}


\section{Axiome}

Die in Tabelle \ref{tab:shapley_marginal_features} präsentierten Ergebnisse bieten eine Grundlage, 
um die Konformität der SHAP-Werte mit den etablierten Axiomen der Shapley-Werte, wie sie im vorherigen 
Kapitel \ref{sec:axiome-shapley} diskutiert wurden, zu beurteilen. Die Axiome der SHAP-Werte stellen 
eine adaptierte und kontextualisierte Anwendung dieser Prinzipien auf die Interpretation von Modellvorhersagen dar \cite{NIPS2017_8a20a862}.


\paragraph{Effizienz}

Das Effizienzaxiom besagt, dass die Summe der SHAP-Werte aller Features für eine gegebene Beobachtung $x^{(i)}$ 
gleich der Differenz zwischen dem Modelloutput für diese Beobachtung $f(x^{(i)})$ 
und dem durchschnittlichen Modellvorhersage $\mathbb{E}(f(X))$ sein muss:

\begin{align}
    \sum_{j=1}^{p}\varphi_j^{(i)}(\mathcal{N}, f) = f(x^{(i)}) - \mathbb{E}(f(X)),
\end{align}

\cite[S. 221]{Molnar_2022}. Für die Beobachtung $x^{(1)}$ aus Kapitel \ref{sec:example} und den Berechnungen 
für $f(x^{(1)})$ (Gleichung \ref{eq:fx1}), sowie $\mathbb{E}(f(X))$ (Gleichung \ref{eq:efx}) ergibt sich:

\begin{align}
    \sum_{j=1}^{3}\varphi_j^{(1)}(\mathcal{N}, f) &=  -130 \\ \notag
    f(x^{(1)}) - \mathbb{E}(f(X)) &= 550 - 680 = -130,   
\end{align}

womit das Effizienzaxiom erfüllt ist. Das beduetet, dass die Differenz der Vohersage 
einer konkreten Beobachtung zur durchschnittlichen 
Modellvorhersage auf alle Merkmale verteilt ist.

\section{SHAP Estimators}
\label{sec:estimators}