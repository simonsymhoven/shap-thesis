\chapter{Fazit}

Das Fazit der vorliegenden Arbeit reflektiert die erzielten Erkenntnisse 
aus der Anwendung des linearen Regressionsmodells zur Vorhersage der Laufzeiten von GPU-Kernen. 
Durch die Integration von SHAP-Werten konnte eine tiefergehende Interpretation der Modellvorhersagen 
erreicht werden, die über die traditionelle statistische Analyse hinausgeht. Es wurde demonstriert,
dass die Koeffizienten des linearen Modells einen direkten Einblick in die Beziehung zwischen den 
unabhängigen Variablen und der Zielvariable bieten, während die SHAP-Werte es ermöglichen, 
diese Beziehungen auf einer granularen Ebene zu interpretieren.

Die Visualisierungen, insbesondere der Waterfall- und Beeswarm-Plots,
haben eine klare und intuitive Darstellung der Merkmals-Beiträge ermöglicht, die für die 
Verständlichkeit des Modellverhaltens von entscheidender Bedeutung ist. Die Arbeit hat auch aufgezeigt, 
dass die lokale Interpretation, obwohl sie präzise und aufschlussreich für einzelne Vorhersagen ist, 
durch die globale Perspektive ergänzt werden muss, um allgemeingültige Schlussfolgerungen über das Modellverhalten zu ziehen.

Insgesamt hat die Untersuchung gezeigt, dass die Kombination aus linearen Regressionsmodellen und SHAP-basierter 
Interpretation ein leistungsstarkes Werkzeug darstellt, um sowohl die Vorhersagegenauigkeit als auch die Transparenz 
und Nachvollziehbarkeit von maschinellen Lernmodellen zu verbessern. Die gewonnenen Einsichten können genutzt werden, 
um das Modell weiter zu verfeinern, und bieten eine Grundlage für fundierte Entscheidungen bei der Optimierung von GPU-Kerneln. 
Die Methodik und die Ergebnisse dieser Arbeit tragen somit zur Verbesserung der Modellverständlichkeit bei und unterstützen 
die Entwicklung von zuverlässigeren und gerechteren maschinellen Lernsystemen.