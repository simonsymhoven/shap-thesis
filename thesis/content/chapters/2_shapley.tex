\chapter{Theorie der Shapley-Werte}

\section{Beispiel: Designwettbewerb}

Angenommen, drei Teilnehmer, Anna, Ben und Carla, haben als Team kooperiert und den ersten Platz bei einem Designwettbewerb belegt\footnote{In Anlehung an Beispiel aus Kapitel 4 \glqq{}Who's going to pay for that taxi?\grqq{}\cite[S.17-20]{Molnar_2023}}. 
Dieser Erfolg führt zu einem Gesamtgewinn von 1000 \euro. Das Preisgeld für den zweiten Platz beträgt 750 \euro und 500 \euro für den dritten Platz.
Die Herausforderung besteht nun darin, den Gewinn auf eine Weise zu verteilen, die den individuellen Beitrag jedes Teilnehmers 
zur Erzielung des ersten Platzes gerecht widerspiegelt.

Die Situation wird komplizierter, wenn man bedenkt, dass jeder Teilnehmer unterschiedlich zu dem Erfolg 
beigetragen hat und ihre individuellen Leistungen auch zu verschiedenen Ausgängen geführt hätten, 
wenn sie alleine oder in anderen Teilkonstellationen angetreten wären.

Um eine faire Aufteilung des Preisgeldes zu erreichen, betrachten wir die hypothetischen Gewinne, 
die Anna, Ben und Carla erzielt hätten, wenn sie in unterschiedlichen Konstellationen am Wettbewerb teilgenommen hätten.
Tabelle \ref{tab:shapley_example} zeigt die gegegbene Gewinnverteilung der verschiedenen Koalitionen. Die Koalition $\emptyset$ entspricht
dabei der leeren Koalition -- der Nichtteilnahme an dem Wettbewerb.

\begin{table}[H]
  \footnotesize
  \begin{tabularx}{\textwidth}{Xrr}
  \toprule
  Koalition & Gewinn & Bemerkung \\
  \midrule
  $\emptyset$ & 0 \euro & Keine Teilnahme \\
  $\{$Anna$\}$ & 500 \euro & 3. Platz als Einzelteilnehmerin \\
  $\{$Ben$\}$ & 750 \euro & 2. Platz als Einzelteilnehmer \\
  $\{$Carla$\}$ & 0 \euro & Kein Gewinn als Einzelteilnehmerin \\
  $\{$Anna, Ben$\}$ & 750 \euro & 2. Platz als Team ohne Carla \\
  $\{$Anna, Carla$\}$ & 750 \euro & 2. Platz als Team ohne Ben \\
  $\{$Ben, Carla$\}$ & 500 \euro & 3. Platz als Team ohne Anna \\
  $\{$Anna, Ben, Carla$\}$ & 1000 \euro & 1. Platz als Gesamtteam \\
  \bottomrule
  \end{tabularx}
  \caption{Potenzielle Gewinne für verschiedene Teilnehmerkonstellationen im Designwettbewerb.}
  \label{tab:shapley_example}
\end{table}

Zur Berechnung der Shapley-Werte ist es erforderlich, den marginalen Beitrag jedes Spielers zu erfassen.
Marginalbeiträge in der Spieltheorie, und speziell im Kontext der Shapley-Werte, sind die zusätzlichen Beiträge, 
die ein Spieler (Teilnehmer) zum Gesamtgewinn einer Koalition beiträgt, wenn er dieser beitritt. 
Die Berechnung des marginalen Beitrags eines Teilnehmers erfolgt, indem man den Wert der Koalition ohne diesen Teilnehmer 
vom Wert der Koalition mit dem Teilnehmer subtrahiert \cite[S. 18]{Molnar_2023}.

In diesem Beispiel mit Anna, Ben und Carla, die an einem Designwettbewerb teilnehmen, ist der marginale Beitrag von 
Anna zur Koalition von $\{$Ben$\}$ der zusätzliche Wert, den sie einbringt, wenn sie sich Ben anschließt, 
ausgehend von Bens individuellem Gewinn.

Die Tabelle \ref{tab:shapley_marginal} illustriert den Gewinn jeder möglichen Koalition ohne den 
betrachteten Spieler und den neuen Gesamtgewinn, sobald dieser Spieler der Koalition beitritt. 
Der marginale Beitrag jedes Spielers wird dann als die Differenz zwischen diesen beiden Werten 
berechnet und gibt Aufschluss über den individuellen Wertbeitrag zum gemeinschaftlichen Erfolg.

\begin{table}[H]
  \footnotesize
  \begin{tabularx}{\textwidth}{XXrrr}
  \toprule
  Teilnehmer & Zur Koalition & Gewinn vorher & Gewinn nachher & Marginalbeitrag \\
  \midrule
  Anna & $\emptyset$ & 0 \euro & 500 \euro & 500 \euro \\
  Anna & $\{$Ben$\}$ & 750 \euro & 750 \euro & 0 \euro \\
  Anna & $\{$Carla$\}$ & 0 \euro & 750 \euro & 750 \euro \\
  Anna & $\{$Ben, Carla$\}$ & 500 \euro & 1000 \euro & 500 \euro\\
  Ben & $\emptyset$ & 0 \euro & 750 \euro & 750 \euro \\
  Ben & $\{$Anna$\}$ & 500 \euro & 750 \euro & 250 \euro\\
  Ben & $\{$Carla$\}$ & 0 \euro & 500 \euro & 500 \euro\\
  Ben & $\{$Anna, Carla$\}$ & 750 \euro &  1000 \euro & 250 \euro\\
  Carla & $\emptyset$ & 0 \euro & 0 \euro & 0 \euro \\
  Carla & $\{$Anna$\}$ & 500 \euro & 750 \euro & 250 \euro \\
  Carla & $\{$Ben$\}$ & 750 \euro & 500 \euro & -250 \euro \\
  Carla & $\{$Anna, Ben$\}$ & 750 \euro & 1000 \euro & 250 \euro \\
  \bottomrule
  \end{tabularx}
  \caption{Marginalbeiträge der einzelnen Teilnehmer zu den möglichen Koalitionen.}
  \label{tab:shapley_marginal}
\end{table}

Nachdem die marginalen Beiträge jedes Teilnehmers für die verschiedenen Koalitionen festgestellt wurden, 
ist der nächste Schritt, die Shapley-Werte zu bestimmen, welche eine faire Aufteilung des Gesamtgewinns 
erlauben. Hierzu wird jede mögliche Reihenfolge (Permutation) betrachtet, in der die Spieler der 
Koalition beitreten könnten. Jede dieser Permutationen liefert unterschiedliche marginale Beiträge 
für die Spieler, je nach der Reihenfolge ihres Beitritts.

Im Falle dieses Beispiels mit Anna, Ben und Carla bedeutet dies, dass alle möglichen Reihenfolgen 
berücksichtigt werden müssen, in denen sie zum ersten Platz beigetragen haben könnten. 
Die Shapley-Werte werden dann als Durchschnitt der marginalen Beiträge über alle Permutationen berechnet. 
Dies gewährleistet, dass jeder Spieler einen Anteil des Preisgeldes erhält, der seinem durchschnittlichen 
Beitrag zum Erfolg entspricht.

Bei drei Teilnehmern exisitieren $3! = 3 \cdot 2 \cdot 1 = 6$ Permutationen:

\begin{enumerate}[itemsep=0pt, parsep=0pt]
  \item Anna, Ben, Carla
  \item Anna, Carla, Ben
  \item Ben, Anna, Carla
  \item Carla, Anna, Ben
  \item Ben, Carla, Anna
  \item Carla, Ben, Anna
\end{enumerate}

Jede Permutation entspricht einer Koalitionsbildung. Anna wird in zwei Koalitionsbildungen (1. und 2.) einer leeren Koalition hinzugefügt.
In weiteren zwei Koalitionsbildungen (5. und 6.) wird Anna der bestehenden Koalition aus Ben und Carla hinzugefügt. 
In den beiden übrigen Koalitionsbildungen wird Anna einmal der Koaliton bestehend aus Ben (3.) und einmal der Koalition bestehend aus Carla (4.)
hinzugefügt. 

Daraus lassen sich nun die gewichteten durchschnittlichen marginalen Beiträge für Anna berechnen:

\[
  \frac{1}{6} ( \underbrace{2 \cdot 500 \text{\euro}}_{\text{A $\rightarrow$ $\{\emptyset$\}}} + \underbrace{1 \cdot 0 \text{\euro}}_{\text{A $\rightarrow$ $\{B$\}}} + \underbrace{1 \cdot 750 \text{\euro}}_{\text{A $\rightarrow$ $\{C$\}}} + \underbrace{2 \cdot 500 \text{\euro}}_{\text{A $\rightarrow$ $\{B, C$\}}} ) \approx 458,34 \text{\euro}  
\]

Analog gilt das für Ben:

\[
  \frac{1}{6} ( \underbrace{2 \cdot 750 \text{\euro}}_{\text{B $\rightarrow$ $\{\emptyset$\}}} + \underbrace{1 \cdot 250 \text{\euro}}_{\text{B $\rightarrow$ $\{A$\}}} + \underbrace{1 \cdot 500 \text{\euro}}_{\text{B $\rightarrow$ $\{C$\}}} + \underbrace{2 \cdot 250 \text{\euro}}_{\text{B $\rightarrow$ $\{A, C$\}}} ) \approx 458,34 \text{\euro}  
\]

und Carla:

\[
  \frac{1}{6} ( \underbrace{2 \cdot 0 \text{\euro}}_{\text{C $\rightarrow$ $\{\emptyset$\}}} + \underbrace{1 \cdot 250 \text{\euro}}_{\text{C $\rightarrow$ $\{A$\}}} + \underbrace{1 \cdot (-250) \text{\euro}}_{\text{C $\rightarrow$ $\{B$\}}} + \underbrace{2 \cdot 250 \text{\euro}}_{\text{C $\rightarrow$ $\{A, B$\}}} ) \approx 83,34 \text{\euro}  
\]

\section{Formale Definition}

Sei $\mathcal{N} = \{1, \ldots, n\}$ eine endliche Spielermenge mit $n := |\mathcal{N}|$ Elementen. Sei $v$ die \textbf{Koalitionsfunktion}, die jeder Teilmenge von $\mathcal{N}$ eine reele Zahl zuweist und insbesondere der leeren Koalition den Wert $0$ gibt. 

\[
\begin{array}{rcccl}
  v &:  &\mathcal P(\mathcal{N}) &\longrightarrow &\mathbb{R}\\
  &: &v(\emptyset) &\mapsto &0\\
\end{array}
\]

Eine nicht leere Teilmenge der Spieler $\mathcal{S} \subseteq \mathcal{N}$ heißt Koalition. $\mathcal{N}$ selbst bezichnet die große Koalition. Den Ausdruck $v(\mathcal{S})$ nennt man den Wert der Koalition $\mathcal{S}$.
Der Shapley-Wert ordnet nun jedem Spieler aus $\mathcal{N}$ eine Auszahlung für das Spiel $v$ zu.

Der marginale Beitag eines Spieler $i \in N$, also der Wertbeitrag eines Spielers zu einer Koalition $\mathcal{S} \subseteq \mathcal{N}$, durch seinen Beitritt, ist

\begin{equation*}
v(\mathcal{S} \cup \{i\}) - v(\mathcal{S}).
\end{equation*}

Der Shapley-Wert eines Spielers $i$ errechnet sich als das gewichtete Mittel der marginalen Beiträge zu allen möglichen Koalitionen:

\begin{equation*}
\varphi_i (\mathcal{N}, v) = \sum_{\mathcal{S} \subseteq \mathcal{N} \setminus \{i\}} \underbrace{\frac{|\mathcal{S}|! \cdot (n - 1 - |\mathcal{S}|)!}{n!}}_{\text{Gewicht}} \underbrace{v(\mathcal{S} \cup \{i\}) - v(\mathcal{S})}_{\substack{\text{marginaler Beitrag von} \\ \text{Spieler $i$ zur Koalition $\mathcal{S}$}}}.
\end{equation*}

\section{Axiome}

\paragraph{Pareto-Effizienz}

Der Wert der großen Koalition wird an die Spieler verteilt:

\begin{equation*}
\sum_{i \in \mathcal{N}} \varphi_i (\mathcal{N}, v) = v(\mathcal{N}).
\end{equation*}

\paragraph{Symmetrie}

Zwei Spieler $i$ und $j$, die die gleichen marginalen Beiträgen zu jeder Koalition haben,

\begin{equation*}
v(\mathcal{S} \cup \{i\}) = v(\mathcal{S} \cup \{j\})
\end{equation*}

erhalten das Gleiche:

\begin{equation*}
\varphi_i (\mathcal{N}, v) = \varphi_j (\mathcal{N}, v).
\end{equation*}

\paragraph{Null-Spieler-Eigenschaft}

Ein Spieler der zu jeder Koalition nichts bzw. den Wert seiner Einer-Koalition beiträgt, erhält null bzw. den Wert seiner Einer-Koalition:

\begin{equation*}
\varphi_i (\mathcal{N}, v) = 0,
\end{equation*}

bzw.

\begin{equation*}
\varphi_i (\mathcal{N}, v) = v(\{i\}).
\end{equation*}

\paragraph{Additivität}

Wenn das Spiel in zwei unabhängige Spiele zerlegt werden kann, dann ist die Auszahlung jedes Spielers im zusammengesetzten Spiel die Summe der Auszahlungen in den aufgeteilten Spielen:

\begin{equation*}
\varphi_i (\mathcal{N}, v + w) = \varphi_i (\mathcal{N}, v) + \varphi_i (\mathcal{N}, w).
\end{equation*}
