\chapter{Theorie der Shapley-Werte}

Durch die Verwendung eines praktischen Beispiels – der Aufteilung 
eines Preisgeldes aus einem Designwettbewerb unter den Gewinnern – wird zunächst eine intuitive Einführung 
in das Konzept gegeben. Anschließend wird die formale Definition der Shapley-Werte erläutert, um die 
theoretischen Grundlagen für ihre Berechnung und Anwendung zu legen.

\section{Wie lässt sich der Gewinn gerecht aufteilen?}

Angenommen, drei Teilnehmer, Anna, Ben und Carla, haben als Team kooperiert und den ersten Platz bei einem Designwettbewerb belegt\footnote{In Anlehnung an das Beispiel aus Kapitel 4.1 \glqq{}Who's going to pay for that taxi?\grqq{} \cite[S.17-20]{Molnar_2023}.}. 
Dieser Erfolg führt zu einem Gesamtgewinn von 1000 \euro. Das Preisgeld für den zweiten Platz beträgt 750 \euro{} und 500 \euro{} für den dritten Platz.
Die Herausforderung besteht nun darin, den Gewinn auf eine Weise zu verteilen, die den individuellen Beitrag jedes Teilnehmers 
zur Erzielung des ersten Platzes gerecht widerspiegelt.

Die Situation wird komplizierter, wenn man bedenkt, dass jeder Teilnehmer unterschiedlich zu dem Erfolg 
beigetragen hat und ihre individuellen Leistungen auch zu verschiedenen Ausgängen geführt hätten, 
wenn sie alleine oder in anderen Teilkonstellationen angetreten wären.

Um eine faire Aufteilung des Preisgeldes zu erreichen, betrachten wir die hypothetischen Gewinne, 
die Anna, Ben und Carla erzielt hätten, wenn sie in unterschiedlichen Konstellationen am Wettbewerb teilgenommen hätten.
Tabelle \ref{tab:shapley_example} zeigt die gegebene Gewinnverteilung der verschiedenen Koalitionen. Die Koalition $\emptyset$ entspricht
dabei der leeren Koalition -- der Nichtteilnahme an dem Wettbewerb.

\begin{table}[!h]
  \caption{Potenzielle Gewinne für verschiedene Teilnehmerkonstellationen im\\Designwettbewerb.}
  \footnotesize
  \begin{tabularx}{\textwidth}{Xrr}
  \toprule
  Koalition & Gewinn & Bemerkung \\
  \midrule
  $\emptyset$ & 0 \euro & Keine Teilnahme \\
  $\{$Anna$\}$ & 500 \euro & 3. Platz als Einzelteilnehmerin \\
  $\{$Ben$\}$ & 750 \euro & 2. Platz als Einzelteilnehmer \\
  $\{$Carla$\}$ & 0 \euro & Kein Gewinn als Einzelteilnehmerin \\
  $\{$Anna, Ben$\}$ & 750 \euro & 2. Platz als Team ohne Carla \\
  $\{$Anna, Carla$\}$ & 750 \euro & 2. Platz als Team ohne Ben \\
  $\{$Ben, Carla$\}$ & 500 \euro & 3. Platz als Team ohne Anna \\
  $\{$Anna, Ben, Carla$\}$ & 1000 \euro & 1. Platz als Gesamtteam \\
  \bottomrule
  \end{tabularx}
  \label{tab:shapley_example}
  \normalsize\\
  Quelle: Eigene Darstellung.
\end{table}

Zur Berechnung der Shapley-Werte ist es erforderlich, den marginalen Beitrag jedes Spielers zu erfassen.
Marginalbeiträge in der Spieltheorie, und speziell im Kontext der Shapley-Werte, sind die zusätzlichen Beiträge, 
die ein Spieler (Teilnehmer) zum Gesamtgewinn einer Koalition beiträgt, wenn er dieser beitritt. 
Die Berechnung des marginalen Beitrags eines Teilnehmers erfolgt, indem man den Wert der Koalition ohne diesen Teilnehmer 
vom Wert der Koalition mit dem Teilnehmer subtrahiert \cite[S. 18]{Molnar_2023}.

In diesem Beispiel mit Anna, Ben und Carla, die an einem Designwettbewerb teilnehmen, ist der marginale Beitrag von 
Anna zur Koalition von $\{$Ben$\}$ der zusätzliche Wert, den Anna einbringt, wenn sie sich Ben anschließt, 
ausgehend von Bens individuellem Gewinn.

\begin{table}[!h]
  \caption{Marginalbeiträge der einzelnen Teilnehmer zu den möglichen Koalitionen.}
  \footnotesize
  \begin{tabularx}{\textwidth}{XXrrr}
  \toprule
  Teilnehmer & Zur Koalition & Gewinn vorher & Gewinn nachher & Marginalbeitrag \\
  \midrule
  Anna & $\emptyset$ & 0 \euro & 500 \euro & 500 \euro \\
  Anna & $\{$Ben$\}$ & 750 \euro & 750 \euro & 0 \euro \\
  Anna & $\{$Carla$\}$ & 0 \euro & 750 \euro & 750 \euro \\
  Anna & $\{$Ben, Carla$\}$ & 500 \euro & 1000 \euro & 500 \euro\\
  Ben & $\emptyset$ & 0 \euro & 750 \euro & 750 \euro \\
  Ben & $\{$Anna$\}$ & 500 \euro & 750 \euro & 250 \euro\\
  Ben & $\{$Carla$\}$ & 0 \euro & 500 \euro & 500 \euro\\
  Ben & $\{$Anna, Carla$\}$ & 750 \euro &  1000 \euro & 250 \euro\\
  Carla & $\emptyset$ & 0 \euro & 0 \euro & 0 \euro \\
  Carla & $\{$Anna$\}$ & 500 \euro & 750 \euro & 250 \euro \\
  Carla & $\{$Ben$\}$ & 750 \euro & 500 \euro & -250 \euro \\
  Carla & $\{$Anna, Ben$\}$ & 750 \euro & 1000 \euro & 250 \euro \\
  \bottomrule
  \end{tabularx}
  \label{tab:shapley_marginal}
  \normalsize\\
  Quelle: Eigene Darstellung.
\end{table}

Tabelle \ref{tab:shapley_marginal} illustriert den Gewinn jeder möglichen Koalition ohne den 
betrachteten Spieler und den neuen Gesamtgewinn, sobald dieser Spieler der Koalition beitritt. 
Der marginale Beitrag jedes Spielers wird als Differenz zwischen diesen beiden Werten 
berechnet und gibt Aufschluss über den individuellen Wertbeitrag zum gemeinschaftlichen Erfolg.

Nachdem die marginalen Beiträge jedes Teilnehmers für die verschiedenen Koalitionen festgestellt wurden, 
ist der nächste Schritt, die Shapley-Werte zu bestimmen, welche eine faire Aufteilung des Gesamtgewinns 
erlauben. Hierzu wird jede mögliche Permutation betrachtet, in der die Spieler der 
Koalition beitreten könnten. Jede dieser Permutationen liefert unterschiedliche marginale Beiträge 
für die Spieler, je nach der Reihenfolge ihres Beitritts, wie Tabelle \ref{tab:shapley_marginal} zeigt \cite[S. 19]{Molnar_2023}.

Im Falle des Beispiels mit Anna, Ben und Carla bedeutet dies, dass alle möglichen Reihenfolgen 
berücksichtigt werden müssen, in denen sie zum ersten Platz beigetragen haben könnten. 
Die Shapley-Werte werden als Durchschnitt der marginalen Beiträge über alle Permutationen berechnet. 
Dies gewährleistet, dass jeder Spieler einen Anteil des Preisgeldes erhält, der seinem durchschnittlichen 
Beitrag zum Erfolg entspricht \cite[S. 20]{Molnar_2023}.

Bei drei Teilnehmern exisitieren $3! = 3 \cdot 2 \cdot 1 = 6$ Permutationen:

\begin{enumerate}[itemsep=0pt, parsep=0pt]
  \item Anna, Ben, Carla
  \item Anna, Carla, Ben
  \item Ben, Anna, Carla
  \item Carla, Anna, Ben
  \item Ben, Carla, Anna
  \item Carla, Ben, Anna
\end{enumerate}

Jede Permutation entspricht einer Koalitionsbildung. Anna wird in zwei Koalitionsbildungen (1. und 2.) einer leeren Koalition hinzugefügt, da Sie die erste ist, 
die der Koalition beitritt.
In weiteren zwei Koalitionsbildungen (5. und 6.) wird Anna der bestehenden Koalition aus Ben und Carla, respektive Carla und Ben hinzugefügt. 
In den beiden übrigen Koalitionsbildungen wird Anna einmal der Koaliton bestehend aus Ben (3.) und einmal der Koalition bestehend aus Carla (4.)
hinzugefügt. 

Zusammen mit Tabelle \ref{tab:shapley_marginal} lässt sich nun der Shapley-Wert mit den gewichteten durchschnittlichen marginalen Beiträge für Anna berechnen:

\begin{equation}
  \frac{1}{6} ( \underbrace{2 \cdot 500 \text{\euro}}_{\text{A $\rightarrow$ $\{\emptyset$\}}} + \underbrace{1 \cdot 0 \text{\euro}}_{\text{A $\rightarrow$ $\{B$\}}} + \underbrace{1 \cdot 750 \text{\euro}}_{\text{A $\rightarrow$ $\{C$\}}} + \underbrace{2 \cdot 500 \text{\euro}}_{\text{A $\rightarrow$ $\{B, C$\}}} ) \approx 458,33 \text{\euro}.  
\label{eq:marginal_anna}
\end{equation}

Analog gilt dies für Ben:

\begin{equation}
  \frac{1}{6} ( \underbrace{2 \cdot 750 \text{\euro}}_{\text{B $\rightarrow$ $\{\emptyset$\}}} + \underbrace{1 \cdot 250 \text{\euro}}_{\text{B $\rightarrow$ $\{A$\}}} + \underbrace{1 \cdot 500 \text{\euro}}_{\text{B $\rightarrow$ $\{C$\}}} + \underbrace{2 \cdot 250 \text{\euro}}_{\text{B $\rightarrow$ $\{A, C$\}}} ) \approx 458,33 \text{\euro},  
  \label{eq:marginal_ben}
\end{equation}

und Carla:

\begin{equation}
  \frac{1}{6} ( \underbrace{2 \cdot 0 \text{\euro}}_{\text{C $\rightarrow$ $\{\emptyset$\}}} + \underbrace{1 \cdot 250 \text{\euro}}_{\text{C $\rightarrow$ $\{A$\}}} + \underbrace{1 \cdot (-250) \text{\euro}}_{\text{C $\rightarrow$ $\{B$\}}} + \underbrace{2 \cdot 250 \text{\euro}}_{\text{C $\rightarrow$ $\{A, B$\}}} ) \approx 83,33 \text{\euro}  
  \label{eq:marginal_carla}
\end{equation}

Auf Basis der gewichteten durchschnittlichen marginalen Beiträge lässt sich feststellen, 
dass Anna und Ben jeweils einen Shapley-Wert von ungefähr 458,33 \euro{} erhalten, 
während Carla einen Shapley-Wert von etwa 83,33 \euro{} zugewiesen bekommt. 
Diese Werte spiegeln den fairen Anteil jedes Teilnehmers an der Gesamtprämie wider, 
basierend auf ihrem individuellen Beitrag zum Erfolg des Teams \cite[S. 20]{Molnar_2023}. Mit dieser konkreten Anwendung der Shapley-Werte 
auf ein alltagsnahes Beispiel wird nun die zugrunde liegende Theorie und die formale Definition 
der Shapley-Werte, die diese Berechnungen ermöglichen, detaillierter betrachtet.

\section{Formale Definition}

Sei $\mathcal{N} = \{1, \ldots, n\}$ eine endliche Spielermenge mit $n := |\mathcal{N}|$ Elementen. Sei $v$ die Koalitionsfunktion, die jeder Teilmenge von $\mathcal{N}$ eine reelle Zahl zuweist und insbesondere der leeren Koalition den Wert $0$ gibt. 

\[
\begin{array}{rcccl}
  v &:  &\mathcal P(\mathcal{N}) &\longrightarrow &\mathbb{R}\\
  &: &v(\emptyset) &\mapsto &0\\
\end{array}
\]

Eine nicht leere Teilmenge der Spieler $\mathcal{S} \subseteq \mathcal{N}$ heißt Koalition. $\mathcal{N}$ selbst bezeichnet die große Koalition. Den Ausdruck $v(\mathcal{S})$ nennt man den Wert der Koalition $\mathcal{S}$.
Der Shapley-Wert ordnet nun jedem Spieler aus $\mathcal{N}$ eine Auszahlung für das Spiel $v$ zu.

Der marginale Beitag eines Spielers $i \in \mathcal{N}$, also der Wertbeitrag eines Spielers zu einer Koalition $\mathcal{S} \subseteq \mathcal{N}$, durch seinen Beitritt, ist

\begin{equation}
v(\mathcal{S} \cup \{i\}) - v(\mathcal{S}).
\label{eq:marignal}
\end{equation}


Sei $i = \text{Anna}$ und $\mathcal{S} = \{\text{Ben}\}$, dann ist $v(\mathcal{\{\text{Ben}\}} \cup \{\text{Anna}\}) - v(\mathcal{\{\text{Ben}\}})$ das
zusätzliche Preisgeld, welches gewonnen wird, wenn Anna der Koalition mit Ben beitritt. 


Der Shapley-Wert eines Spielers $i$ errechnet sich als das gewichtete Mittel der marginalen Beiträge zu allen möglichen Koalitionen:

\begin{equation}
\varphi_i (\mathcal{N}, v) = \sum_{\mathcal{S} \subseteq \mathcal{N} \setminus \{i\}} \underbrace{\frac{|\mathcal{S}|! \cdot (n - 1 - |\mathcal{S}|)!}{n!}}_{\text{Gewicht}} \underbrace{v(\mathcal{S} \cup \{i\}) - v(\mathcal{S})}_{\substack{\text{marginaler Beitrag von} \\ \text{Spieler $i$ zur Koalition $\mathcal{S}$}}}.
\end{equation}

Die Summationsnotation \(\sum_{\mathcal{S} \subseteq \mathcal{N} \setminus \{i\}}\) erfasst die marginalen Beiträge, 
die der Spieler \( i \) zu allen Koalitionen leistet, welche diesen noch nicht einschließen. Die Verwendung von 
\(\mathcal{N} \setminus \{i\}\) stellt sicher, dass Spieler \( i \) nur für jene Koalitionen berücksichtigt wird, 
zu denen er noch beitragen kann. Im Falle von Anna etwa, beziehen sich die Berechnungen auf die Koalitionen bestehend 
aus der leeren Koalition \(\emptyset\), aus \(\{\text{Ben}\}\), \(\{\text{Carla}\}\), oder beiden 
zusammen \(\{\text{Ben, Carla}\}\) (vgl. Berechnung \ref{eq:marginal_anna}).

Die Formel \(\frac{|\mathcal{S}|! \cdot (n - 1 - |\mathcal{S}|)!}{n!}\) in der Shapley-Wert-Berechnung 
reflektiert den Gewichtungsfaktor für die marginalen Beiträge eines Spielers. Hierbei gibt \(|\mathcal{S}|!\) die Permutationen 
der Spieler innerhalb der Koalition \(\mathcal{S}\) an, während \((n - 1 - |\mathcal{S}|)!\) die Anordnungen der 
außenstehenden Spieler repräsentiert, nachdem der betrachtete Spieler beigetreten ist. 
Der Bruchteil \(\frac{1}{n!}\) normalisiert diesen Wert über alle möglichen Koalitionszusammensetzungen, 
wodurch die Wahrscheinlichkeit der Bildung einer spezifischen Koalition ausgedrückt wird.

Betrachten wir Anna als den Spieler $i$ und die Koalition \(\mathcal{S} = \{\text{Ben, Carla}\}\). 
Die Formel \(\frac{|\mathcal{S}|! \cdot (n - 1 - |\mathcal{S}|)!}{n!}\) berechnet den Gewichtungsfaktor 
für Annas marginalen Beitrag zur Koalition \(\mathcal{S}\). In diesem Fall ist \(|\mathcal{S}| = 2\) und \(n = 3\). 
Somit ergibt sich \(|\mathcal{S}|! = 2!\) und \(n - 1 - |\mathcal{S}| = 0!\), da nach dem Beitritt von 
Anna keine weiteren Spieler übrig sind. Der Normalisierungsfaktor ist \(n! = 3! = 6\). Daraus folgt:

\begin{equation}
\frac{2! \cdot 0!}{3!} = \frac{2 \cdot 1}{6} = \frac{1}{3}.
\end{equation}

Dies bedeutet, dass unter allen möglichen Permutationen der Spielerreihenfolge, Annas Beitritt zu der Koalition \{Ben, Carla\} 
in genau ein Drittel der Fälle geschieht. Somit wird ihr marginaler Beitrag mit diesem Faktor gewichtet, 
um den Shapley-Wert zu berechnen (vgl. Berechnung \ref{eq:marginal_anna}) \cite[S. 21f]{Molnar_2023}.


\section{Axiomatische Grundlage}
\label{sec:axiome-shapley}

Nachdem die Berechnung des Shapley-Werts für das Beispiel konkretisiert wurde, ist es nun von Bedeutung, die zugrundeliegenden Axiome zu betrachten, 
welche die theoretische Rechtfertigung für die Methode liefern. Der Shapley-Wert wird nicht nur durch seine Berechnungsmethode, 
sondern auch durch eine Reihe von Axiomen charakterisiert, die seine Fairness und Kohärenz im Kontext kooperativer Spiele sicherstellen. 
Lloyd Shapley leitete den Shapley-Wert ursprünglich aus diesen Axiomen ab und bewies, 
dass dieser der einzige ist, der den Axiomen genügt\footnote{Eine detaillierte Darstellung dieser Axiome und des Beweises 
ihrer Einzigartigkeit findet sich in Shapleys Originalarbeit, deren umfassende Behandlung jedoch den Rahmen dieser Arbeit überschreiten würde \cite[S. 307-318]{Shapley+1953+307+318}.}. 
Diese Axiome sind wesentliche Bestandteile, die die Einzigartigkeit und die wünschenswerten Eigenschaften des Shapley-Werts als Lösungskonzept definieren \cite[S. 22]{Molnar_2023}. 

\paragraph{Effizienz}

Der Wert der großen Koalition wird an die Spieler verteilt:

\begin{equation}
\sum_{i \in \mathcal{N}} \varphi_i (\mathcal{N}, v) = v(\mathcal{N}).
\end{equation}

Dies bedeutet, dass die Summe der Shapley-Werte aller Spieler dem Gesamtwert entspricht, 
den die Koalition aller Spieler zusammen erreichen kann. Der Gesamtwert, den die große Koalition $\mathcal{N}$, 
bestehend aus Anna, Ben und Carla, generiert, wird komplett unter den Spielern aufgeteilt \cite[S. 22]{Molnar_2023}. 
Unter Vernachlässigung minimaler Rundungsdifferenzen entspricht die Summe der Shapley-Werte, 
berechnet in den Gleichungen \ref{eq:marginal_anna}, \ref{eq:marginal_ben} und \ref{eq:marginal_carla}, 
dem kollektiven Ertrag der großen Koalition:

\begin{equation}
  458,33 \text{\euro} + 458,33 \text{\euro} + 83,33 \text{\euro} \approx 1000 \text{\euro} 
\end{equation}


\paragraph{Symmetrie}

Zwei Spieler $i$ und $j$, die die gleichen marginalen Beiträgen zu jeder Koalition haben, bekommen den gleichen Wert
zugewiesen:

\begin{equation}
v(\mathcal{S} \cup \{i\}) = v(\mathcal{S} \cup \{j\}), \; \forall\, \mathcal{S} \subseteq \mathcal{N} \setminus \{i, j\} \Rightarrow \varphi_i (\mathcal{N}, v) = \varphi_j (\mathcal{N}, v).
\end{equation}

Obwohl Anna und Ben den gleichen Shapley-Wert erhalten, ist dies nicht auf das Symmetrieaxiom zurückzuführen, 
da ihre marginalen Beiträge zu den Koalitionen variieren. Zum Beispiel leistet Anna keinen Beitrag zur Koalition, 
wenn Ben bereits Teil davon ist, während Ben einen positiven Beitrag leistet, 
wenn Anna bereits zur Koalition gehört (vgl. Tabelle \ref{tab:shapley_marginal}). Dies zeigt, dass die Gleichheit ihrer Shapley-Werte ein Ergebnis der 
spezifischen Zahlenkonstellation in diesem Szenario ist und nicht aus der symmetrischen Interaktion 
zwischen den beiden Spielern resultiert.

\paragraph{Null-Spieler-Eigenschaft (Dummy-Spieler-Eigenschaft)}

Ein Spieler $i$, der zu jeder Koalition nichts beiträgt, erhält den Wert Null:

\begin{equation}
  v(\mathcal{S} \cup \{i\}) =  v(\mathcal{S}), \; \forall\, \mathcal{S} \subseteq \mathcal{N} \setminus \{i\} \Rightarrow \varphi_i (\mathcal{N}, v) = 0.
\end{equation}

Dies stell sicher, dass ein Spieler, der keinen Beitrag leistet, auch nicht belohnt wird. 

\paragraph{Additivität} 

Wenn das Spiel in zwei unabhängige Spiele mit Koalitionsfunktionen $v$ und $w$ zerlegt werden kann, dann ist die Auszahlung jedes Spielers im 
zusammengesetzten Spiel die Summe der Auszahlungen in den aufgeteilten Spielen:

\begin{equation}
\varphi_i (\mathcal{N}, v + w) = \varphi_i (\mathcal{N}, v) + \varphi_i (\mathcal{N}, w).
\end{equation}

Wenn Anna, Ben und Carla neben dem ersten Wettbewerb an einem zweiten, unabhängigen Wettbewerb teilnehmen, 
besagt das Additivitätsaxiom, dass die Shapley-Werte jedes Spielers aus beiden Wettbewerben einfach die Summe ihrer individuellen 
Shapley-Werte aus jedem einzelnen Wettbewerb sind. Dies impliziert, dass die faire Aufteilung der Gewinne aus beiden Wettbewerben 
konsistent bleibt, indem die aus dem ersten Wettbewerb abgeleiteten Prinzipien auf den zweiten Wettbewerb übertragen und 
dann addiert werden \cite[S. 5573, S.22f]{ijcai2022p778, Molnar_2022}.