\chapter{Historischer Kontext und Begriffsdefinitionen}



\section{Die Genese der Shapley-Werte in der kooperativen Spieltheorie}

Der Ursprung der Shapley-Werte liegt in der kooperativen Spieltheorie, einem fundamentalen Zweig der Spieltheorie. 
Dieser Bereich beschäftigt sich mit der Analyse von Situationen, in denen Akteure zusammenarbeiten, um gemeinsame Ziele zu erreichen. 
Zentrales Anliegen ist dabei die gerechte Verteilung der entstehenden Gewinne unter den Akteuren. Ein Schlüsselkonzept dieser Theorie 
ist die sogenannte \glqq{}Charakteristische Funktion\grqq{}, welche die Bewertung der Gewinnverteilung einer Koalition von Akteuren ermöglicht.

Die Shapley-Werte, entwickelt von Lloyd Shapley in den 1950er Jahren, bieten einen methodischen Ansatz, um den individuellen Beitrag 
eines jeden Akteurs zur kooperativen Zusammenarbeit gerecht zu bewerten. Dies geschieht durch die Durchschnittsbewertung der Beiträge 
über sämtliche mögliche Koalitionen hinweg. Diese Methode erweist sich als äußerst nützlich, um eine gerechte und rationale Verteilung 
von Gewinnen in vielfältigen Szenarien zu ermöglichen, sei es in wirtschaftlichen Verhandlungen oder der Aufteilung von Ressourcen.

Das Verständnis der kooperativen Spieltheorie und ihrer Anwendung in Form der Shapley-Werte ermöglicht es, dieses theoretische Konzept 
auf den Bereich des maschinellen Lernens zu übertragen. In dieser Arbeit wird der Übergang von abstrakten Spieltheorie-Konzepten 
zu konkreten Anwendungen in der Welt der datengetriebenen Modelle erforscht.

Zur Erreichung dieses Ziels werden in den kommenden Abschnitten nicht nur die formalen Definitionen und Eigenschaften der Shapley-Werte 
erläutert, sondern auch ihre Adaption und Anwendung auf Machine Learning-Modelle in Betracht gezogen. Die Anwendbarkeit wird durch die 
praktische Anwendung auf einen realen Datensatz verdeutlicht.


\section{Shapley-Werte, SHAP, SHAP-Werte und \textsf{shap}}

Zur Verdeutlichung und Abgrenzung der verschiedenen, jedoch verwandten Begrifflichkeiten, die im Kontext dieser Arbeit Verwendung finden, 
ist eine kurze Einordnung essenziell.

Beginnend mit den Shapley-Werten, entstammt dieser Begriff der kooperativen Spieltheorie und beschreibt eine Methode, 
um den fairen Beitrag eines Spielers zu der Gesamtauszahlung eines kooperativen Spiels zu bestimmen. 

SHAP (SHapley Additive exPlanations) ist ein Interpretationsframework, das die Shapley-Werte in den Bereich des maschinellen Lernens überträgt. 
Der Begriff wurde erstmals von Lundberg und Lee eingeführt \cite[S. 1]{NIPS2017_8a20a862}.

Die SHAP-Werte sind dann die konkreten quantitativen Beiträge der einzelnen Merkmale zu einer bestimmten Vorhersage, 
berechnet basierend auf dem SHAP-Framework.

Das Python-Paket \textsf{shap} schließlich ist eine Implementierung, die es praktikabel macht, SHAP-Werte in der Anwendung zu berechnen 
und zu visualisieren. Es stellt eine reiche Auswahl an Werkzeugen zur Verfügung, um diese Werte und ihre Auswirkungen zu interpretieren.