% ##################################################
% Unterstuetzung fuer die deutsche Sprache
% ##################################################
%\usepackage{ngerman}
\usepackage[ngerman, english]{babel}
\usepackage[UKenglish]{datetime}
\usepackage{graphicx}
\usepackage{wrapfig}

% ##################################################
% Dokumentvariablen
% ##################################################

% Persoenliche Daten
\newcommand{\docNachname}{Symhoven}
\newcommand{\docVorname}{Simon}
\newcommand{\docStrasse}{Boschetsriederstraße 59A}
\newcommand{\docOrt}{München}
\newcommand{\docPlz}{D-81379}
\newcommand{\docEmail}{simon.symhoven@hm.edu}
\newcommand{\docMatrikelnummer}{49651418}

\newcommand{\docFakultaet}{Fakultät Für Informatik und Mathematik}
\newcommand{\docSchwerpunkt}{SCHWERPUNKT}
\newcommand{\docInstitut}{INSTITUT}

% Dokumentdaten
\newcommand{\docTitle}{Interpretation linearer Modelle mit SHAP}
\newcommand{\docUntertitle}{Interpretation of linear models with SHAP}
% Arten der Arbeit: Bachelorthesis, Masterthesis, Seminararbeit, Diplomarbeit
\newcommand{\docArtDerArbeit}{Master-Thesis}
%Studiengaenge: Allgemeine Informatik Bachelor, Computer Networking Bachelor,
% Software-Produktmanagement Bachelor, Advanced Computer Scinece Master
\newcommand{\docStudiengang}{Stochastic Engineering in Business and Finance}
\newcommand{\docAbgabedatum}{\today}
\newcommand{\docErsterReferent}{Prof. Dr. Andreas Zielke}
\newcommand{\docZweiterReferent}{-} % Wenn es nur einen Betreuer gibt
%\newcommand{\docZweiterReferent}{Dr. Tobias Redele}

% ##################################################
% Allgemeine Pakete
% ##################################################

% Abbildungen einbinden
\usepackage{graphicx}

% Zusaetsliche Sonderzeichen
\usepackage{dingbat}

% Symbole Haken und X [OPTIONAL]
%\usepackage{pifont}
%\newcommand{\cmark}{\ding{51}}
%\newcommand{\xmark}{\ding{55}}

% Farben
\usepackage{color}
\usepackage[usenames,dvipsnames,svgnames,table]{xcolor}

% Maskierung von URLs und Dateipfaden
\usepackage[hyphens]{url}
% Deutsche Anfuehrungszeichen
\usepackage[babel, german=quotes]{csquotes}

% Pakte zur Index-Erstellung (Schlagwortverzeichnis)
\usepackage{index}
\makeindex

% AMS Packages
\usepackage{amsmath}
\usepackage{amsthm}
\usepackage{amssymb}

% Ipsum Lorem
% Paket wird nur für das Beispiel gebraucht und kann gelöscht werden
\usepackage{lipsum}

% ##################################################
% Seitenformatierung
% ##################################################
\usepackage[
	portrait,
	bindingoffset=1.5cm,
	inner=2.5cm,
	outer=3cm,
	top=3cm,
	bottom=2cm,
	%showframe, %Aktivieren um Seitengrenzen anzuzeigen
	%includeheadfoot
	]{geometry}

% ##################################################
% Kopf- und Fusszeile
% ##################################################

\usepackage{fancyhdr}

\pagestyle{fancy}
\fancyhf{}
\fancyhead[EL,OR]{\rmfamily\thepage}
\fancyhead[ER,OL]{\rmfamily\nouppercase{\leftmark}}

\fancypagestyle{headings}{}

\fancypagestyle{plain}{}

\fancypagestyle{empty}{
  \fancyhf{}
  \renewcommand{\headrulewidth}{0pt}
}

%Speichert \chaptermark in \oldchaptermark damit 
% es für die Anhänge zurückgesetzt werden kann
\let\oldchaptermark\chaptermark

%Kein "Kapitel # NAME" in der Kopfzeile
\renewcommand{\chaptermark}[1]{
	\markboth{#1}{}
   	\markboth{\thechapter.\ #1}{}
}

% ##################################################
% Schriften
% ##################################################

% Stdandardschrift festlegen
\renewcommand{\familydefault}{\rmdefault}

% Standard Zeilenabstand: 1,5 zeilig
\usepackage{setspace}
\onehalfspacing 

% Schriftgroessen festlegen
\addtokomafont{chapter}{\rmfamily\large\bfseries} 
\addtokomafont{section}{\rmfamily\normalsize\bfseries} 
\addtokomafont{subsection}{\rmfamily\normalsize\mdseries} 
\addtokomafont{caption}{\rmfamily\normalsize\mdseries} 

%Einrücken von Absätzen deaktivieren
\setlength{\parindent}{0pt}

%Zeilenabstand bei abstätzen
\usepackage{parskip}

% ##################################################
% Inhaltsverzeichnis / Allgemeine Verzeichniseinstellungen
% ##################################################

\usepackage{tocloft}
\urlstyle{same}

% Punkte auch bei Kapiteln
\renewcommand{\cftchapdotsep}{3}
\renewcommand{\cftdotsep}{3}

% Schriftart und -groesse im Inhaltsv erzeichnis anpassen
\renewcommand{\cfttoctitlefont}{\rmfamily\large\bfseries}
\renewcommand{\cftloftitlefont}{\rmfamily\large\bfseries}
\renewcommand{\cftlottitlefont}{\rmfamily\large\bfseries}

\renewcommand{\cftchapfont}{\rmfamily\normalsize}
\renewcommand{\cftsecfont}{\rmfamily\normalsize}
\renewcommand{\cftsubsecfont}{\rmfamily\normalsize}
\renewcommand{\cftchappagefont}{\rmfamily\normalsize}
\renewcommand{\cftsecpagefont}{\rmfamily\normalsize}
\renewcommand{\cftsubsecpagefont}{\rmfamily\normalsize}

%Zeilenabstand in den Verzeichnissen einstellen
\setlength{\cftparskip}{.5\baselineskip}
\setlength{\cftbeforechapskip}{.1\baselineskip}

%Einrücken von Absätzen deaktivieren
%\setlength{\parindent}{0pt}

%Zeilenabstand bei abstätzen
\usepackage{parskip}

% ##################################################
% Abbildungsverzeichnis und Abbildungen
% ##################################################

\usepackage{caption}

\usepackage{wrapfig}

% Nummerierung von Abbildungen
\renewcommand{\thefigure}{\arabic{figure}}
\usepackage{chngcntr}
\counterwithout{figure}{chapter}
\counterwithout{footnote}{chapter}

% Abbildungsverzeichnis anpassen
\renewcommand{\cftfigpresnum}{Abbildung }
\renewcommand{\cftfigaftersnum}{:}

% Breite des Nummerierungsbereiches [Abbildung 1:]
\newlength{\figureLength}
\settowidth{\figureLength}{\bfseries\cftfigpresnum\cftfigaftersnum}
\addtolength{\figureLength}{2mm} %extra offset
\setlength{\cftfignumwidth}{\figureLength}
\setlength{\cftfigindent}{0cm}

% Schriftart anpassen
\renewcommand\cftfigfont{\rmfamily}
\renewcommand\cftfigpagefont{\rmfamily}

%standardpfad anpassen
\graphicspath{ {../src/content/pictures/} }

% ##################################################
% Tabellenverzeichnis und Tabellen
% ##################################################

% Nummerierung von Tabellen
\renewcommand{\thetable}{\arabic{table}}
\counterwithout{table}{chapter}

% Tabellenverzeichnis anpassen
\renewcommand{\cfttabpresnum}{Tabelle }
\renewcommand{\cfttabaftersnum}{:}

% Breite des Nummerierungsbereiches [Abbildung 1:]
\newlength{\tableLength}
\settowidth{\tableLength}{\bfseries\cfttabpresnum\cfttabaftersnum}
\addtolength{\tableLength}{3mm} %extra offset
\setlength{\cfttabnumwidth}{\tableLength}
\setlength{\cfttabindent}{0cm}

%Schriftart anpassen
\renewcommand\cfttabfont{\rmfamily}
\renewcommand\cfttabpagefont{\rmfamily}

% Unterdrueckung von vertikalen Linien
\usepackage{booktabs}
\usepackage{eurosym}
%Multi row für spezifische zellen
\usepackage{multirow}
\usepackage{enumitem}

%Additional table package
\usepackage{tabu}
\usepackage{tabularx}

% ##################################################
% Listings (Quellcode)
% ##################################################

\usepackage{listings}

%use typewriter font which supports bold characters
\usepackage{beramono}

\definecolor{codegreen}{rgb}{0,0.6,0}
\definecolor{codegray}{rgb}{0.5,0.5,0.5}
\definecolor{codepurple}{rgb}{0.5,0,0.33}
\definecolor{codepurblue}{rgb}{0.16,0.0,1.0}
\definecolor{backcolour}{rgb}{0.95,0.95,0.92}

\lstdefinestyle{codestyle}{
    backgroundcolor=\color{backcolour},   
    commentstyle=\color{codegreen},
    keywordstyle=\bfseries\color{codepurple},
    numberstyle=\tiny\color{codegray},
    stringstyle=\color{codepurblue},
    basicstyle=\scriptsize\ttfamily,
    breakatwhitespace=false,         
    breaklines=true,                 
    captionpos=b,                    
    keepspaces=true,                 
    numbers=left,                     
    numbersep=5pt,                 
    showspaces=false,                
    showstringspaces=false,
    showtabs=false,                  
    tabsize=2
}

\lstset{style=codestyle}

%Code auschnitt importieren aus datei
%\mylisting{from}{to}{language}{file}{descr}{path}
\newcommand{\mylisting}[6]{
\lstinputlisting[language=#3,
				firstnumber=#1,
				firstline=#1,
				lastline=#2,
				caption={#4, #5}, 
				label={implementation_listing_#4_#5}]
				{#6}
}

% ##################################################
% Appendix
% ##################################################

%Calc packet für berechnungen
\usepackage{calc}
\usepackage{float}
%Appendix paket, setzen der flags für das TOC
\usepackage[toc,title,titletoc]{appendix} 

%Umbenennen der überschrift für die Anhänge 
\renewcommand{\appendixtocname}{Anhänge}

%Befehl für einen neuen Bericht und die erste seite als bild
\newcommand{\appendixsection}[2]{
\section{#1}
\appendixsingle{#2}
}

%Befehl für einzelne seite als bild eingefasst, damit überschrift und kopfzeile
% bestehen bleibt. 
\newcommand{\appendixsingle}[1]{
\vspace{-10cm}
\vfill
\mbox{}\hspace{-1.5cm}\includegraphics[width=\linewidth+3cm]{#1}\hspace{-1.5cm}\mbox{}
\vspace{-10cm}
\vfill
\mbox{}
}

%Datenträger Tabelle
\definecolor{lightgray}{gray}{0.85}
\definecolor{ultralightgray}{gray}{0.95}
\definecolor{mygray}{gray}{0.70}

% ##################################################
% Theoreme
% ##################################################
  	
% Umgebung fuer Beispiele
\newtheorem{beispiel}{Beispiel}

% Umgebung fuer These
\newtheorem{these}{These}

% Umgebung fuer Definitionen
\newtheorem{definition}{Definition}
  	
% ##################################################
% Literaturverzeichnis
% ##################################################

% ##################################################
% Abkuerzungsverzeichnis
% ##################################################

\usepackage[printonlyused]{acronym}

% ##################################################
% PDF / Dokumenteninternelinks
% ##################################################

\usepackage[
	  colorlinks=false,
   	linkcolor=black,
   	citecolor=black,
  	filecolor=black,
	  urlcolor=black,
    bookmarks=true,
    bookmarksopen=true,
    bookmarksopenlevel=3,
    bookmarksnumbered,
    plainpages=false,
    pdfpagelabels=true,
    hyperfootnotes,
    pdftitle ={\docTitle},
    pdfauthor={\docVorname~\docNachname},
    pdfcreator={\docVorname~\docNachname},
    hidelinks]{hyperref}

% ####################################################
% Command für einfache QUellenangabe bei Bilder, etc.
% ####################################################

\newcommand{\source}[1]{\caption*{Quelle: {#1}} }

\usepackage[intoc]{nomencl}
\makenomenclature